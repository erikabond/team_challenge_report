\section{Citizen engagement}\index{Citizen engagement}

    \subsection{Importance of citizen engagement} \label{Citizen engagement}
    %Author:  - Sarah
    Citizen engagement is an essential component of any research project. It promotes awareness, education and interaction with members of the public, formation professional collaborations, and provides a means to gain funding. Public engagement is, by definition, a two-way process \cite{Public_engagement_definition}. Firstly, it serves to benefit the public by involving and educating them in a particular area of scientific research. Secondly, it serves to benefit the researcher by gaining insight into a specific user need, or as a means to collect larger, more diverse and representative data sets. In accordance with this definition, the team outlined key objectives prior to conducting any outreach events. \Cref{fig:vendiagram} was used to ensure that our project's overall public engagement would be mutually beneficial. There were three main events planned in order to fulfil the above objectives: Immerse Cambridge Summer School, MakeSpace Make-a-thon, and Agri-Tech East Hack-a-thon. These are discussed further in the following sections. 
    
    \begin{figure}
        \centering
        \includegraphics[width=0.8\linewidth]{Pictures/Outreach/vendiagram.png}
        \caption{Key objectives for public engagement events. These were proposed in order to obtain a balanced approach to public engagement and ensure mutual benefit for SoliCamb and the end user.}
        \label{fig:vendiagram}
    \end{figure}
    
       % \begin{table}[h!]
        %\resizebox{\textwidth}{!}{%
        %\begin{tabular}{@{}lllll@{}}
        %\toprule
        %Objective & SoliCamb’s gain & Audience gain & Service &  \\ \midrule
        %Promote STEM in teenage populations &  & X & X &  \\
        %Educate the public &  & X & X &  \\
        %\begin{tabular}[c]{@{}l@{}}Raise awareness of\\ the global soil health problem\end{tabular} & X & X & X &  \\
        %Provide an open-source platform & X & X &  &  \\
        %Understand user need (farmers, gardeners, academics) & X & X &  &  \\
        %Generate data for device and model validity & X &  &  &  \\
        %Generate user feedback on device design and usability & X & X &  &  \\ \bottomrule
        %\end{tabular}%
        %}
        %\caption{\small Key objectives to be met with public engagement events. Meeting these targets %will ensure SoliCamb's project's engagement is mutually beneficial for parties involved. }
        %\label{tab:public_engagement}
        %\end{table} 

 %\begin{table}[h!]
        %\resizebox{\textwidth}{!}{%
  %      \centering
   %     \begin{tabular}{@{}lcccc@{}}
    %    \toprule
     %   Objective & SoliCamb & Audience & Service &  \\ \midrule
     %   Promote STEM  &  & $\surd$ & $\surd$ &  \\
      %  Education &  & $\surd$ & $\surd$ &  \\
      %  \begin{tabular}[c]{@{}l@{}}Raise awareness \end{tabular} & $\surd$ & $\surd$ & $\surd$ &  \\
      %  Provide open-source platform & $\surd$ & $\surd$ &  &  \\
       % Understand users' need & $\surd$ & $\surd$ &  &  \\
    %    Generate data and model validity & $\surd$ &  &  &  \\
     %   Obtain feedback & $\surd$ & $\surd$ &  &  \\ \bottomrule
      %  \end{tabular}%
        
       % \caption{Key objectives for public engagement events. These were designed to provide a balanced approach to public engagement and ensure mutual benefit for SoliCamb and the end user.}
        %\label{tab:public_engagement}
        %\end{table} 
    
   
    
   %citizen science - 
   In contrast to public engagement, which has a symbiotic element, citizen science can be carried out without any obvious participant gain, other than sheer interest and enjoyment. Although there is yet to be an international consensus of what encompasses \enquote{citizen science} \cite{heigl2019opinion}, the Oxford English dictionary definition is \enquote{scientific work undertaken by members of the general public, often in collaboration with or under the direction of professional scientists and scientific institutions}. Participation from citizen scientists is hugely beneficial for the purpose of developing the team's project. Firstly, for understanding if the device is user-friendly and fit for in-field operation by a non-expert. Secondly, as the technology is open-source, the team hoped to benefit from interactions and collaborations with a network of global citizen scientists, who are interested in this field of work. For the purpose of this project, SoliCamb defined target citizen scientists as non-experts, but someone who is interested in either the design and functionality, or using and testing the device. Whilst some of SoliCamb's public engagement events did have elements of citizen science, separate events solely focused on citizen science were conducted at Anglesey Abbey and Radwell Grange Farm (see \cref{Sample Sites}).
   
 
    \subsection{Professional collaborators} \label{Collaborators}
    % Elena: List the main sets/categories of groups/people we reach out to and those that replied and we successfully collaborated with (with contact list in appendix) 
    %Elena can I write this with you if you wanted to take the lead and let me know what to fillin :)  (Katie) ,
    Throughout the duration of the Team Challenge, SoliCamb has contacted multiple academics and other experts in the field of soil science, farming, geology, sensor technologies, as well as data management and visualisation. The main goal of this was to understand the field of soil health and soil health monitoring better, and to estimate what is feasible to achieve within the scope of the project, in regards to budget and time. The main questions were targeted at the end user and focused on technology currently on the market. What is being used at the moment? Where are drawbacks with these methods? What does not exist yet but would be of use for the end user? Which technologies or tests are too expensive or too complicated to use? Which field tests are not quite accurate enough as of yet? 
    Both input and feedback from our contacts have affected the decision making process, and have influenced the direction of our work. %A full list of our contacts can be found at \todo{insert Appendix link}
    
    The \textbf{\gls{UCPP}} were crucial in setting the problem, and gave us first insights into what is needed of a sensor technology. They required a device, which would improve efficiency of field work and allow quantification of the impact and success of efforts to restore soil health, in developing countries. 
    
   \textbf{Miguel Hernandez (GS-fresh)} has provided knowledge about state-of-the-art of sustainable farming methods, soil treatment, soil sampling, and soil analysis within the UK and Europe.
    
    The choice of sensible measurement parameters, which would result in a holistic overview of soil health and seemed feasible to develop in the scope of the project, was supported by \textbf{Philippa Arnold (National Farmers Union)} and \textbf{Paul Flynn (Soil Association)}. Beyond this, Paul Flynn has validated the design and setup of the sensor platform when milestones of the project were reached and expressed interested in further collaborations. 
    
    \textbf{Zimmer and Peacock} have played a major role in our decision to use the colour sensor not only as an educational tool, but as an actual peripheral sensor of our sensor platform to monitor nutrients and pH. Martin Peacock has assured us that electro-chemical methods (industry standard) are not yet robust enough for our purposes, and that indicator paper strips are an appropriate solution to the problem given our time and budget.
    
    \textbf{Dr. Sam Stanier} from the Department of Engineering at the University of Cambridge has been our external expert, who was consulted for questions about the feasibility of the moisture retention probe and its use.
    
    With \gls{SOM} being a primary soil health parameter, but feasibility studies showing no promise in creating a working \gls{SOC} device within the scope of the project, a collaboration with \textbf{PhotosynQ} was initiated. PhotosynQ is an open-source project focusing on plant health originating in the USA. They have developed a tool to monitor the active pool (5-20\%) of \gls{TOC}, which is a good indicator of biological activity, nutrient availability and can be used to predict crop yield. 
    PhotosynQ shared their protocols and \gls{BOM} with us. Sean Reed and Dr. Dan TerAvest were very supportive and willing to share information about the sensor setup along the way. SoliCamb conducted preliminary experiments to assess the feasibility of combining this sensor within our toolkit, the details of which are discussed in \cref{CO2sensor}.
    
    Through a program run at Earthwatch and the University of Oxford, which focuses on fighting soil erosion, we have started talking to \textbf{Dr. Martha Crockatt (Earthwatch)} and \textbf{Dr. Anna Krzywoszynska (University of Sheffield)}. Beyond the soil erosion project, both are currently working on a new project commencing October 2019. The aim is to create a citizen engagement/outreach program to raise awareness for soil health, and getting citizens and technology more involved in the subject, as well as bringing them closer together. Both Dr. Crockatt and Dr. Krzywoszynska focus on citizen science in agriculture and offered SoliCamb the opportunity to take on the technology aspect of the program by introducing people to our sensing platform. At the moment, we are in the process of putting together a program, as well as a detailed prescription on how to use our sensors for the event.
    
    Dr. Krzywoszynska has brought to our attention \textbf{SectorMentor} as a potential collaborator. SectorMentor works on soil health monitoring, and has created an app to easily store and manage data of several soil health parameters. They seem interested in a collaboration with SoliCamb, especially for their work in vineyards. As of now, this is in very early stages.
    
    We had the pleasure to test soil at \textbf{Anglesey Abbey}, property of the \textbf{National Trust}. Head gardener, David Jordan offered SoliCamb the opportunity for future collaboration, and access to the gardens for soil testing,  volunteering to test our future prototypes with colleagues. SoliCamb was also offered the chance to extend this collaboration to other NAtional Trust properties, based on the results of this initial collaboration.
    
    \textbf{Agri-Tech East} are actively supporting us with finding future collaborators and funding. They have invited us to several conferences and events that revolve around sensor technologies, soil health monitoring, and farming, as well as entrepreneurship.
    
    During our participation at the \textbf{Agri-Tech East Hack-a-thon}, we modified our idea to UK-based needs, and thus to an autonomous sensing platform that would be deployed in the field (see \cref{smrpcontinuousmonitoring}). The \textbf{BASF} representative approached SoliCamb and suggested that we get in contact once the project has developed further, to pitch the idea to BASF in the UK or BASF in Germany. 
    
    Depending on future directions, \textbf{Let's Grow}, a Dutch company focused on data visualisation in the agriculture sector, has expressed interest in working with us to improve our online application and user interface.

    
    \subsection{Soil sample sites}\label{Sample Sites} 
    %Author:  - Katie
    As mentioned above, field testing was conducted in Anglesey Abbey and Radwell Grange Farm. These experiments were used to evaluate the user-friendliness of our device by non-experts and to generate in-field data.
    
    
   At Radwell Grange Farm, valuable information on robustness of C-3NpH was obtained by SoliCamb members. In this instance, field-testing was conducted by members of the SoliCamb team only. It was noted that waiting up to 30 minutes for soil to settle was cumbersome, and there was high variability in separation. For some samples, siphoning off the supernatant was challenging. This is something that requires further work to optimise the user experience, especially if multiple sites are to be sampled to map nitrate concentration across a field. This would not be such a hindrance if all soil samples across the field were collected, mixed, homogenised and then measured, where one 30 minute interval for soil to settling would be acceptable.
   
  At Anglesey Abbey, we had the pleasure of working with the head gardener,  as the citizen scientist who had previous experience of horticultural training. They were  especially interested in pH testing and being able to conduct the tests themselves on site. This contrasted to previous conversations with an arable farmer, who showed interest in obtaining the results, but not in performing the tests. Current field testing protocols were easy-to-follow, however, again settling time (observed as \SI{20}-\SI{30}{\minute} on the day) was cumbersome. Therefore, developments with the WG are proposed to be beneficial to the end user by reducing sampling time. Concerns raised by the citizen scientist at Anglesey Abbey pertained to cross-contamination between strips inserted, which requires further experimental investigation (alongside protocol optimisation) despite not being an initial concern during laboratory experiments. On a number of occasions, the colour pad of strips fell off during measurement. This needs to be considered for future design iterations. 
  
   The moisture retention probe was also taken to the above sites (as detailed in \cref{sec:moistureprobe_field}). 
   
   In addition to in-field testing, laboratory validation was conducted on soil from the Cambridgeshire area, including Grange Farm (Lolworth), Madingley Mulch Outdoor Supplies, King's College Allotments, Wolfson College Grounds,  a home garden in Cherry Hinton, and a public walkway in West Cambridge. 
   
   \section{Traditional methods for advertisement}
    Traditional 'print' methods are also a valuable tool for advertisement and generating interest at local events. For departmental and university-wide engagement, SoliCamb posters were made and hung up in the department of \gls{CEB} as well as several other departments. Moreover, during the latter half of the project, an introductory advertisement video for SoliCamb was broadcast on departmental televisions. Team T-shirts, business cards, stickers and professional conference banners were also designed to promote the brand at professional events (\cref{fig:traditional_advertisement}).  Designing several products, targeted at different academic and demographic populations, aimed to increase SoliCamb's marketing reach.
    
    \begin{figure}[ht]
    	\centering
    	\includegraphics[width=0.9\linewidth]{Pictures/Outreach/traditional_advertisement.png}
    	\captionsetup{justification = centering}
    	\caption{ Traditional methods used for advertising SoliCamb. These included posters, business cards, T-shirts, stickers and conference merchandise }
    	\label{fig:traditional_advertisement}
     \end{figure}

  \section{Outreach}
    Online platforms play a crucial role in shaping how information and influence spread among citizens. Social media tools are deemed as very important triggers of change for teaching and learning practices \cite{manca2016facebook}. Such platforms aim to radically transform the academic environment to be more social, open and collaboration-oriented, enabling scientists to communicate their research quickly and efficiently to professionals, as well as to non-expert public members. Although the exact definition of social media is constantly in a state of change, social network sites, blogs and multimedia platforms are typically part of the today's social media landscape \cite{tess2013role}. Currently, Facebook, Twitter and Instagram are the most popular and widely-used social media platforms with 1.56 billion users around the world active on Facebook every day as of March 2019 \cite{investopedia}. Therefore, for both advertising and promoting SoliCamb and affiliated events, the team made use of these three platforms targeting different demographics. In particular, Facebook is more popular amongst middle-aged adults living in rural, suburban, and urban areas, at every income level and educational background \cite{investopedia}. By contrast, internet users living in urban areas, within the academia environment, are more likely to use Twitter as their primary source of news. Younger internet users (18-29 years old), however, make high use of Instagram as a platform for photo/video sharing purposes.
     
        
        \subsection{Social media platforms}\label{Social media}
        % Chiara: importance of this platform for advertising/engaging.  Give stats on this over time course of project, what posts received most likes, how many signed up to events from these platforms etc. 
       A trio of Facebook, Instagram and Twitter accounts have been simultaneously set up on behalf of SoliCamb, in order to engage with the public. As aforementioned, most of the posts published on Facebook and Instagram are aimed to reach the teenage public, keeping them abreast with ongoing work and progress within SoliCamb. To this end, snapshots of the team working on the development of the sensor platform, doing in-field or laboratory testing and attending  popular outreach events (e.g. 105 Radio Cambridge, BBC Radio and That's TV Cambridge), have been shared throughout the 10-week project timeline (\cref {fig:OutreachRadio}). 
       
       
          \begin{figure}[ht]
    	\centering
    	\includegraphics[width=0.55\linewidth]{Pictures/Outreach/OutreachRadio.jpg}
    	\captionsetup{justification = centering}
    	\caption{ Some of SoliCamb's outreach events held between July and August 2019. Upper left shows the team's feature in the Cambridge Independent. Bottom left and upper right show the feature with Cambridge 105 Radio and BBC Radio, respectively. Middle right shows a workshop with Cambridge TV and bottom right shows the team advertising with SoliCamb branded T-shirts}
    	\label{fig:OutreachRadio}
     \end{figure}
       
       
       By contrast, the Twitter page was very useful in reaching a wider and more mature audience.  SoliCamb's initiatives and events were shared within the academic network, gathering interest from soil scientists, engineering enthusiasts and environmental experts. Finally, social media was employed as a tool to share several articles/issues about soil science (e.g. \enquote{\textit{Not another climate horror story}}, The Guardian (2019) \cite{guardian}) 
       
SoliCamb's Facebook, Instagram and Twitter accounts have 185, 75 and 65 followers, respectively, to date.  With regards to the outcome of the social media activities, some insightful data has been collated about the most liked and popular posts as shown, for example, in \cref{fig:FbFollowers}, which has allowed us to better understand the impact and reach of both our online presence as well as our other outreach activities. These data and other insights are discussed further below. 
      
          \begin{figure}[ht]
    	\centering
    	\includegraphics[width=0.8\linewidth]{Pictures/Outreach/Picture3.jpg}
    	\captionsetup{justification = centering}
    	\caption{An overview of SoliCamb's Facebook page followers as of today. The page was originally set up on the 23 June 2019 and ever since, a steadily increasing trend can be seen.}
    	\label{fig:FbFollowers}
     \end{figure}
     
\paragraph{Facebook}

Based on the data from 15 July-15 August, the number of people reached (i.e. the number of people who were registered as having seen any posts from SoliCamb's page on their screen) was 5490 and the number of posts engaged with (i.e. the number of times that people have engaged with our posts through likes or comments) was 699. With respect to the 17 posts published over the weeks, 10 were photos, 6 links and 1 video. The post that raised the most interest among the followers was published on the 6 August, showing footage of our interview feature with That's TV Cambridge, which reached over two and a half thousand people. The second most successful post was related to the Cambridge Independent article release, which reached 1.7 thousand people. In regards to the demographics of those who followed the SoliCamb's Facebook page, the data was grouped by age and gender. The information reported in \cref{fig:FbAge} shows that 46\% and 54\% were women and men, respectively, who were in the 25-34 age range. This demonstrated that the SoliCamb's advertising approach was capable of targeting both genders equally, along with expected demographic statistics, as reported by Facebook \cite{investopedia}. As discussed previously, younger age-groups were targeted through different social media platforms and older age groups through radio and newspaper platforms.

 \begin{figure}[h!]
    	\centering
    	\includegraphics[width=0.95\linewidth]{Pictures/Outreach/FbAgeRange.jpg}
    	\captionsetup{justification = centering}
    	\caption{Facebook SoliCamb followers grouped by age and gender. Overall, both women and men have been reached equally, within the age range 26-34 years old.}
    	\label{fig:FbAge}
     \end{figure}

%\begin{figure}[h!]
    	%\centering
    %	\includegraphics[width=0.7\linewidth]{Pictures/Outreach/MostLikedPict.jpg}
    %	\captionsetup{justification = centering}
    %	\caption{Overview of Facebook posts published with information about the type of post, reach and engagement.}
    %	\label{fig:MostLikedPict}
    % \end{figure}
 
\paragraph{Instagram}
 
The Instagram and Facebook platforms were ran concomitantly by sharing the same pictures and videos in order to reach as many people as possible. To advertise SoliCamb's name on Instagram, SoliCamb  `followed' pages relevant to the project on a daily basis, as well as engaging new `followers' in turn. Post shared on Instagram employed relevant tags (e.g. \@solicamb, \@EPSRC, etc) and hashtags (e.g. \#soilhealth, \#citizenscience, \#southafrica) to extend the target audience to those involved in soil science, the environment and academia.

\begin{figure}[h!]
    	\centering
    	\includegraphics[width=0.55\linewidth]{Pictures/Outreach/Instagram.jpg}
    	\captionsetup{justification = centering}
    	\caption{An Overview of SoliCamb's Instagram frontpage.}
    	  \end{figure}

\paragraph{Twitter} 
The SoliCamb's Twitter account was created as an additional means for broadening the target audience, specifically targeted at experts and enthusiasts from the academia/industry environment. Based on the data from August, the SoliCamb's twitter page raised 17.1 thousand impressions (about 608 impressions per day), \footnote{Impressions: the number of times a tweet is shown on one's timeline} 397 profile visits and 20 `retweets' \footnote{Retweet: followers who advertised SoliCamb by sharing posts from the team page on theirs}. As reported in \cref{fig:Twitter}, the tweet that raised the highest engagement was in regards to SoliCamb's article in the Cambridge Independent, showing again the success of this media release. 

  \begin{figure}[ht]
    	\centering
    	\includegraphics[width=0.9\linewidth]{Pictures/Outreach/Twitter.jpg}
    	\captionsetup{justification = centering}
    	\caption{Twitter statistics showing the main activities over the past 28 days period. The post relating to the release of the Cambridge Independent article raised the most impressions and `retweets'.}
    	\label{fig:Twitter}
     \end{figure}
  
    \subsection{Website and newsletter}
    %Author: Sarah
        The website was created using Wix Website editor, and acted as a platform to document SoliCamb's progress, give interested parties a point of contact, and to raise awareness of our challenge and the functionality of our device. The website has 5 pages which encompass SoliCamb's mission statement, team information, device design, a blog, and a contacts page (\cref{fig:website}). 
        
        \begin{figure}[h!]
	\centering
	\begin{subfigure}[b]{0.45\linewidth} 
		\centering
		\includegraphics[height=5cm]{Pictures/Outreach/website_a.png}
		\caption{}
		\label{subfig:website_a}
	\end{subfigure}
	%\vfill
	\begin{subfigure}[b]{0.45\linewidth}
	\centering
		\includegraphics[height=5cm]{Pictures/Outreach/website_b.png}
		\caption{}
		\label{subfig:website_b}
	\end{subfigure}
	\captionsetup{justification = centering}
	\caption{SoliCamb's (a) website homepage and (b) blog page, where highlighted in red are the interactive functions for user engagement.}
	\label{fig:website}
\end{figure}   	

        
        %\begin{figure}[ht]
    %\centering
    %	\includegraphics[width=\linewidth]{Pictures/Outreach/website.png}
    %	\captionsetup{justification = centering}
    %	\caption{SoliCamb's website homepage (A) and blog page (B). Highlighted in red are the interactive functions which can be used to engage with the user. }
    %	\label{fig:website}
    %\end{figure}
        
     % Number or users and session time
Starting from the release date of the website, up until the day of SoliCamb's final presentation (24 June - 15 August), the website had over 190 users and over 960 page views. The average `session' or viewing time on the site was defined as \enquote{session starting when a user views a page on the website and ends either when they leave or after 30 minutes of inactivity}. Our average user session time was $2.5$ minutes. This is comparable to the industrial benchmark, which states an average session duration of $2-3$ minutes is `good' \cite{user_session_stats, user_session_stats1} and shows that the  website is successful in engaging the user.
        
    % Bounce rates
    Bounce rate is another parameter often used as a measure of user interest and retention. A bounce is defined as \enquote{a single-page session} and bounce rate as  \enquote{the percentage of all sessions in which users viewed only a single page before exiting the site}. In SoliCamb's case, a low bounce is desirable, as the website contains more than one page. Interestingly, the bounce rate for \texttt{Our Missions} and \texttt{About Us} pages were the highest at 100\% and 75\% respectively, whereas the blog page had the lowest bounce rate of less than 30\%. The blog page contains links and interactive tool-boxes for the user to engage with, including options to `like' and leave comments on posts, as well as sign-ups for the weekly newsletter (\cref{subfig:website_b}). In contrast, the other two pages are largely descriptive, which highlight the importance of page design in retaining user interest. In the future, pages with high bounce rates should be redesigned to include more interactive elements, such as videos and links to our other pages.
    
    % User reach - geographical location
    Obtaining information relating to geographical location of SoliCamb's users can be used to assess the success of the project's reach. Although 76\% of users were located locally in the UK, it was interesting to see that the website had a global reach of 10 countries (\cref{fig:user_location}), including those which we have reached out to, or formed collaborations with. Continuing to gain global recognition will help to further promote and develop SoliCamb's project. 
        
    \begin{figure}[ht]
    	\centering
    	\includegraphics[width=0.8\linewidth]{Pictures/Outreach/user_location.png}
    	\captionsetup{justification = centering}
    	\caption{Geographical location of website users.}
    	\label{fig:user_location}
     \end{figure}
        
    % User site acquisition 
    Analysing user acquisition is a useful way to understand where users originate and how successful advertising/marketing schemes are. From  \cref{fig:user_traffic}, it is seen that the majority of users originate from direct and organic sources, with 75\% typing SoliCamb's web-address directly into the URL search bar or via a search engine respectively. Given that the majority of users are locating the website via a direct web-search, demonstrates that the team is advertising the SoliCamb brand well, and not relying too heavily on one media platform to generate interest. Users which were referred from another site came mainly from the Sensor CDT webpage, and interestingly, BlueBear Systems Research Ltd website. BlueBear is a company based in the field of avionics and data management and they featured SoliCamb in their online newsletter after a team visited. This demonstrates the power of forming relationships with experts,in order to reach larger target audiences. Out of the website's social media referrals, 75\% of users came from the Facebook page, suggesting Facebook may be more successful than other social media platforms in reaching a subset of SoliCamb's audience. 
       
    
        \begin{figure}[ht]
    	\centering
    	\includegraphics[width=0.8\linewidth]{Pictures/Outreach/user_traffic.png}
    	\captionsetup{justification = centering}
    	\caption{Source of website user acquisition.}
    	\label{fig:user_traffic}
     \end{figure}
     
     % Newsletter
    The newsletter was initially set up as a one page PDF document that was sent out via email to interested parties and collaborators. The purpose of the newsletter was to track SoliCamb's progress, to keep the audience up to date and to promote public engagement. Mid-way through the project, a blog page was added to the website where all newsletters could be accessed and updated. The blog page  further promoted user engagement by utilising interactive tools such as `like' and `comment' buttons. After the blog page went live, there was a surge in website views and session time (see \cref{fig:media_peak} in \cref{Social media}). The addition of the blog page also made it easier for new users to sign up to the newsletter emailing list which, to date, has 75 sign-ups. 

    \subsection{Media Appearances}
    
    \subsubsection{Newspaper}
        % Chiara/katie: 
        Designed to coincide with other press releases including radio and TV, an article was featured in The Cambridge Independent Newspaper on 31st July 2019. The article was authored by Paul Brackley, Editorial Director, after an interview with two members of the SoliCamb team. The piece featured photographs taken by Keith Heppell and covered an introduction to the importance of soil health, the motivations behind SoliCamb as well as an overview of the sensor hardware. 
        The two-page article highlighted the multiple components of our device, which at the time were being developed independently.  %In the first instance, would have been used by local farmers in South Africa in order to help them out in the assessment of soil quality according to the UCPP programme. As aforementioned, such national plan aims to address the severe land degradation and invasive alien plant infestation in the Umzimvubu catchment. Furthermore, such device could be used by anyone with an interest in soil health. 
        One of the key motivations for organising this article was to provide and advertise that SoliCamb had created a platform that was accessible and open to citizen scientists.
        
        In order to fully engage with the end user it was necessary to first generate interest through readily available channels, and second it was crucial to emphasise the value of our device to local enthusiasts. The press campaign sought to make it clear that this project, and any sensors made, were not reserved only for experts in the field of soil health but focused on making soil health measurement open to anyone. Awareness of the importance of soil health is growing, especially in relation to meeting the Sustainable Development Goals \cite{Keesstra2016} and one of the desired outcomes of SoliCamb was to emphasise that this is a local issue as well as one found in South Africa. Therefore it was crucial that we engaged with local communities through platforms such as newspapers to not only highlight our sensors but iterate the attitude towards preserving our soil in Cambridge. 
        It was important to organise the release of the article at a time when it would have the highest impact. For this reason it was decided that the article should precede radio or TV coverage as an additional form of advertisement guiding local readers to these other platforms.
        
   
        
 \subsubsection{Radio}
         On 31st July 2019, two members of the team went on air at BBC Radio Cambridgeshire for a live interview, discussing soil health and the motivation behind SoliCamb. The aim of this broadcast was threefold: firstly, to raise awareness on the importance of soil health and the impact of degrading soil on the environment and living ecosystems; secondly, the feature sought to invite farmers, allotment holders etc to test their soil with our sensor allowing us to validate our prototype and generate data for analysis. It was hoped that the broadcast would engage user interest and promote opportunities to attend upcoming SoliCamb outreach events (e.g.the STEM festival in Peterborough), as well as aiming to set up external collaborations scientific partners. On the 1st August 2019 Neil Whiteside hosted a live interview with two members of the SoliCamb team on Cambridge 105 at 10:30 am. Feedback from Mr. Whiteside showed that the number of listeners was reported as incredibly high demonstrating the effectiveness of SoliCamb's outreach. Expert advice from Mr. Whiteside was key to understanding how best to use the power of these public platforms to generate enthusiasm for our project, due to this the team coordinated all TV, radio and newspaper to release within one week of each other as well as coinciding with the upcoming Make-a-thon at Cambridge Makespace. \\
 \subsubsection{Television}
 
         That's TV Cambridge visited the department on 31st July 2019 and recorded three members of the team in both B-Roll footage and interviews. The segment featured on their show that evening and has since been posted across our social media channels. The footage gathered incorporated the modular device as well as SoliCamb's opinion on the importance of soil health and our incentive to get involved at a local level.  Altogether, the TV and radio releases were a good opportunity to discuss the projects main features and future plans.\par 
         
In contrast to pre-scripted articles that could be drafted, radio took the form of live  information sharing. Prior to appearing on radio and TV, the release of information about our project had been relatively self directed. For example, it was the SoliCamb outreach team who drafted pieces detailing information presumed to be relevant for a wider audience but formed entirely from an internal perspective. Only when questioned by external parties (e.g. radio hosts) did it become apparent that those outside the project were perhaps more interested in a different perspective, to what had already been pitched. These insights will be valuable moving forward as the team engage more with local communities and are expected to improve our communication strategy with diverse audiences.
         
        % media peaks 
        Following the aforementioned media releases on 31st July and 1st August, it was promising to see a surge in online engagement on social media pages and website (\cref{fig:media_peak}). Again, following a re-post of said events, spikes in online engagement were also noted. Observing peaks in online engagement, after major outreach events, demonstrates the value and success of advertisement in boosting overall reach. This is an avenue which should be pursued further in the future. 
        
        \begin{figure}[ht]
    	\centering
    	\includegraphics[width=0.9\linewidth]{Pictures/Outreach/media_peak.png}
    	\captionsetup{justification = centering}
    	\caption{ User engagement on website and Facebook platforms over the duration of our project. After major outreach events and media releases, SoliCamb's user engagement show a surge in activity.}
    	\label{fig:media_peak}
     \end{figure}
    
    
    \section{Public Engagement Events}
    % brief overview of the types/categories of events run.- Katie. 
    With a key component of the Team Challenge being citizen science and community engagement, the outreach and public engagement work focused on sustaining a bi-directional relationship.  As each event required detailed plans, considerable effort was focused on understanding the main objectives and desired outcomes, ensuring that the event was mutually beneficial.
    
    Firstly, C-3NpH was intended as an educational tool. This would allow SoliCamb to collaborate with Immerse Cambridge Summer School, working with teenagers (16-18 years old) as citizen scientists who engaged in learning while generating data to validate the pH models between multiple C-3NpH sensors. The participants learnt basic sensor design and electronics, which was a topic not previously studied for most of them. The summer school work was tailored to a specific audience and was pitched to individuals with no experience of engineering or sensor technology. 
    In contrast, the second public engagement piece was the mini Make-a-thon run at Cambridge Makespace. The audience here was more in line with what SoliCamb envisaged as their citizen scientist end user. The third event and again a different category was the Agritech East Hack-a-thon. This was intended not only to showcase SoliCamb's work but also to network and engage with local industry, with a focus on generating interest in setting up potential collaborations. The approach taken by the team highlights the well-rounded and diverse nature of public engagement, the intricate aims, structures and outcomes, of which are detailed in the following sections.
    
\subsection{Cambridge Immerse: Summer school workshop }

        The first opportunity to interact with potential users and carry out usability studies of C-3NpH was via summer schools. We collaborated with Immerse Education,  which runs a series of summer schools including classes and activities related to the field of engineering. The students were aged between 16-18 and were from various backgrounds. Although the students had no formal engineering education, they were generally in the process of applying to \gls{STEM} courses at university.
        The workshop was structured so that the students attended a morning session on the principles of electronics, programming and sensing. The session culminated in a practical session where the students built their own Arduino-based fire alarm, which demonstrated the principles of sensor input, data processing and actuator output.
        The afternoon session introduced the principles underpinning the C-3NpH sensor, and allowed students to build and test their own system using components produced by SoliCamb. The session was mutually beneficial as it allowed us to carry out basic user interaction and ease of use testing for this design and prototype. (For further details, see \cref{para:immerse}.)
        The workshops were very well received by the student and Cambridge Immerse Education, who invited SoliCamb to run a second workshop after the initial agreement, which had been for only one day. %and the SoliCamb team who all benefited from the days that were run.
        
    %%Added by Chyi: Feel free to omit :) %%%
   \begin{figure}[ht]
	\centering
	\begin{subfigure}[b]{0.45\linewidth} 
		\centering
		\includegraphics[width=\linewidth]{Pictures/Outreach/summerschool1.jpg}
		%\captionsetup{justification = centering}
		%\caption{}
%		%\label{subfig:summer_assem}
	\end{subfigure}
	\begin{subfigure}[b]{0.45\linewidth}
		\includegraphics[width=\linewidth]{Pictures/Outreach/summerschool2.jpg}
%		\captionsetup{justification = centering}
%		\caption{}
%		\label{subfig:summer_C3NpH}
	\end{subfigure}
%	\captionsetup{justification = centering}
	\caption{SoliCamb at the Cambridge Immerse Summer School}
	\label{fig:immerse}
\end{figure}  

\subsection{Cambridge Makespace Make-a-thon}

%\todo{jh2109 Explain the main achievements made there, how the attendees responded to our work}
% Chiara?: Explain the purpose of this event and explain the schedule of the day. State how many people attended, what was observed from this event and what the key outcomes were. 
The team held a Make-a-thon on 3 August, from 11~am to 4~pm, at the Cambridge Makespace venue, after arranging this with Ward Hills (Director of Makespace). %jh2109 removed, he doesn't play a role in the rest of the text, does he?
%(see \Cref{MSpace}). 
Overall, 23 people signed up for the one-day event, out of which 5 were members of SoliCamb. The event was advertised via the major outreach channels (radio and newspaper), in parallel to the other social media platforms. Eventbrite and Meetup links were set up for registration. By running this public engagement event, the SoliCamb team engaged %the public (i.e.
with attendees from different backgrounds, such as PhD students, electrical engineers, software experts, and science enthusiasts.   %\todo{KG- chiara where were the experts from you need to specify here what their field was I think} and enthusiasts from the \enquote{maker} movement. %in collaborating with SoliCamb. 
This event was very rewarding and represented a platform for open discussion and exchange of ideas. By joining forces with this community the foundations of a new network and knowledge base have been laid. This means that in the future, should SoliCamb continue with hardware development, there will be a wider and more diverse group of potential collaborators with whom relationships have already been formed. 

%In addition to soliciting general feedback and suggestions throughout the event, 
It was interesting to hear general opinions, and have this user group critique SoliCamb's device. In addition, there were two particular challenges proposed for the participants to solve. This was in order to %so as to
streamline the event, and %give an initial focus to all attendees:
focus discussion around areas that were of particular importance to SoliCamb's hardware optimisation. The questions posed were:

\begin{itemize}
  \item How do we improve our design in order for it to be waterproof, easy to use and cheap?
  \item How do we measure small capacitance changes using a moisture probe with fewer components (accuracy-complexity trade-off)?
\end{itemize}

The attendees were asked to decide which challenge they would prefer to address, and then were split into two subgroups accordingly. One group worked on designs for device enclosures, while the other worked on the capacitance problem for the moisture retention probe. This made best use of the time available.

%For the SoliCamb team, 
%This event %turned out to be 
%was very rewarding as it %offered the chance to talk and discuss with people from different backgrounds.
%provided a platform for open discussion and ideas exchange. %with people from a range of different scientific backgrounds. %Amongst the participants were PhD students in electrical engineering, experts in Arduino and Raspberry PI as well as software engineers.\todo{chaira I'd mention why this was the intended audience} %and others. % .% I feel this sentence repeats previous content -jh2109
%In particular, the following outcomes were recorded by the SoliCamb team: 
There was a wide range of feedback and advice offered throughout the event, but of particular interest were the comments below.

        \begin{figure}[ht!]
    	\centering
    	\includegraphics[width=0.6\linewidth]{Pictures/Outreach/MSpace.jpg}
    	\captionsetup{justification = centering}
    	\caption{Mini Make-a-thon event at Makespace on 3rd August.}
    	\label{fig:MSpace}
     \end{figure}
\paragraph{Approval of current design}   
   \begin{itemize}[label={$\checkmark$}]
\item Design decisions made to date were broadly correct.
   \item JLBPCB was the correct PCB manufacturer to use for high quality and low cost with a 5 day turnaround. 
   %	\item Consider that not everyone in South Africa will have a smart phone.
   	%\item Cheaper smartphones are generally less powerful, which might not suit our needs.
 \item Modular approach is an extremely good idea. 

   \end{itemize}
   
\paragraph{Suggestions for design alterations}   
\begin{arrowlist}
  \item JLBPCB can also assemble boards.
  \item Eurocircuits was the advised PCB manufacturer to use for a faster 3 day turnaround at higher cost. 
\item Capacitive touch buttons are \enquote{almost for free} at high volume - so should be used.
\item Replacing the base unit with a smart phone would allow for significant cost reductions.
\item Consider Bluetooth rather than USB cables as a cheaper more flexible way to communicate with sensors.
\item In order to increase production volume, a change of the design should be anticipated.
\item Single side component load and double sided board will be cheap for mass manufacture.
\item Integrate all breakout boards onto a double sided PCB with single sided component. %load to got to 1000 volume. 
\item Using a reed switch for power will help with waterproofing.
\item A wire and a grommet would be cheaper than adding a USB connector.
\item Microcontroller only should be considered for measuring capacitance.
\item For simplified phone-app development, frameworks exist; although these might become restrictive later on.
\item A micro-USB port could be used for charging,due to its ubiquity in the smartphone space.

\end{arrowlist}
  
For a discussion of how these changes will be integrated into future designs, see Future Plans section 5. 
  
   
%\begin{itemize}
	%\item When optimising cost there is a danger of buying fake parts so always purchase from the manufacturer. 
	%\item The number of vendors selling a component indicates how popular it is. 
	%\item For 1 million volume have a semiconductor company manufacture a custom chip.
	%\item These frameworks offload app signing to an already-verified third party, considerably simplifying. distribution
% as a microUSB is generally associated with charging 
%\end{itemize}
   
%\textbf{Feedbacks: Sagnik, James, Jan}
%{The suggestions on modifying the device design can be summarised as follows:
%\begin{itemize}
    %\item Modification of the user interface to a "plug-and-measure" type experience (discussed briefly in \ref{section:gui}). Lesser user interaction makes the device easier to use.
   % \item Reduction to 2 buttons ("Select" and "Cancel") instead of 3 and use of LED or buzzer to indicate initiation and completion of a measurement.
   % \item Use of single USB port for charging as well as connecting sensors. If multiple ports are necessary, making them different for the user's ease of perception.
  %  \item Incorporation of a waiting time after the device is switched on, to allow the GPS to lock.}
%\end{itemize}

%\todo{KG - Chiara can you add what was implemented\ not implemented and why?}
 
\subsection{Agri-Tech East Hack-a-thon}
Agri-Tech East is an organisation that focuses on improving the productivity and sustainability of agriculture by bringing together scientists and entrepreneurs, with farmers and growers, to create a global innovation hub in Agri-Tech. Agri-Tech East organised a Grow--Hack-a-thon centered on agriculture in the UK, at the Future Business Centre in Cambridge on 5-7th July 2019. 
The event focused on combating different agricultural problems associated with negative environmental impacts, and aimed to develop sustainable solutions that guaranteed the longevity of food production.
Barclays Eagle Labs, WWF and BASF, the main sponsors of the event, set four challenges; soil health improvement, smarter water irrigation systems, managing soil as a main carbon sink and how to enhance biodiversity.

SoliCamb's motivation for attending this event, was the potential for receiving input on how to define and tackle the main challenges that farmers and the agriculture sector face within the UK and Europe. Therefore, SoliCamb sought to improve their understanding of local needs and market gaps, outside of the initial SoliCamb directive that targets developing countries. 

During the two-day competition, the moisture retention probe was re-built with materials available at the Hack-a-thon, demonstrating the adaptability of our design. Guided by feedback from mentors and other attendees, and the event's focus on UK agriculture, the concept of an easily adjustable moisture retention probe was established. This design would allow the measurement of soil water content at certain depths correlating to the root system specific to the crop. The idea was to deploy multiple water retention probes in a field, creating a network of sensors that wirelessly communicated with a base unit, before the collected data can be uploaded onto a cloud. The design concept comprised a smartphone front-end for data access and visualisation to enable on-demand agronomic decision making (see \cref{fhackathonteam,fhackathondemo}). 
Concerning nitrate leaching, future plans are to implement the moisture retention probe with a smart soil sampling mechanism by integrating an inner, removable rod to allow effortless soil sampling to greater depths. The soil samples could potentially be analysed with C-3NpH. 

Different to the rest of SoliCamb's outreach, the event allowed us to interact with farmers, experts in agriculture as well as Agri-Tech mentors in person and for longer than the length of a phone call. This allowed for extensive conversations about agriculture and sensor technologies in the field and led to new ideas. The event created new perspectives for certain aspects of the SoliCamb project.

Main outcomes of the event were the opportunity to interact and build a collaboration with Agri-Tech East (\cref{Collaborators}), and an invitation by the representative of BASF Agricultural Solutions UK to pitch our project idea to BASF (\cref{Collaborators}).


\begin{figure}
	\centerline{\includegraphics[width=.7\linewidth]{Pictures/agritech_hackathon/hackathon_team.jpg}}
	\captionof{figure}{A photograph taken of the SoliCamb Team present at the Hack-a-thon, with the hardware shown during the final presentation and demo.}
	\label{fhackathonteam}
\end{figure}
\begin{figure}
	\centerline{\includegraphics[width=.7\linewidth]{Pictures/agritech_hackathon/hackathon_demo.jpg}}
	\captionof{figure}{The demo shown as part of the presentation during the Hack-a-thon.}
	\label{fhackathondemo}
\end{figure}
        

\section{Overall insights}
 In order to evaluate our project's public engagement, we used a public engagement impact grid, which is commonly used to assess the level of mutual benefit obtained by involved parties (\cref{tab:impact_grid}). It can be seen that after the completion of our team's outreach events the benefits obtained by SoliCamb and target audiences were balanced. This is in accordance with the definition of public engagement and our key objectives, as outlined in \cref{fig:vendiagram}. %\ref{Citizen engagement}
 
\begin{table}[h!]
	\centering
	\begin{tabular}{l c l l l}
			\toprule
			& {Knowledge \& Awareness} 
			& {Attitudes} 
			& {Skills} 
			& {Empowerment} \\
			\midrule
			SoliCamb & $\Diamond$ $\clubsuit$ $\triangle$ 
					& \hspace{1pt} $\Diamond$  
				& $\Diamond$ $\clubsuit$ $\triangle$
				& \hspace{27pt} $\clubsuit$ $\triangle$ \\
			Audience & $\Diamond$ \hspace{7pt}  $\triangle$ 
				& \hspace{1pt} $\Diamond$ $\clubsuit$ $\triangle$
				& $\Diamond$ 
				& \hspace{18pt} $\Diamond$ $\clubsuit$ \hspace{1pt}  \\	
				\bottomrule
			\multicolumn{5}{c}{\small Key: $\Diamond$ Summer School; $\clubsuit$ Make-a-thon; $\triangle$ Hack-a-thon} \\
  		\end{tabular}
  	    	\caption{Public engagement impact grid used to assess the level of mutual benefit obtained by both parties, after the completion of all outreach events. Presence of a symbol indicates that event fulfilled that requirement.}
    	\label{tab:impact_grid}
    \end{table}



   
      %   \begin{table}[h!]
       % \resizebox{\textwidth}{!}{
    %    \begin{tabular}{@{}lcccc@{}}
     %   \toprule
      %   & Knowledge and Awareness & Attitudes & Skills & Empowerment \\ \midrule
      %  SoliCamb & \begin{tabular}[c]{@{}l@{}} Summer School \\ Make-a-thon \\ Hack-a-thon \\ \end{tabular} & Summer school & \begin{tabular}[c]{@{}l@{}}Summer school \\ Make-a-thon \\ Hack-a-thon\end{tabular} & \begin{tabular}[c]{@{}l@{}}Make-a-thon\\ Hack-a-thon\end{tabular} \\
        %Audience & \begin{tabular}[c]{@{}l@{}}Summer school\\ Hack-a-thon\end{tabular} & \begin{tabular}[c]{@{}l@{}}Summer school\\ Hack-a-thon \\ Hack-a-thon\end{tabular} & Summer school & \begin{tabular}[c]{@{}l@{}}Summer school\\ Hack-a-thon\end{tabular} \\ \bottomrule
        %\end{tabular}%
        %}
        %\caption{Public engagement impact grid used to assess the level of mutual benefit obtained by both parties, after the completion of all outreach events. Key: $\faSunO$ summer school; }
        %\label{tab:impact_grid}
        %\end{table}
    
    Outreach %citizen science 
    events successfully obtained feedback on the usability of the device from citizen scientists and also generated in-field data. The main conclusion drawn from %these experiments 
    time spent measuring nitrates and water retention in Anglesey Abbey and Radwell Grange Farm, was the need for optimising the soil sampling and extraction method, considering both the user practicality of the device in addition to ensuring the validity of results. 
    
    Our online platforms were successful in generating a community of interested users. The social media pages were important tools for advertising the team name, spreading information about the motivation and aims of the project and setting up important networks and collaborations for future outreach events. Our website user session times and overall user traffic were comparable to industrial benchmark standards. Furthermore, users from over 10 different countries interacted with the website. In the future, the high bounce rate observed on a specific set of website pages should be improved by re-designing and including interactive elements.   
    
    % chiara-give a summary of social media main conclusions
    

    



