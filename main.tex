%%%%%%%%%%%%%%%%%%%%%%%%%%%%%%%%%%%%%%%%%
% The Legrand Orange Book
% LaTeX Template
% Version 2.2 (30/3/17)
%
% This template has been downloaded from:
% http://www.LaTeXTemplates.com
%
% Original author:
% Mathias Legrand (legrand.mathias@gmail.com) with modifications by:
% Vel (vel@latextemplates.com)
%
% License:
% CC BY-NC-SA 3.0 (http://creativecommons.org/licenses/by-nc-sa/3.0/)
%
% Compiling this template:
% This template uses biber for its bibliography and makeindex for its index.
% When you first open the template, compile it from the command line with the 
% commands below to make sure your LaTeX distribution is configured correctly:
%
% 1) pdflatex main
% 2) makeindex main.idx -s StyleInd.ist
% 3) biber main
% 4) pdflatex main x 2
%
% After this, when you wish to update the bibliography/index use the appropriate
% command above and make sure to compile with pdflatex several times 
% afterwards to propagate your changes to the document.
%
% This template also uses a number of packages which may need to be
% updated to the newest versions for the template to compile. It is strongly
% recommended you update your LaTeX distribution if you have any
% compilation errors.
%
% Important note:
% Chapter heading images should have a 2:1 width:height ratio,
% e.g. 920px width and 460px height.
%
%%%%%%%%%%%%%%%%%%%%%%%%%%%%%%%%%%%%%%%%%

%----------------------------------------------------------------------------------------
%	PACKAGES AND OTHER DOCUMENT CONFIGURATIONS
%----------------------------------------------------------------------------------------
\documentclass[11pt,a4paper,fleqn, oneside]{book} % Default font size and left-justified equations

% jh2109 moved this here to define xcolor options before other packages load it internally without options
\usepackage[table,xcdraw]{xcolor} 


\usepackage{booktabs} % used for generating scientific looking tables
\usepackage{graphicx}
\usepackage{pdfpages}
\usepackage{array}
\usepackage{gensymb}
\newenvironment{conditions}
  {\par\vspace{\abovedisplayskip}\noindent\begin{tabular}{>{$}l<{$} @{${}={}$} l}}
  {\end{tabular}\par\vspace{\belowdisplayskip}}
%----------------------------------------------------------------------------------------
\usepackage{enumitem}
\newlist{arrowlist}{itemize}{1}
\setlist[arrowlist]{label=$\Rightarrow$}

\input{structure} % Insert the commands.tex file which contains the majority of the structure behind the template

\makeglossaries
 
 % Acronyms added by Erika
 
\newacronym{api}{API}{application programming interface}

\newacronym{gis}{GIS}{geographic information system}

\newacronym{sql}{SQL}{structured query language}

\newacronym{gps}{GPS}{global positioning system}

\newacronym{qr}{QR}{Quick Response code}

\newacronym{sdk}{SDK}{software development kit}

\newacronym{rfid}{RFID}{radio-frequency identification}

\newacronym{os}{OS}{operating system}

\newacronym{mvp}{MVP}{minimum viable product}

\newacronym{ngo}{NGO}{non-governmental organisation}

\newacronym{csv}{CSV}{comma separated values}

\newacronym{json}{JSON}{JavaScript Object Notation}

\newacronym{paas}{PaaS}{Platform as a Service}

\newacronym{cad}{CAD}{computer aided design}

%%general
\newacronym{UCPP}{UCPP}{Umzimvubu Catchment Partnership Programme}
\newacronym{CGE}{CGE}{Centre for Global Equality}
\newacronym{EPSRC}{EPSRC}{Engineering and Physical Sciences Research Council}
\newacronym{SA}{SA}{South Africa}

%%For Elena; added by Chyi
\newacronym{CO2}{CO$_2$}{carbon dioxide}
\newacronym{SOM}{SOM}{soil organic matter}

%%C-3NpH
\newacronym{EC}{EC}{electrochemistry}

\newacronym{UHP}{UHP-H$_2$O}{ultra-high pressure}
\newacronym{BC}{BC}{bench-top centrifuge}

\newacronym{PR}{PR}{polynomial regression}

\newacronym{KNN}{k-NN}{k-nearest neighbours}
\newacronym{WG}{WG}{Whirligig}
\newacronym{FFF}{FFF}{fused filament fabrication}
\newacronym{ISE}{ISE}{ion-selective electrode}


\newacronym{QNT}{QNT}{quick nitrates test}
%%%%%%%%%%%%Outreach: Added by chyi
\newacronym{BOM}{BOM}{bill of materials}
\newacronym{CEB}{CEB}{Chemical Engineering and Biotechnology}


%-------Main device hardware---------
\newacronym{OOPI}{OOPI}{object oriented peripheral interface}
\newacronym{OOP}{OOP}{object oriented paradigm}
\newacronym{SPI}{SPI}{serial peripheral interface}
\newacronym{pcb}{PCB}{printed circuit board}
\newacronym{gui}{GUI}{graphic user interface}
\newacronym{LOI}{LOI}{loss on ignition}
\newacronym{SOC}{SOC}{soil organic carbon}
\newacronym{TOC}{TOC}{total organic carbon}
\newacronym{IR}{IR}{infrared}
\newacronym{NIR}{NIR}{near infrared}
\newacronym{MIR}{MIR}{mid infrared}
\newacronym{ATR}{ATR}{attenuated total reflection}
\newacronym{MEMS}{MEMS}{microelectromechanical systems}
\newacronym{USB}{USB}{universal serial bus}
\newacronym{IDE}{IDE}{integrated development environment}

%-----------Moisture Retention Probe----------

\newacronym{FEM}{FEM}{finite element modelling}
\newacronym{TDR}{TDR}{time domain reflectrometry}

%-----------citizen engagement----------

\newacronym{STEM}{STEM}{Science Technology Engineering and Maths}


\begin{document}

%----------------------------------------------------------------------------------------
%	TITLE PAGE
%----------------------------------------------------------------------------------------

\begingroup
\thispagestyle{empty}


	\begin{center}
	
	%		\vspace*{1cm}
	
	\parbox[t][4cm]{\textwidth}{
		%		\begin{minipage}{\textwidth}	
		\begin{center}
			\addtolength{\baselineskip}{0.5\baselineskip}
			{\huge\textsc{SoliCamb: An open source soil health monitoring platform}} %fontsize{22pt}{24pt}\selectfont0
			
			\vspace{2em}
			
			{\Large \textsc{T. Ba{\"i}ssas, C. Barberio, S. L. Barron, E. Bondareva, \\ C. W. Chung, K. Gibson, J. Heck, J. T. Meech, \\S. Middya, S. Pavagada, E. Sch{\"a}fer,  X. Tan, \\ D. C. van Niekerk, B. Woodington, Y. Wu}} %\fontsize{18pt}{20pt}\selectfont

			\vspace{10em}
			
						
		\end{center}
	}


	%		\end{minipage}
	\vspace{5em}%\vspace{\fill}
	
	\includegraphics[height=50mm]{Pictures/title_page/coat_of_arms.pdf}
	
	\vspace{8em}
	
	\parbox[t]{\textwidth}{
		
		\begin{center}
			
			{\Large\bfseries MRes Team Challenge} %\fontsize{16pt}{18pt}\selectfont
			
			\vspace{1em}
			
			{\Large Centre for Doctoral Training \\ 
				\vspace*{5pt} in Sensor Technologies for a Healthy and Sustainable Future} %\fontsize{16pt}{18pt}\selectfont 
			
			\vspace{1em}
			
			{\Large Department of Chemical Engineering \& Biotechnology} %\fontsize{16pt}{18pt}\selectfont
			
			
			\vspace{1em}
			
			{\Large University of Cambridge} %\fontsize{16pt}{18pt}\selectfont
			
			\vspace{1em}
			
			%{\large \@supervisor}
			%{\large \@college\hfill \hfill \@date} %\fontsize{14pt}{16pt}\selectfont
			
			\vspace{1em}
			
			{\large \today}
			
		\end{center}		
	}
	
	\vspace{1em}
	\includegraphics[height=20mm]{Pictures/title_page/sensor_cdt.png}
	
\end{center}
	
	
\clearpage




%\begin{tikzpicture}[remember picture,overlay]
%\node[inner sep=0pt] (background) at (current page.center) {\includegraphics[width=\paperwidth]{background}};
%\draw (current page.center) node [fill=ocre!30!white,fill opacity=0.6,text opacity=1,inner sep=1cm]{\Huge\centering\bfseries\sffamily\parbox[c][][t]{\paperwidth}{\centering SoliCamb\\[15pt] % Book title
%{\Large An open source soil health monitoring platform}\\[20pt] % Subtitle
%{\huge T. Baissas, C. Barberio, S. L. Barron, E. Bondareva, C. W. Chung, K. A. W. Gibson, J. Heck, J. T. Meech, S. Middya, S. S. Pavagada, E. Sch$\ddot{a}$efer,  X. Tan, D. C. van Niekerk, B. J. Woodington, Y. Wu}}}; % Author name
%\end{tikzpicture}
%\vfill
%%\endgroup
%


%----------------------------------------------------------------------------------------
%	TABLE OF CONTENTS
%----------------------------------------------------------------------------------------

%\usechapterimagefalse % If you don't want to include a chapter image, use this to toggle images off - it can be enabled later with \usechapterimagetrue

\chapterimage{Pictures/title_page/tableofcontents.jpg} % Table of contents heading image

\pagestyle{empty} % No headers

\tableofcontents % Print the table of contents itself
%\listoftodos
%\cleardoublepage % Forces the first chapter to start on an odd page so it's on the right

\clearpage

\chapterimage{Pictures/chapter_heads/dictionary.png}
\printglossaries
%\todo{There wasn't a print glossary command; I added one here. Feel free to move -jh2109}

\pagestyle{fancy} % Print headers again

%----------------------------------------------------------------------------------------
%	PART
%----------------------------------------------------------------------------------------

%\part{Part One}
%\usechapterimagefalse
\chapterimage{Pictures/title_page/teamphoto.jpg} 

    \chapter*{Executive Summary}
    \setcounter{page}{1}
    \vspace{-1cm}
    In collaboration with the \gls{CGE}, this year’s Sensor CDT Team Challenge was proposed to address soil erosion due to topsoil loss in the Umzimvubu Catchment Area, South Africa. 
%%Stretching 200 km from its source in the Drakensberg Mountains to where it meets the Indian Ocean, the Umzimvubu river flows in parallel to the northern boundary of the Eastern Cape. It has a biodiverse mosaic of ecosystems, which supports more than 70 vulnerable species of flora and fauna.
In recent decades, this former Transkei homeland has seen significant topsoil erosion, primarily due to farming malpractices and the invasion of alien plant species. Soil erosion, however, is not restricted to South Africa: it is reported that the equivalent of 30 football fields of fertile topsoil is lost due to erosion, every day across the world. \cite{saveoursoils} This is primarily driven by the pressures of rapid urbanisation and socio-economic development, for which Umzimvubu forms a classic example. Noting that 99.7$\%$ of our food comes directly from the soil \cite{saveoursoils}, the severe implications of topsoil loss become forefront and highlight the global pertinence of this project. 

SoliCamb has developed a simple, low-cost, modular platform for quantitative measurements indicative of soil quality to be made in-field; this, in turn, enables rapid evaluation of mitigating practices at a local level. The base unit provides the display for the user; it is equipped with data-logging, and time and GPS stamping capabilities. Moreover, it allows the user to attach different peripheral sensors; the plug-and-play nature of the platform leaves scope for further addition of sensors in the future. Currently, two peripheral sensors (i.e. the moisture retention probe and C-3NpH) have been developed. Designed based on the request from the \gls{UCPP} for a measure of soil erosion, the moisture retention probe consists of capacitive sensors positioned at different depths on a rod; it gives the end user a moisture score indicative of irrigation requirement. On the other hand, C-3NpH is a colour sensor that provides qualitative analysis for nitrates and pH test strips; it is aimed towards a UK-based end user, interested in measuring soil health for crop-growing. Design considerations have been made to ensure the affordability (the whole system costs <\textsterling$70$, with the base unit making up 70$\%$) and user-friendliness of the final system. This is in line with the fact that SoliCamb aims to bridge the gap between laboratory and field measurements.  Acknowledging that some measurements (e.g. soil organic carbon) are unfeasible for field measurements, RFID-tagged bags are included to enable the end user to keep track of measurements when sent off for laboratory testing. An online web application with a ubiquitous Google Maps interface has been developed in conjunction; this allows users to input, access, and share data with ease.  Moreover, the open-source setup of the project means that users can access source code to better understand the science behind the system, and potentially build their own sensors and collaborate with SoliCamb. For validation of use, field testing with the complete system was performed in sites around Cambridgeshire, and collected data (including GPS, moisture score, nitrate levels) was successfully uploaded onto the online web application. 

Corroborating with the open-source philosophy, a defining aspect of this project was citizen engagement. SoliCamb participated in and organised events with Agri-Tech East and Cambridge Makespace respectively, where invaluable feedback for the ergonomic design of the base unit and the moisture retention probe was received. Moreover, C-3NpH was tested by students at the Immerse Cambridge Summer School, where it was found that there is minimal variability between different sensors. A host of advertisement strategies were used to promote the project. Within the Department, this consisted of circulating a weekly newsletter, handing out posters and business cards, and broadcasting a television advertisement. To engage the local Cambridgeshire community in soil health, SoliCamb was featured in a medley of media, which ranged from newsprint to radio and television. Moreover, online presence was established through social media and the website; this enabled outreach to a wider audience, with the website impressively receiving views from 10 different countries. 

Over the last 12 weeks, SoliCamb has forged a strong network consisting of academic and industrial collaborators, citizen scientists, and soil health-related charities. This will be useful for the future development of the project and the team; discussions have begun for potential expansion into real-time monitoring, smart agriculture, global development, and many other exciting avenues. \\

\begin{flushright}
\textbf{Team SoliCamb}
\end{flushright}


\begin{figure}[htb]
\centering
\includegraphics[width=0.4\linewidth]{Pictures/title_page/logo_noborder.png}
\end{figure}


    
\chapterimage{Pictures/title_page/ceb_acknowledgements.jpg} 
    \chapter*{Acknowledgements}
        \vspace{-1cm}
    We would sincerely like to thank: \\

\setlength{\leftskip}{0.5cm} \noindent The \gls{EPSRC} [grant number: $NQAG/041$, $RG71288$], for funding the project; \\

\setlength{\leftskip}{0.5cm} \noindent Lara Allen from the \gls{CGE}, for her vision, and constant guidance; \\

\setlength{\leftskip}{0.5cm} \noindent Our advisors, Philip Mair, Oliver Hadeler, Axel Zeitler, Toby Jackson, and Sebastien Cosnefroy, for their invaluable feedback and encouragement; \\

\setlength{\leftskip}{0.5cm} \noindent Karen Scrivener, for her organisational and logistical support; \\

%\setlength{\leftskip}{0.2cm} \noindent Sam Stanier, and Paul Flynn (British Soil Association), for their technical advice; \\
%Anyone else???

\setlength{\leftskip}{0.5cm} \noindent Researchers from various departments of the University, for their advice and generosity in lending equipment and facilities; \\

\setlength{\leftskip}{0.5cm} \noindent Agri-Tech East, Cambridge Makespace, and Cambridge Immerse, for collaborating with us on outreach events; \\

\setlength{\leftskip}{0.5cm} \noindent Cambridge Independent, BBC Radio Cambridge, Radio Cambridge 105, and That's TV Cambridge, for featuring us in print and on broadcast; \\ 
%Paul Brackley \& Keith Hoppel (Cambridge Independent), Sue Merchant (BBC Radio) and Neil Whiteside (Radio Cambridge 105), and Ben Baldwin (That's TV Cambridge), for featuring us in print and in broadcast; \\ 

\setlength{\leftskip}{0.5cm} \noindent Madingley Mulch Outdoor Supplies, Grange Farm (Lolworth), Anglesey Abbey, Radwell Grange Farm, and ChemTest, for facilitating soil testing; \\ 

\setlength{\leftskip}{0cm} \noindent And finally, everyone else, who has supported us over the last twelve weeks. 



%\usechapterimagetrue

%----------------------------------------------------------------------------------------
%	CHAPTER 1 - project outline
%----------------------------------------------------------------------------------------

\chapterimage{Pictures/title_page/portstjohns_rivermouth.jpg} % Chapter heading image

\chapter{Project Outline}
    \setcounter{page}{4}

\iffalse % comment out our instructions
%Jan
We can add editorial notes and TODOs with the \texttt{\textbackslash{}todo\{\}} command, and they show up in the margins as well as giving us a list of all TODOs still open in the document.\\

%Erika
If you would like to use an abbreviation and are not sure if it has ever been used in the report before, for convenience I've added a glossaries package. To use it, you should use command \texttt{\textbackslash{}gls\{\}}, and then the acronym will be defined if it has never been used before, and will be used as an abbreviation in all subsequent instances. To define a new acronym please go to the \texttt{glossaries.tex} file. For example, if I type \texttt{\textbackslash{}gls\{api\}}, the LaTeX would display \textbf{\gls{api}}.

For capitalised version in the beginning of a sentence, please use \texttt{\textbackslash{}Gls\{\}}. Example: \textbf{\Gls{gis}}.

For further info refer to \url{https://texblog.org/2014/01/15/glossary-and-list-of-acronyms-with-latex/}\\

%Erika
If you would like to use a number with a unit, please use command \texttt{\textbackslash{}SI\{\}\{\}}. For example, if I would like to say five joules, I would type in \texttt{\textbackslash{}SI\{5\}\{\textbackslash joule\}} and LaTeX would display \SI{5}{\joule}. You can also type in \texttt{\textbackslash{}SI\{5\}\{J\}} for the same result.

Read more here \url{http://mirror.ox.ac.uk/sites/ctan.org/macros/latex/contrib/siunitx/siunitx.pdf}.
\textbf{Check page 9 for unit macros - a list of units and how they can be called.}\\

%Erika
If you would like to insert a hyperlink, you can either use commands \texttt{\textbackslash{}url\{\}} or \texttt{\textbackslash{}href\{\}\{\}}. First inserts a clickable link, and second inserts a clickable defined string which will forward you to a link (first define link, then the clickable string).

Read more here: \url{https://www.latex-tutorial.com/tutorials/hyperlinks/}\\

%Doug
For referencing figures, tables, sections or anything with \textit{\textbackslash{}label\{...\}} the cleveref package is great! Normally, to reference a figure, say, one would type something like "fig.\textbackslash{}ref\{\textit{FigureLabel}\}". Instead, type: "\textbackslash{}cref\{\textit{FigureLabel}\}" and to reference multiple figures: "\textbackslash{}cref\{FigureLabel1,FigureLabel2,...\}". To capitalise the first letter of the of "fig." use a captial 'C', like: "\textbackslash{Cref\{...\}}". Remember that you can use this to refer to anything which has a \textbackslash{label\{\}} command added below it/in its environment. The package is hosted at \url{https://www.ctan.org/pkg/cleveref} and there is a little tutorial at \url{https://texblog.org/2013/05/06/cleveref-a-clever-way-to-reference-in-latex/}

\fi
    \vspace{-0.5cm}

\section{Motivations and aims}\index{Motivations and aims}

Under the remit of the Team Challenge 2019, there were two fundamental aspects of this project. First was to design, build and validate a sensor system capable of measuring specific parameters linked to soil health. This arose from a collaboration with the \gls{CGE} and the \gls{UCPP} in South Africa. This municipality has a history of soil erosion and degradation of soil health, from over-exploitation of the land in conjunction with the introduction of non-native plant species. UCPP has been in place for 6 years working on implementing new management strategies and stewardship programmes to mitigate the consequences of this previously unsustainable approach. Whilst this has been underway for some time, methods to determine the extent of soil erosion are qualitative only and require expert understanding of soil science for the interpretation of the results. This hinders the ability, at a local level, to rapidly instil management practices that improve soil health. To this end, SoliCamb sought to develop methods to track the erosion and quality of soil and thus provide an evaluation framework for their conservation practices. 

Simultaneous to sensor development, SoliCamb applied itself to public engagement, outreach and advertisement. A fundamental challenge was to understand problems concerning soil health and requirements to facilitate soil health management. Therefore, at the initiation of the project it was crucial to decide on a subset of soil health parameters that were feasible to build, and relevant to the UK, within the allotted time frame and budget. Outreach was the primary mechanism for finding this information and was used to direct the design of this sensor system towards sensing chemical properties of the soil such as pH and nutrients (\cref{Collaborators}). Effective communication was key and intended as an iterative process continuing throughout the 12 weeks. Public engagement, for a range of audiences, focused on obtaining information for design decisions as well as generating validation data for the nutrient sensing unit (\cref{Citizen engagement}). %Engagement events were proposed for a range of audiences and required thorough understanding and definition of the desired outcome. %These events were a key resource that this project sought to engage especially that involving Cambridge Makespace and local engineering experts.
Advertisement through social media and local press releases were proposed as the best way to achieve collaborations (\cref{Social media}). %with a significant workforce tasked with coordinating time frames and contributed to by making SoliCamb apparent within local networks. 

Bridging the two seemingly diverse focus points of the project outlined above, was the second aim of the team challenge. The impetus was to incorporate citizen science at the heart of the design and defining the value of the sensing system. The technology and the outreach aspects are joint through their symbiotic relationship with the end user. In practice, this means that at early stages of the development, outreach facilitates the information transfer from citizen scientist to the hardware while detailing the prerequisites and refining the prototyping sequence. Although over time information flows in the returning direction as the citizen scientist then relies on SoliCamb's outreach to stay updated with development, user protocols and translation of the expert science into the ``need to know'' manuals. %Therefore, building a valuable sensor is defined through the conversations guided by outreach and public engagement.
The sensor proposed in this work required a complementary method for data storage and visualisation, forming a link between data collected in-field and any additional data for which laboratory testing was more appropriate. %An RFID tagging system, used in combination with the web application, combines all measurements taken and therefore gives the user a holistic overview of their soil health. 

%\todo{ shifted this section and integrate it in User requirements, paragraph starting with: In the UK...}

%------------------------------------------------
\subsection{Soil health}\index{Soil health}
%Define soil health and explain its importance - looks good 
%Check to add moisture retention 
%Add a sentence on erosion
Life could not exist without soil; it provides the food, structure, nutrients, and habitat for the all ecosystems that depend on it.
%Soil is at the basis of living, encompassing food, fibre, habitat, shelter, clean air and water. 
Therefore, it can be described as a multi-component and multi-functional system, with definable parameters and diverse properties \cite{kibblewhite2007soil}. Soil health is defined not only by its physical characteristics, e.g. water retention and porosity, but also by certain chemical and biological parameters (e.g. the content of bacteria, fungi, earthworms, insects), that are essential for balanced ecosystems.
%it does not only consist of the physical constituent that it itself is made of, e.g. water retention and porosity, but also includes both different chemical and biological parts (e.g. bacteria, fungi, earthworms, insects).
For instance, soil microbes drive several crucial %soil
processes, such as nutrient recycling, storage, and release -- for plants' sustenance, as well as decomposition of organic matter, carbon sequestration, and nitrogen fixation \cite{chaparro2012manipulating}. 
Thereby, the continuous capacity of soil to function as a vital living ecosystem that sustains plants, animals, and humans (i.e. soil health), needs to be preserved.  %\cite{moebius2016comprehensive}.
By definition, a healthy and well-aggregated soil should be capable of sustaining productivity, within ecosystems, maintaining environmental quality, and promoting plant and animal health \cite{doran1994defining}. Moreover, healthy soil is significantly more resistant to adverse natural events, such as soil erosion (i.e. the displacement of topsoil caused by wind and rain), excess rainfall, and extreme drought. To date, several conservation practices are being employed to combat soil degradation, such as no-till planting, cover cropping, or crop rotation \cite{atkinson2019crop}. Although global efforts for soil health management are increasingly expanding into %the
traditional agriculture, conservation of the ecosystem while optimising agricultural yields remains a major challenge \cite{kibblewhite2007soil}. 



%------------------------------------------------
\subsection{User requirements}\index{User requirements}
%Soil health is an all encompassing term that perhaps glances over the intricate complexities that require thorough understanding of the physical nature of soil, in addition to understanding the wealth of interlinked chemistry and biology of soil. 
%It is not unreasonable to expect that production of a complete guide to all aspects of soil health was ambitious for a project of this scale. Therefore 

 \begin{figure}[h]
    \centering
    \includegraphics[width=\linewidth, trim={0 0 0 2cm},clip]{Pictures/Project_outline/solibridge.png}
    \captionsetup{justification = centering}
    \caption{Graphical representation of where SoliCamb aimed to position itself between the two extremes of existing soil health sensors: conventional field testing and laboratory standards. The objective was to bridge the gap and provide a solution that was the optimal compromise defined by the user requirements. }
    \label{fig:solibridge}
\end{figure}

It was important to define end users and project requirements prior to project initiation. 

Aligned with the \gls{UCPP} mission statement, a portable and low-cost ($<$\textsterling{100}) sensor capable of measuring soil nutrient level, soil pH, and moisture retention, a key indicating factor of the extent of erosion, was proposed. SoliCamb's technology  bridges the gap between existing low-cost but inaccurate field sensors and laboratory gold-standards, by exploiting the advantages of both extremes.  The device together with the data storage and visualisation system should be accessible to citizen scientists in South Africa.

 The end user in the UK was identified as a citizen scientist --- a person with enthusiasm for engineering and/or concern for soil health (e.g allotment holders, subsistence farmers, or engineering enthusiasts). %These were the intended end users in the UK that SoliCamb sought to target, where outreach efforts concluded that soil nutrient and moisture retention measurements were the parameters of interest.  

Through literature research, as well as input from experts in the field, the importance of \gls{SOC} content, as well as the lack of an in-field \gls{SOC} content sensor was realised.  However, due to the project constraints (Appendix \cref{SOC_feasibility}), \gls{SOC} monitoring was instead investigated in conjunction with sample tracking (\gls{rfid} labelling). This would permit lab-based \gls{SOC} measurements, e.g. \gls{LOI} results, to be cross-referenced with the data collected from the in-field soil sampling sites. 

% criteria outlined by the UCPP, in addition to the limitations of portable \gls{SOC} sensors favoured a 
 %This should form a modular system that could be tailored to the user requirements. %These requirements may also be applied to other developing countries. 




%open source and modular aims to be easily accessible and have the possibility of modifying the users chosen measurement parameters

% \begin{table}[h!]
 %   \centering
  %    %\resizebox{\textwidth}{!}{
   % \begin{tabular}{@{}lllll@{}}
    %\toprule
    %\textbf{User Requirement} &\ %\textbf{UCPP} &\ \textbf{UK} &\ %\textbf{SoliCamb} & \\
    %\midrule
    %\textbf{Robustness} &  High & Medium & High  \\
%    \textbf{Affordability} & High & Medium & High  \\
 %   \textbf{Portability} & High & Low & High  \\
  %  \textbf{Accuracy} & Medium & High & Medium  \\
   % \textbf{Ease of use} & High & Medium & High  \\
    %\bottomrule
    %\end{tabular}
    %    \caption{List of device requirements as specified by the two key end user groups in comparison to what SoliCamb aimed to provide. Priority of each requirement was categorised as high, medium or low.}
       % \label{tab:user}
        %}
        %\end{table}




%As aforementioned, Solicamb's partners within the CGE and UCPP who detailed an end user as a citizen scientist or subsistence farmer interested in soil erosion mitigation, and detailed issues that prevented continual and remote monitoring. 

% A high SOC indicates a better water holding capacity of the soil and hence will improve resistance to erosion. Therefore \gls{SOC} was the initial target as it ties together %the two aspects of 
%soil erosion management with the fundamental and wide ranging requirement of soil health monitoring. %Hence, determination of SOC was the initial target. 
%There was good evidence in the literature that supported the proposed value of developing a \gls{SOC} sensor
%Primary evidence in the literature had also been promising. 
%However, critical assessment of the feasibility of developing this, %idea, especially 
%considering the time constraint of this project, %proved otherwise
%meant that the capability for measuring \gls{SOC} was investigated in conjunction with labelling methods to allow lab based measurement if required. Comparison of different SOC %estimation 
%measurement techniques and reasons for not pursuing any of those avenues is explained in the Appendix \Cref{SOC_feasibility}.

%Instead it was decided to build p Encompassed within sensor design and development there was a desire to apply the sensor to local needs. %Once valuable data is acquired, it needs to be stored and represented intuitively for insightful analysis.
%The extremely complex nature of soil also forecasts that measuring each and every relevant parameter in the field is an impossibility. %This being said, measuring moisture retention property of soil or its pH and nitrate content is too little information to gain a comprehensive knowledge of soil health. 
%Again, few laboratory-based tests, for example, \gls{LOI} tests for SOC, are too inexpensive and standard to be replaced by in-field sensors. Hence, sending off soil samples for laboratory analysis is often the wisest choice. 
 

%------------------------------------------------
\section{Project management}\index{Project management}

Instead of going for a more conservative autocratic management system, the team has agreed to develop a variation of the scrum project management framework. It allowed the team to focus on the product to be delivered by the end of the project. Three work-streams emerged based on the specifications of the project: hardware (with three sub-work-streams: base unit, moisture retention probe, and C-3NpH), software, and outreach. A backlog\footnote{Backlog -- in scrum project management, a list of goals for the project: most often it is a list of features that the product must have. An important part of managing backlog is prioritising the most valuable features first.} was created together with the leaders from each of the work-streams for the whole duration of the project. Based on the backlog a list of tasks to be accomplished was created, and a board was set up to visualise the project progression (\cref{fig:board}). 

\begin{figure}[h]
    \centering
    \includegraphics[width = 0.75\textwidth]{Pictures/Project_outline/board.jpg}
    \captionsetup{justification=centering}
    \caption{A picture of the board prepared for effective project management, split into separate work-streams, and four columns: to do, doing, done, blocked. In each work-stream, groups of tasks were created (groups can be seen on pink post-it notes, and the tasks on blue post-it notes).}
    \label{fig:board}
\end{figure}


%------------------------------------------------
\clearpage
\subsection{Allocation of tasks }\label{sec:Task_allocation}
\Cref{tab:tasks} shows the team structure and team members' involvement in various parts of the project.\\

\begin{table}[h]
\centering
\resizebox{\textwidth}{!}{\begin{tabular}{l l c c c c c c c c c c c c c c c}
\toprule 
 & & \rotatebox[origin=c]{90}{\small Ben} & \rotatebox[origin=c]{90}{\small Chiara} & \rotatebox[origin=c]{90}{\small Chyi} & \rotatebox[origin=c]{90}{\small Douglas} & \rotatebox[origin=c]{90}{\small Elena} & \rotatebox[origin=c]{90}{\small Erika} & \rotatebox[origin=c]{90}{\small James} & \rotatebox[origin=c]{90}{\small Jan} & \rotatebox[origin=c]{90}{\small Katie} & \rotatebox[origin=c]{90}{\small Sagnik} & \rotatebox[origin=c]{90}{\small Sarah} & \rotatebox[origin=c]{90}{\small Suraj} & \rotatebox[origin=c]{90}{\small Theo} & \rotatebox[origin=c]{90}{\small Xianglong} & \rotatebox[origin=c]{90}{\small Yafan} \\
\midrule
 \includegraphics[height=1em]{Pictures/Project_outline/pm.png}
  &\small Project management & & & & & \cellcolor{blue!25} &\cellcolor{blue!25} & \cellcolor{blue!25} & & \cellcolor{blue!25} & & & & \cellcolor{blue!25} & & \\
\midrule

\includegraphics[height=1em]{Pictures/Project_outline/hw.png} & \small \small Base unit & \cellcolor{teal!25} & & & \cellcolor{teal!25} & & & \cellcolor{teal!25} & \cellcolor{teal!25} & & \cellcolor{teal!25} & & & & & \\

& \small Moisture retention probe & \cellcolor{teal!25} & & & \cellcolor{teal!25} & & & \cellcolor{teal!25} & \cellcolor{teal!25} & & \cellcolor{teal!25} & & & \cellcolor{teal!25} & & \\

& \small C-3NpH & & & \cellcolor{teal!25} & \cellcolor{teal!25} & & & \cellcolor{teal!25} & & \cellcolor{teal!25} & \cellcolor{teal!25} & & \cellcolor{teal!25} & & \cellcolor{teal!25} & \\

& \small Whirligig & & \cellcolor{teal!25} & \cellcolor{teal!25} & \cellcolor{teal!25} & & & & & \cellcolor{teal!25} & & & \cellcolor{teal!25} & & & \\

& \small Field testing & \cellcolor{teal!25} & \cellcolor{teal!25} & \cellcolor{teal!25} & & & & & & \cellcolor{teal!25} & & \cellcolor{teal!25} & \cellcolor{teal!25} & \cellcolor{teal!25} & & \\

& \small Feasibility studies & \cellcolor{teal!25} & & & & \cellcolor{teal!25} & & & \cellcolor{teal!25} & & \cellcolor{teal!25} & & & \cellcolor{teal!25} & & \\
\midrule

\includegraphics[height=1em]{Pictures/Project_outline/sw.png} & \small Logo design & & & \cellcolor{cyan!25}& &\cellcolor{cyan!25} & \cellcolor{cyan!25}  & & & & & & & & & \\
& \small Web application & & & & & & \cellcolor{cyan!25}  & & & & & & & & & \\
\midrule

\includegraphics[height=1em]{Pictures/Project_outline/sm.png} & \small Newsletter & \cellcolor{olive!25} & & & & & & & & & & \cellcolor{olive!25} & & & & \\

& \small Setting up collaborations & \cellcolor{olive!25} &  \cellcolor{olive!25} & & & \cellcolor{olive!25}  & & &  \cellcolor{olive!25} & & & & \cellcolor{olive!25} &  \cellcolor{olive!25} & &  \\

& \small Social media & & \cellcolor{olive!25} & & & & & & & & & & & & &  \\

& \small Traditional advertisements & & & \cellcolor{olive!25} & & \cellcolor{olive!25} & & & & & & \cellcolor{olive!25} & & & &  \\

& \small Videos & & & \cellcolor{olive!25} & & \cellcolor{olive!25} & & & & \cellcolor{olive!25} & & \cellcolor{olive!25} & & & &  \\

& \small Website & & & & & & & & & & & \cellcolor{olive!25} & & \cellcolor{olive!25} & &  \\
\midrule

\includegraphics[height=1em]{Pictures/Project_outline/ma.png} & \small BBC Cambridge & & \cellcolor{lime!25} & & & & \cellcolor{lime!25} & & & & & & & & &  \\
& \small Cambridge 105 & & & & & & & & & \cellcolor{lime!25} & & & & \cellcolor{lime!25} & &  \\
& \small Cambridge Independent & & \cellcolor{lime!25} & & & & & & & \cellcolor{lime!25} & & & & & &  \\
& \small \small That's TV Cambridge & & & & & & & \cellcolor{lime!25} & & \cellcolor{lime!25} & & \cellcolor{lime!25} & & & &  \\

\midrule
\includegraphics[height=1em]{Pictures/Project_outline/events.png} & \small Agri-Tech Hack-a-thon & & & & \cellcolor{green!25}& \cellcolor{green!25} & & & \cellcolor{green!25} & & & & & \cellcolor{green!25} & & \cellcolor{green!25} \\
& \small Immerse Cambridge & \cellcolor{green!25} & \cellcolor{green!25} & \cellcolor{green!25} & & \cellcolor{green!25} & & & \cellcolor{green!25} & \cellcolor{green!25} & & & & & & \\
& \small Makespace Make-a-thon & & \cellcolor{green!25} & & \cellcolor{green!25} & & \cellcolor{green!25} & \cellcolor{green!25} & \cellcolor{green!25} & & \cellcolor{green!25} & & & & & \\
\bottomrule
\end{tabular}
}
\caption{Contributions of the team members to the project.}
\label{tab:tasks}
\end{table}


 %\begin{figure}[htb!]
%            \centering
%            \includegraphics[width=0.8\linewidth]{Pictures/Project_outline/Teamstructure.png}
%            \caption{Team roles}
%        \label{fig:team_roles}
%\end{figure}


%------------------------------------------------
\clearpage
\subsection{Project progression}\index{Project progression}\label{project_progression}
A Gantt chart demonstrating the project progression can be seen in \cref{fig:gantt}.

\begin{figure}[H]
    \centering
    \includegraphics[width = \textwidth, trim={1.5cm 2.5cm 1.5cm 1.5cm},clip]{Pictures/Project_outline/Gantt_chart.pdf}
    \captionsetup{justification=centering}
    \caption{A Gantt chart showing the tasks within each work-stream and the completion timeline.}
    \label{fig:gantt}
\end{figure}




%----------------------------------------------------------------------------------------
%	CHAPTER 2 - Science and technology
%----------------------------------------------------------------------------------------

\chapterimage{Pictures/chapter_heads/hardware_heading.jpeg} % Chapter heading image for hardware

\chapter{Science and Technology -- Hardware}
  \setcounter{page}{9}

\vspace{-1.5cm}
\section{Design concept}\index{Design concept}
%Talk about the modularity of our approach - Doug

	    \begin{figure}[ht]
		    \centering
		    \includegraphics[width=0.4\textwidth]{Pictures/Hardware/System.pdf}
		    \captionsetup {justification = centering}
			\caption{Simplified block diagram showing the two proposed peripheral sensors, communicating to the base unit using the \gls{SPI} communication protocol.}
			\label{fig:JM_System}
    	\end{figure}
	    

The complexity and variability of soil are such that a host of sensors is required to fully characterise any one instance of the substance. Different combinations of sensors may be utilised for different soil types or investigations. Incorporating all possible sensors into a single design is a task of prohibitive complexity while also being counter productive, as the system produced would be cumbersome and over-equipped for any one investigative effort. In order to address this, the concepts of flexibility through modularity and separation of concerns have been appropriated by the authors from classical computer science. In particular, the  architecture is one where a generic base unit encapsulates responsibilities common to any scenario, e.g. display, data logging, and \gls{gps} tagging; while interchangeable peripheral sensor modules encapsulate the responsibility of performing a given type of measurement, both in hardware (transduction) and software (processing). By obeying this separation of responsibilities and avoiding interdependent implementations, sensor modules compatible with the framework can be developed \textit{ad infinitum}, yielding a suite of sensors from which one may select a subset to tailor to the research task at hand. 

 Feasibility studies for a range of other approaches were also performed. Specifically, several strategies were reviewed regarding the measurement of \gls{SOC} content, e.g. spectroscopy (\cref{SOC_feasibility}), \gls{CO2} sensing (\cref{CO2sensor}).

The current implementation includes two sensor modules: a capacitive moisture retention probe to measure the penetration of water into the soil, as well as a colour sensor to objectively determine the soil pH and nitrate concentration from colour strips (\cref{fig:JM_System}).

%\subsection{Whole system (high level description) } 

   %% Figure: photo of system (Writer Doug) 


 %%Monitoring soil health involve measuring multiple properties of soil. Considering the complex nature of soil itself, there is no limit to the number of parameters that should be targeted for a comprehensive understanding.
 
% In order to get an idea of how erosion-prone the soil is, its capability to retain water was identified as a promising metric. In that direction, a moisture retention probe has been designed that measures the penetration of water into the soil when irrigated. Additionally, since top-soil loss is associated with depletion of soil nutrients, measuring the nitrate content and pH was prioritised. A strip based colorimetric sensor, C-3NpH, has been developed for this purpose. Apart from measuring these parameters, recording them alongside the geographic location and date is essential to track the effects and progress of erosion. Instead of designing a single device for all these capabilities, a modular design concept was followed. This is illustrated in the schematic diagram of Figure \ref{fig:JM_System}. The moisture retention and colour sensors were developed as separate entities which connect to a base unit equipped with the most important functions of geo-tagging and data recording. On one hand, this approach made the device easy to use, portable and extendable to other sensing applications in the future. On the other hand, it also made the device development independent and faster. 


%%\subsection{Basic mention of modular system - Doug}

%\subsubsection{Communication between the base unit and peripherals (Writer Sagnik + Doug) }

%The device design was based on the concept of modularity as described earlier. Such a design required a robust communication between the base unit unit and the peripherals. The SPI protocol was chosen due to its simpler implementation and capability to communicate over longer distances. Quite intuitively, the base unit serves as the SPI-master while the sensors are SPI-slaves. The user interacts only with the base unit and therefore, it is the master that directs the connected sensor when to initiate or abort a sensing operation. In SPI communication, all the slaves share the same data and clock lines but have individual slave-selects (SS). Thus, multiple slave means multiple SS which may complicate the hard-wired connections. On the master side, this translates to multiple used up output pins. But in case of our device the sensors connect through the same port one by one. Thus, only one SS connection is shared by all the sensors. Although the lack of parallelism may appear as a limitation, it makes the hardware simpler and above all, easy for the user.

     \begin{figure}[H]
        \centering
        \includegraphics[width=0.95\textwidth]{Pictures/Hardware/OOPI_Transaction.png}
        \vspace{5mm}
        \caption{Activity diagram illustrating the atomic transaction between master and slave. Red arrows indicate synchronised sub-atomic communication events while black arrows indicate the flow of control.}
        \label{fig:OOPI_Transaction}
    \end{figure}

\subsection{Object oriented peripheral interface (OOPI)} %Author: Doug 

The conventional approach to modular systems, consisting of a single ``Master'' and multiple ``Slaves'' (peripherals), is for the Master unit to take on the responsibilities of the peripherals, effectively rendering the latter to simple data accessors. The difficulty with this approach is that a strong interdependence (also referred to as coupling) is developed between the Master and the Slave with the proper functioning of both intimately linked to each other. Perhaps more distressing, however, is the generation of an unseemly amount of technical debt. In order to change the behaviour of a single sensor, or add a sensor to the suite of sensors compatible with the base unit, the entire firmware of both the base unit and the peripherals would need to be refactored. The same would be required for all pre-existing sensors due to the aforementioned interdependence. Furthermore, with each additional sensor, the firmware of the base unit would become increasingly monolithic, consuming increasing amounts of on-chip resources in an unsustainable manner. To avoid this, an alternative firmware framework, dubbed \gls{OOPI}, has been developed. This allows the base unit firmware to become independent of the sensor with which it is communicating by delegating all sensor-specific responsibilities to the sensor module itself. The primary value added by this firmware is its intrinsic flexibility, which allows existing sensors to be updated and new ones to be added to the suite, without the need to update the base unit. This supports the modular architecture of the system, and allows community-driven development of new sensors within the open-source context of this project.

There are three primary features within this framework. The first is the novel communications protocol on which the firmware is built. This protocol has been designed to be atomic, and therefore robust to data corruption and timing discrepancies between Master and Slave; the flow of control of a single transaction is shown in \cref{fig:OOPI_Transaction}. Secondly, \gls{OOPI} models sensors as smart entities, which carry out step-wise measurement procedures communicated to the base unit; \gls{OOPI} allows the sensor to issue instructions to the user using the base unit as a proxy, as well as wait for user confirmation. The flow of control for a measurement procedure is given in \cref{fig:OOPI_Measurement}. Finally, the base unit component of OOPI models the sensor as an object with a multiple-inheritance driven polymorphic interface, allowing robust and maintainable code (for additional information, see \cref{sec:OOPI_Appendix}) \footnote{Detailed documentation is hosted on GitHub: \url{https://github.com/solicamb/main/tree/master/object_oriented_peripheral_interphace}}.
\clearpage



%------------------------------------------------
\section{Base unit}\index{base unit}

The base unit was designed with a \gls{gps} and SD card built-in, as these components are indispensable for any soil health monitoring system. Whatever measurements are carried out, whether the ones developed within the scope of the project or any other parameter in the future, the data must be geo-tagged and recorded in memory. As outlined earlier, the samples collected from the field should be tracked; this could be achieved by means of labelling the samples with \gls{rfid}s. Thus, the base unit also has a \gls{rfid} reader. \gls{rfid} was chosen over a bar-code scanner as it is significantly cheaper (\pounds 1.30 vs \pounds 65)~\cite{JM_barcode}. Support for peripheral sensors was added in such a way that any new sensors could be easily integrated with the base unit without requiring any re-design. \Cref{fig:JM_Main} shows a block diagram of the base unit. The user interacts with the touch buttons and observes instructions on the screen; this abstracts away all of the complex operations that are required to take sensor readings and associate them with a timestamp, \gls{gps} location and \gls{rfid} tag. Among these functions, the accuracy of geo-tagging depends on the \gls{gps} module used. This was tested by recording the coordinates of the same location as read by the \gls{gps} over time. Detailed explanation is available in \cref{section:gps_accuracy}. \\

The base unit was designed to have the following features: 

\begin{itemize}
    \item battery charging via a \gls{USB} cable;
    \item reverse battery protection;
    \item fuse protection for all supplies;  
    \item rugged and waterproof design; 
    \item support of two peripheral sensors; 
    \item minimum 32-hour battery life;
    \item typical 90-hour battery life.
\end{itemize}

\Cref{fig:JM_Main} shows a schematic diagram of different parts of the base unit, illustrating the communications among them. %\Cref{fig:front_view} depicts an image of the base unit as viewed from the front. 
\Cref{subfig:labelled_1} and \ref{subfig:labelled_2} show labelled images of the hardware components on the front and reverse side of the base unit. \Cref{subfig:labelled_3} shows the image of the \gls{gps} antenna and \gls{USB} ports on the top.

\begin{figure}[h]
\centering
\includegraphics[width=0.9\textwidth]{Pictures/Hardware/MainDevice.pdf}
\captionsetup{justification = centering}
\caption{Base unit system block diagram, all the modules within the  and the communication links between them are shown.}
\label{fig:JM_Main}
\end{figure} 

%\begin{figure}[h]
%\centering
%\includegraphics[height=7cm]{Pictures/Hardware/JM_FrontLidOn.jpg}
%\caption{Front view of the base unit.}
%\label{fig:front_view}
%\end{figure} 

\begin{figure} [t]
\centering
\begin{subfigure}{\linewidth}
  \centering
  \includegraphics[width = 0.6\textwidth]{Pictures/Hardware/labelled_device/labelled_front_view.jpg}
  \captionsetup{justification  = centering}
  \caption{Image of base unit and the components as seen after removing the front panel. The threaded metal inserts in the case mean that the front panel can be removed many times without the threads stripping.}
  \label{subfig:labelled_1}
\end{subfigure}
\bigskip
\begin{subfigure}{0.48\linewidth}
  \centering
  \includegraphics[width = 0.6\textwidth]{Pictures/Hardware/labelled_device/labelled_back_view.jpg}
  \captionsetup{justification  = centering}
  \caption{Image of the reverse side of the base unit showing the RFID reader and the batteries.}
  \label{subfig:labelled_2}
\end{subfigure}
\begin{subfigure}{0.48\linewidth}
  \centering
  \includegraphics[width = 0.6\textwidth]{Pictures/Hardware/labelled_device/antenna_view.jpg}
  \captionsetup{justification  = centering}
  \caption{Top view of the base unit showing the GPS antenna and waterproof double \gls{USB} port.}
  \label{subfig:labelled_3}
\end{subfigure}
\caption{Images of the main device and its features.}
\label{fig:pcb}
\end{figure}

%\todo{jh2109 I have removed all the figure float specifiers ht. This strict placement specification caused them to spill over long into the moisture retention probe section, which is quite confusing. Let's give latex some more freedom and places figures by hand if needed at the end of copyediting}


\subsection{Printed circuit board}

Prior to the design of the \gls{pcb}, the spatial layout of the breakout boards was determined by iteratively sketching their locations on a notepad. This approach is faster and more flexible than using a \gls{cad} tool. The PCB was designed with two layers, the bare minimum required for a design of this complexity. Two layer boards are significantly cheaper to manufacture than boards with three layers or more. Confining the design to two layers added significant complexity to the placement of components and routing of traces. Components were grouped and placed in small clusters to allow individual circuit modules to be routed and then connected to form the entire system. The power supply module was routed first by using thick power traces between the connectors, batteries, charging modules, and the regulator. The thick traces prevent large voltage drops across connections, which could cause the circuit to malfunction. Then, the thinner signal traces were routed out in buses to minimise the number of vias\footnote{Vias -- holes in a PCB used to jump a trace from one side of the board to the other.} required. Finally, the top side of the board was flood filled with a \SI{3.3}{V} copper pour to connect the power supplies of all of the breakout boards to \SI{3.3}{V} through relatively thick copper. The same was done for ground on the reverse side of the PCB. These tactics made it possible to integrate a complex circuit into a small form factor. Revision 3 of the PCB can be seen in \cref{fig:pcb}. It should be noted that the PCB design also included careful consideration and selection of every component on the \gls{BOM}~\cite{JM_BOM}.

Revision 4 through hole PCB was designed after the final presentation and supports charging on both \gls{USB}s and also a sensor peripheral on both \gls{USB}s. The re-design was required to make the base unit more user-friendly and easy to assemble independently for citizen scientists. The PCB diagram is available on GitHub \cite{JM_PCB}. Surface mount header was required to allow the traces to run underneath the Blue Pill and battery chargers on the reverse side of the board. It would not have been possible to make the base unit as compact otherwise.


\begin{figure}
\centering
	\begin{subfigure}[b]{0.48\linewidth}
		\centering
		\includegraphics[width=\textwidth]{Pictures/Hardware/PCB_Front.pdf}
		\captionsetup{justification = centering}
		\caption{Front side of the PCB.}
		\label{fig:JM_PCB_Front}
	\end{subfigure}
	\begin{subfigure}[b]{0.48\linewidth}
		\centering
		\includegraphics[width=\textwidth]{Pictures/Hardware/PCB_Back.pdf}
		\captionsetup{justification = centering}
		\caption{Back side of the PCB.}
		\label{fig:JM_PCB_Back}
	\end{subfigure}
\caption{Drawings of the \gls{pcb} Revision 3.}
\end{figure}


\subsection{Parts selection and cost} 

In general, the majority of component choices were driven by cost. Only some exceptions, such as the choice of capacitive touch sensors, the \gls{rfid} and \gls{gps} module, as well as the battery charger unit and the regulator breakout boards, were selected in interests of time. A single board design solution is possible, but was not feasible within the scope of the project. The bill of materials \gls{BOM} can be seen in \cref{Table:JM_BillOM}, and can also be found online \cite{JM_BOM}, to allow anyone to to rebuild the base unit.

By purchasing components in bulk or substituting pricey components in future designs, the cost of the device can be reduced further. Custom designing these modules would be another possible way to reduce the cost.

\begin{table}
\centering
\begin{tabular}{|l|l|l|l|}
\hline
\textbf{Description} & \textbf{Quantity} & \textbf{Price (£)} & \textbf{Total (£)} \\ \hline
\rowcolor[HTML]{adffca} 
100 Ohm Through Hole Resistor      & 2        & 0.06      & 0.12      \\ \hline
\rowcolor[HTML]{adffca} 
1k Ohm Through Hole Resistor       & 1        & 0.06      & 0.06      \\ \hline
\rowcolor[HTML]{adffca} 
470 Ohm Through Hole Resistor      & 1        & 0.06      & 0.06      \\ \hline
\rowcolor[HTML]{adffca} 
10 k Ohm Through Hole Resistor     & 3        & 0.06      & 0.18      \\ \hline
\rowcolor[HTML]{adffca} 
Flexible GPS Antenna               & 1        & 3.27      & 3.27      \\ \hline
\rowcolor[HTML]{adffca} 
8V 1.5A Re-setable Fuse            & 1        & 0.04      & 0.04      \\ \hline
\rowcolor[HTML]{adffca} 
6V 3A Re-setable  Fuse             & 1        & 0.19      & 0.19      \\ \hline
\rowcolor[HTML]{ffd49c} 
Micro SD Card                      & 1        & 3.89      & 3.89      \\ \hline
\rowcolor[HTML]{ffd49c} 
Momentary Capacitive  Touch Switch & 3        & 4.74      & 14.21     \\ \hline
\rowcolor[HTML]{adffca} 
Waterproof Enclosure               & 1        & 6.75      & 6.75      \\ \hline
\rowcolor[HTML]{adffca} 
SMD Header                         & 3        & 0.58      & 1.75      \\ \hline
\rowcolor[HTML]{adffca} 
PMOSFET                            & 3        & 0.36      & 1.08      \\ \hline
\rowcolor[HTML]{ff8585} 
Waterproof Double \gls{USB}              & 1        & 13.00     & 13.00     \\ \hline
\rowcolor[HTML]{ffd49c} 
Nokia 5110 LCD                     & 1        & 2.48      & 2.48      \\ \hline
\rowcolor[HTML]{ffd49c} 
Battery Charger                    & 2        & 1.00      & 2.00      \\ \hline
\rowcolor[HTML]{ff8585} 
Male to Female Jumper Wires        & 1        & 2.21      & 2.21      \\ \hline
\rowcolor[HTML]{ffd49c} 
Regulator Module                   & 1        & 6.00      & 6.00      \\ \hline
\rowcolor[HTML]{adffca} 
18650 Cell Holder                  & 2        & 1.99      & 3.97      \\ \hline
\rowcolor[HTML]{adffca} 
NMOSFET                            & 3        & 0.09      & 0.27      \\ \hline
\rowcolor[HTML]{ffd49c} 
Microcontroller                    & 1        & 2.82      & 2.82      \\ \hline
\rowcolor[HTML]{ffd49c} 
Micro SD Breakout Board            & 1        & 3.72      & 3.72      \\ \hline
\rowcolor[HTML]{adffca} 
Male Header                        & 1        & 0.29      & 0.29      \\ \hline
\rowcolor[HTML]{adffca} 
Female Header                      & 5        & 0.19      & 0.93      \\ \hline
\rowcolor[HTML]{adffca} 
18650 Cell                         & 2        & 3.24      & 6.48      \\ \hline
\rowcolor[HTML]{ffd49c} 
GPS Module                         & 1        & 6.71      & 6.71      \\ \hline
\rowcolor[HTML]{ffd49c} 
RFID Breakout                      & 1        & 1.30      & 1.30      \\ \hline
\rowcolor[HTML]{adffca} 
Switch                             & 1        & 0.58      & 0.58      \\ \hline
\rowcolor[HTML]{adffca} 
Switch Cover                       & 1        & 0.32      & 0.32      \\ \hline
\rowcolor[HTML]{adffca} 
47 uF Electrolytic Capacitor       & 2        & 0.05      & 0.10      \\ \hline
\end{tabular} 
\captionsetup{justification = centering}
\caption{\gls{BOM} where components highlighted in green will cost less in volume, in orange can be replaced with lower cost alternative with some design changes and in red should be substituted out of the design for a lower cost alternative.The total cost for a device is £77.74.} 
\label{Table:JM_BillOM}
\end{table}

A case study \gls{BOM} to prove that the claimed price reductions can be achieved can be seen in~\cite{JM_antenna}. The £$4.74$ capacitive touch switch cost can be reduced to £$0.96$ in one off quantity and £$0.65$ in 1000 quantity. \Cref{fig:JM_EconOfScale} shows that increasing the volume has diminishing returns for cost reduction. Assuming that all of the components highlighted in orange can be reduced to 14\% of their original cost (as the capacitive touch switch was shown to be) and that the components in red are substituted for components that cost 10\% of the original (as was shown for the capacitive touch buttons, the price for the redesigned  in 1000 quantity would be £$48.50$.

 \begin{figure}
		    \centering
		    \includegraphics[width=\linewidth]{Pictures/Hardware/EconOfScale.eps}
		    \captionsetup{justification = centering}
			\caption{The diminishing returns of economies of scale for the case study of a momentary capacitive touch switch.}
			\label{fig:JM_EconOfScale}
    	\end{figure}


\subsection{Base unit firmware}

The base unit and sensor modules are based on STM32 Blue Pill micro-controllers. These devices are compatible with the Arduino \gls{IDE} platform, which was used for developing the  firmware\footnote{The firmware is hosted on Github: \url{https://github.com/solicamb/main/tree/master/Main\%20\%20Code}}. The flowchart shown in \cref{firmware_flowchart} depicts the sequence of functions the Blue Pill runs through after the base unit is switched on. The red boxes indicate the processes that rely on input(s) from the user. A detailed explanation of the selection of different process flows is provided in \cref{appendix:processflow}.

\begin{figure}
            \centering
            \includegraphics[height=400pt]{Pictures/Hardware/firmware_flowchart.jpg}
            \captionsetup{justification = centering}
            \caption{Flowchart describing the sequence of operations performed by the  after it is started. The red boxes indicate processes that require interactions from the user.}
            \label{firmware_flowchart}
        \end{figure}

The base unit combines multiple functionalities, of which some are executed on their own, while others require an input from a user. For example, both geo-tagging by the \gls{gps} module and data logging onto the SD card operate in the background. Once a peripheral sensor is connected, measurements commence only when indicated by the user, due to the required initialisation time of the sensor unit. Since user has to interact with the sensor, the \gls{gui} becomes an indispensable part of the base unit. \Cref{subfig:gui_1,subfig:gui_2} show the starting display and menu. Options for the moisture retention probe and C-3NpH are valid only when those sensors are plugged in. Navigation through the menu options, selecting or cancelling an operation is performed using the touch buttons.

Selecting a sensor results in the connection being checked, after which sensor specific options or instructions are displayed (\cref{subfig:gui_3}). Brief step-wise instructions are meant to support and direct the in field testing procedure and increase the repeatability of the measurements. Representativly \Cref{subfig:gui_4,subfig:gui_5} show a few instructions displayed for colour sensing and moisture retention probing respectively. The way the results are displayed was arranged to achieve an intuitive understanding of the user (\cref{subfig:gui_6}).

As part of our citizen engagement study, suggestions and feedback on the user interface indicated that allowing the user to choose from different sensors is not anticipated to be effective. The concept of a plug-and-measure system is envisioned to first ask whether a specific sensor is required for the analysis or not, instead of providing options to the user in the form of a main menu as discussed earlier. Only when a peripheral sensor is connected, the sensor-specific screen appears displaying the instructions and eventually the results.

\begin{figure}[h]
	\centering
	\begin{subfigure}[t]{0.3\linewidth} 
		\centering
		\includegraphics[height=3.5cm]{Pictures/Hardware/GUI/gui_1.jpg}
		\captionsetup{justification = centering}
		\caption{SoliCamb logo displayed upon starting the device. }
		\label{subfig:gui_1}
	\end{subfigure}
	\begin{subfigure}[t]{0.3\linewidth}
	\centering
	\includegraphics[height=3.5cm]{Pictures/Hardware/GUI/gui_2.jpg}
	\captionsetup{justification = centering}
		\caption{Main menu displayed after the device is booted.}
		\label{subfig:gui_2}
	\end{subfigure}
	\begin{subfigure}[t]{0.3\linewidth}
	\centering
		\includegraphics[height=3.5cm]{Pictures/Hardware/GUI/gui_3.jpg}
		\captionsetup{justification = centering}
		\caption{Options provided upon C-3NpH colour sensor connection.}
		\label{subfig:gui_3}
	\end{subfigure}
	
	\bigskip
\begin{subfigure}[t]{0.3\linewidth}
  \centering
  \includegraphics[height=3.5cm]{Pictures/Hardware/GUI/gui_4.jpg}
  \captionsetup{justification = centering}
  \caption{Instruction to inserting a colour strip into the C-3NpH.}
  \label{subfig:gui_4}
\end{subfigure}
\begin{subfigure}[t]{0.3\linewidth}
  \centering
  \includegraphics[height=3.5cm]{Pictures/Hardware/GUI/gui_5.jpg}
  \captionsetup{justification = centering}
  \caption{Instruction to irrigate nearby soil once the moisture retention probe is inserted into the ground.}\label{subfig:gui_5}
\end{subfigure}
\begin{subfigure}[t]{0.3\linewidth}
  \centering
  \includegraphics[height=3.5cm]{Pictures/Hardware/GUI/gui_6.jpg}
  \captionsetup{justification = centering}
  \caption{Moisture retention scores displayed after the sensing ended.}\label{subfig:gui_6}
\end{subfigure}

	\captionsetup{justification = centering}
	\captionsetup{justification = centering}
	\caption{Pictures of the base unit screen with the GUI.}
	\label{fig:gui_all}
\end{figure}   




 
%------------------------------------------------
\clearpage
\section{Moisture retention probe}\index{Moisture retention probe}

\subsection{Background and idea}% Author: jh2109
\index{Erosion}
Consider one of the central issues of low soil health: erosion.\cite{soilvegetationsystemstrudgill1977} Whenever rain falls onto soil, there are naturally two paths it can take: into the ground, or running off along the surface. In the latter case, it will take up and carry along topsoil until it reaches a final destination at some lowest point of the local topology, often a river.\cite{unsaturatedsoilsresearchmancusoetal2012} This is an issue for two reasons. The topsoil it carries away is fertile ground that allows plants to grow. It might also contain recently sprouting plants, which are uprooted before they can establish themselves. This is a negative feedback cycle, as no stable ecosystem can form to improve soil health.

The other concern of soil erosion is that when this fertile topsoil meets a river, the river's ecosystem is also negatively affected by an oversupply of nutrients, as well as creating a mud bed within the river.\index{Water Infiltration} Hence, infiltration of water into the soil is a key indicator of soil health.\cite{soilvegetationsystemstrudgill1977} The soil moisture retention probe measures this quantitatively by sensing infiltration of water into the soil. Specifically, moisture content at three different depths is monitored, and from the time dynamics of these measurements as irrigation occurs either by the efforts of the scientist taking these measurements, or natural rain, the moisture retention probe can sense water infiltration.

\paragraph{Comparison of different sensing methods} %Author: jh2109
There are several sensing strategies that can be employed to detect soil moisture.\cite{surveymethodssoilschmuggeetal1980} The arguably simplest method of measuring soil moisture is a resistance measurement between two exposed wires acting as electrodes in the ground. Higher water content increases conductivity, hence a simple resistance measurement provides a way to track this parameter. There are key weaknesses to this technique that led to it being rejected for our efforts. Firstly, while moisture content changes resistance, so do many other factors, in particular salinity and soil composition (e.g. rocks will hence change the measurement significantly). Disentangling this compound measurement is unfeasible unless the system is to be restricted to a very uniform soil type, defeating the purpose of a soil health measurement toolkit.\cite{criticalreviewsoilsushalekshmietal2014} Second, exposed wires in moist soil are subject to corrosion. Furthermore, the necessary current sent through them to measure resistance is likely to cause electrolysis. Both these processes deteriorate the electrode, changing its contact resistance, thus giving a significant measurement drift over time.\cite{criteriasoilaggressivenessboothetal1967}

\index{Capacitive Moisture Sensing}
In comparison, capacitive moisture sensing is a more robust technique to measure moisture content. By assembling a capacitor with significant fringe field (see \gls{FEM} simulations below for a quantitative view), capacitance can be affected by the dielectric properties of the surrounding medium. This measurement does not require the electrodes to be exposed, giving a system with negligible change over time even when left in the ground continuously. Nonetheless, capacitive moisture sensing retains the drawback of also measuring effects of soil composition. Hence, it was decided from the start to work towards a system that can be self-calibrated in the field. This will be described in the sections below.

\index{Time Domain Reflectrometry}
For completeness, two other common techniques are to be mentioned. One of them is {\gls{TDR}}, where one or more wires led into the ground are driven with a series of electrical signals. The electrical reflections of these pulses can subsequently be analysed with transmission line models.\cite{coaxialmultiplexertimeevett1998}\cite{timedomainreflectrometrypettinellietal2002} It offers the simplicity of the sensing element in the ground (smallest possible footprint), which is merely a series of plain wires embedded in a plastic sheath; as well as direct measurements of depth-resolved moisture profiles by sweeping frequencies and knowledge of their attenuation. Two factors motivated our choice against this technique: the significant complexity of the analysis electronics (spectral analyzer), and the more complex calibration\cite{solutetransportwaterperssonetal2000}\cite{timedomainreflectrometrynicholetal2003}. This means such a system would not have the appeal of an easily \enquote{hackable} system (analog circuit design is generally seen as more intricate than digital), and would be less suited to portable (non-continuous) measurement toolkits.

\index{Neutron probe}
Perhaps the most advanced techniques to measure soil moisture revolve around the {use of neutron radiation}.\cite{neutronprobesoilhodnett1986} These approaches are able to offer real-time, large scale, volumetric 3D scans of a soil's moisture content as well as other properties.\cite{intraseasonaldynamicssoilhupetetal2002}\cite{influenceaccessholeabeele1979} However, their use of strong radiation sources combined with large and expensive detectors make them unsuitable for a project of our scale.

\subsection{Measurement protocol}
\label{smeasurementprotocol}


%DONE\todo{jh2109 explain how the measurement proceeds from the perspective of the user}
\index{Irrigation (Moisture Retention Probe)}
The moisture retention probe is primarily designed to sense and monitor water moving through the soil as opposed to static moisture content (albeit that this measurement is also returned to the user as an added benefit). In this design decision, it departs from commonly available moisture sensors, which aim for long-time (hours) averaging measurements, primarily to support gardeners and farmers in their decision when to artificially irrigate the field.

Consequently, the moisture retention probe relies on irrigation occurring during its measurement period. In most cases, this will be supplied by the field scientist, but it could also be replaced by natural rain in a continuous monitoring solution.

The complete procedure to take a moisture retention measurement is shown in \cref{fmeasurementprotocol}, and consists of four steps. First, at the desired measurement location, the work begins by \textbf{coring} a hole of the same diameter and depth as the probe. In our experiments, a metal rod was driven in either by hand or by using a mallet, depending on the condition of the soil. Additionally, using the same implement or a small shovel, a small well (diameter \SI{5}{cm}, depth \SI{5}{cm}) is made a short distance away (\SI{3}{cm}) from the probe. The lateral distance between the wells is designed to avoid percolation of water directly along the probe. On the other hand, if the secondary well is placed at a large distance away (further than \SI{5}{cm}), the infiltration of water towards the sensors becomes exceedingly slow to retrieve meaningful data regarding percolation rates. The optimal distance therefore depends on the depth of the three sensors on the probe and the type of soil. The value of \SI{3}{cm} was chosen as an acceptable compromise after several tests carried out in different fields. It could however be adjusted if needed simply by monitoring the time needed for water to reach the sensors on the probe.

\begin{figure}[b]
    \centering
    \includegraphics[width=\linewidth]{Pictures/moisture_retention/measurement_protocol.pdf}
    \captionsetup{justification = centering}
    \caption{Measurement protocol for the moisture retention probe. The user is guided through this process with the base unit's screen. A core is made by a suitable implement, e.g. a metal rod, to the same size as the probe. Next, the probe is inserted, and then returns the soil moisture depth profile. Another small well is made close-by, and kept topped up with water until the  returns a measurement.}
    \label{fmeasurementprotocol}
\end{figure}


The \textbf{probe is inserted into the ground}, as directed by the base unit's screen. Once the user confirms by press of a button that this step has been completed, the probe can give an indicative reading of the soil moisture depth profile. As discussed above, this measurement cannot be highly accurate without soil-specific calibration, and is therefore intentionally reported to the user as a semi-quantitative classification (\enquote{dry}, \enquote{moist}, \enquote{wet}) to avoid a false impression of higher accuracy. The source code gives the thresholds signal levels for these classifications, and was derived from the entirety of our field testing data. Absolute quantitative moisture measurement was seen as a stretch goal from the start, and would be better achieved with one of the more expensive technologies discussed above. Classification with words, rather than numbers with large associated error, allows us to nonetheless provide this information to the field scientist, while also avoiding misleading citizen scientist users who might not be accustomed to working with associated numerical uncertainty in a scientific context.

After showing this information to the user for a few seconds, the screen instructs the user to \textbf{begin irrigation}. The field scientist fills the well with water, and keeps it topped-up regularly. This gives a fairly constant pressure head of the water being driven into the soil. The original design was to add a fixed amount of water volume, but this was found to not work for all soil types tested. Very absorbent soils wick away the moisture in all directions before enough water can percolate down to allow a definitive measurement. Hence, to avoid the user having to make an a-priori, pre-measurement estimation of how much water to use, a constant pressure (water fill level) solution was chosen. The downside of this choice is an increased user involvement, and an uncertainty associated with topping up at different intervals.

From this point onwards, the user only needs to keep the well topped up, and the remaining measurement details are handled by the probe. Using pre-irrigation data, i.e. when it was inserted into the ground, it calibrates its signal measurements to give a baseline which is independent of the local soil type. Once the topmost probe registers a significant moisture reading (thresholds given in the source code), the measurement timer is started. This avoids the user having to interact with the  while irrigating, reducing potential usage errors.

In this way, the probe \textbf{monitors percolation}, and can give soil moisture velocities from the time differences between its multi-depth sensing elements. These are reported as a \enquote{moisture retention score}, with a value given in \SI{}{\centi\meter\per\minute}. The term \enquote{score} was chosen over the more technical term \enquote{soil moisture velocity}\cite{soilmoisturevelocityogdenetal2017,soilwaterretentionbuitenwerfetal2014} to indicate that higher values imply better soil health, and to avoid the citizen scientists making comparisons to literature values for soil moisture velocity. Since these measurements are usually carried out in a laboratory setting with a standardised set of conditions (pressure, temperature, and soil compaction), they measure the same concept but under different conditions.\cite{soilmoisturevelocityogdenetal2017} Our measurements take into consideration in-the-field conditions that typical soil moisture velocity testing intentionally tries to eliminate, so as to make results comparable between laboratories. These are different approaches, and hence must be differentiated properly; our measurements give a local, in-situ ground truth of how long water takes to move through the ground; laboratory testing of soil moisture velocity gives a well-defined moisture retention value for a given soil composition within a region of interest, which would then need to be corrected for local soil conditions. Both address somewhat different needs, and offer different points of view which must not be confused, hence the choice of the word \enquote{score}. Being open-source, the source code and its comments detail the physical meaning of the scores reported for any user who is interested in delving deeper into the science behind the developed sensor.\footnote{\url{https://github.com/solicamb/main}}

Finally, for very low quality soils, with slow ingress of water into the soil, it was found that the measurement could take over 10 minutes. Because the variability of soil health at the bottom of the spectrum is of less interest -- even visually, it is clear to see that almost all water added to them runs off and causes erosion --, a timeout of five minutes has been added. Measurements not completing within this period are then reported with the lowest possible moisture retention score of $1$. A common case in field testing was to find a quantitative reading for the topsoil (which is generally more porous even for lower quality fields due to shallow-rooting weeds), but encounter the timeout for lower layers of soil (as no deep-rooting plants were present to condition these strata). In this way, a meaningful measurement is still reported to the field scientist, indicating that different layers have different properties.

\subsection{Simulations and Analysis}

\subsubsection{FEM simulations} %Author: Doug
%\todo{Doug: Would like to include the FEM models to substantiate choice for coil form factor and further illustrate mechanism of operation. To be confirmed with team.}Good with me -jh2109

\begin{figure*}
    \centering
    \begin{subfigure}[b]{0.7\textwidth}
        \centering
        \includegraphics[width=\textwidth]{Pictures/moisture_retention/FEMM_Moisture_Probe_2.png}
        \captionsetup{justification = centering}
        \caption{Heat map illustrating the slice finite element modelling voltage field generated by the coil at a steady state voltage.}
        \label{fig:FEMM_Voltage}
    \end{subfigure}%
    \vspace{5mm}
    \begin{subfigure}[b]{0.7\textwidth}
        \centering
        \includegraphics[width=\textwidth]{Pictures/moisture_retention/FEMM_Capacitance_2.png}
        \captionsetup{justification = centering}
        \caption{Plot of the simulated change in capacitance as a function of horizontal and vertical distance between the simulated coil and a sphere of water.}
        \label{fig:FEMM_Capacitance}
    \end{subfigure}
    \captionsetup{justification = centering}
    \caption{Illustration of the finite element modelling results using FEMM 4.2 of a reduced version of the coil-form factor capacitor.}
\end{figure*}
%\todo{Could we have these in vector, or at least higher resolution? Then we could make them larger, too}


The functioning of the coil form-factor capacitive probe was investigated using the finite element modelling software FEMM 4.2. The electric field created by the probe at steady state is shown in \cref{fig:FEMM_Voltage}. This illustrates that the probe's form-factor allows for a pseudo-spherical field, which is more appropriate for measuring the presence of moisture at some radial distance away from the sensor. A high gradient is, however, observed in the field along the horizontal axis. Indeed, by examining the change in capacitance as a function of distance between the \textcolor{red}{sensor} and the water body it can be seen that the probe is more sensitive to moisture variation along its horizontal axis. Furthermore, the change in capacitance with distance appears to be monotonic (increasing for reducing distance) which implies easier interpretability of moisture variation along the horizontal plane. It is, however, also noted that the sensitivity of the \textcolor{red}{probe} is strongly non-linear, becoming highly sensitive at small distances from the sensor.

Finally, from the simulation, it can be shown that, choosing the capacitance of the sensor for an interferer at \si{5}{mm} to be the datum, the capacitance decreases to less than 5\% of the datum (and the sensitivity effectively to zero) when the water interferer is placed \si{40}{mm} from the sensor along the horizontal axis and \si{30}{mm} along the vertical axis. Thus the volume interrogated by the sensor can be approximated to be an ovaloid volume with long and short axes defined by the aforementioned 5\% values. 

%\todo{Could you add an estimate of our sensing region? ... small distances from the sensor, becoming negligible (<5\%)for distances larger than ... or something similar}  

\subsubsection{Extracting Soil Parameters from Probe Output} %Author: Doug

The measurement enacted by the moisture probe is rich in information; using parameters extracted from raw sensor data, the necessary mappings have been derived (\cref{sec:Extracting_Soil_params}) to calculate approximations of the infiltration rates and the hydraulic conductivity of the upper-most layer of the soil.

A simplified model for the infiltration of water into the soil from the water reservoir is that of radial dispersion under the influence of diffusion and gravity. This model is illustrated by \cref{fig:moisture_retention_sketch} and assumes the soil to be homogeneous. By considering the sensors and the reservoir to each be point elements in space, as illustrated in \cref{fig:moisture_retention_schematic}, it becomes clear that the volume of water which causes the primary deflection in each sensor must travel along some radial path with path length $\Delta r_i$ over the time interval $\Delta t_i$. The task is thus to derive the vertical infiltration velocity, $\frac{dy}{dt} \equiv \dot{y}$ as seen by each sensor. In particular, it can be shown that the infiltration rate, which is the vertical component of the velocity of the moisture front, at the depth of the $i^{th}$ sensor, is given by:

\begin{equation}
 \dot{y}_i \approx \frac{y_i\cdot x^2_{sw}}{\Delta t_i\cdot(x^2_{sw}+y^2_i)}
 \label{eqn:Infiltration_rate}
\end{equation}
where:
\begin{conditions}
 \dot{y}_i   &  Infiltration rate at the depth of the $i^{th}$ sensor $[m/s]$ \\
 y_i  &  Depth of the $i^{th}$ sensor $[m]$\\   
 x_{sw}    &  Horizontal distance between probe and reservoir $[m]$\\
 \Delta t_i & Time from the filling of the reservoir to the response of the $i^{th}$ sensor $[s]$
\end{conditions}

\begin{figure*}[h!]
\centering
\begin{subfigure}[b]{.75\textwidth}
  \centering
  \includegraphics[width=\textwidth]{Pictures/moisture_retention/Moisture_probe_sketch.png}
  \captionsetup{justification = centering}
  \caption{Stylised sketch of the moisture retention probe and water reservoir with radial infiltration of water from reservoir into soil.}
  \label{fig:moisture_retention_sketch}
\end{subfigure}%
\vspace{10mm}
\begin{subfigure}[b]{.75\textwidth}
  \centering
  \includegraphics[width=\textwidth]{Pictures/moisture_retention/Moisture_probe_Schematic.png}
  \captionsetup{justification = centering}
  \caption{Schematic illustrating the geometric relationships between the sensors and water reservoir, all modelled as point entities.}
  \label{fig:moisture_retention_schematic}
\end{subfigure}
\captionsetup{justification = centering}
\caption{Figures illustrating the simplified radial infiltration model.}
\label{fig:Moiture_retention_Radial_infiltration}
\end{figure*}

A comprehensive equation describing the vertical infiltration rate has been proposed by Ogden~\textit{et al.}~\cite{soilmoisturevelocityogdenetal2017} and describes the infiltration rate (approximated by \cref{eqn:Infiltration_rate}) as a function of the unsaturated hydraulic conductivity, $K$, the capillary pressure head, $\Psi$ and the soil water diffusivity, $D$, and is considered to consist of an advection-like term (defined by $K$ as a pre-factor) and a diffusion-like term (defined by $D$ as a prefactor.)

The hydraulic conductivity is a parameter of interest in the soil science community as it measures the ease with which water percolates through the soil and is therefore a reflection of the soil porosity and interconnectedness. 

By assuming that the diffusion-like term can be regarded as negligible relative to the advection-like term, which encapsulates the effect of porosity and gravity~\cite{soilmoisturevelocityogdenetal2017}, and by assuming that the water inflitrates radially, causing an inverse square decay of the capillary pressure head, it can be shown that the inflitration rate can be approximated by:

\begin{align}
    \dot{y} &\approx -C\cdot \frac{K}{y} + K 
    \label{eqn:Approx_y_dot}
\end{align}
where:
\begin{conditions}
 \dot{y}   &  Infiltration rate $[m/s]$ \\
 y  &  Depth at which the rate is observed $[m]$\\   
 x_{sw}    &  Horizontal distance between probe and reservoir $[m]$\\
  K     &  Unsaturated Hydraulic Conductivity $[m/s]$ \\
  C & Arbitrary constant $[]$
\end{conditions}
By making the appropriate approximations in \cref{eqn:Approx_y_dot}, the hydraulic soil conductivity for the soil surrounding the upper-most sensor can be approximated by:
\begin{align}
    K_1 &\approx \frac{\dot{y}_1}{\bigg(\frac{C_{Approx}}{y_1}-1\bigg)}\\
    \intertext{Where,}
    \begin{cases}
    C_{Approx} &\approx -(\dot{y}_2 - K_{Average})\frac{y_2}{K_{Average}}\\
     K_{Average} &\approx \dot{y}_3\\
     y_i &\equiv \text{\ref{eqn:Infiltration_rate}}
     \end{cases}
\end{align}
Which gives the Hydraulic Conductivity of the top soil layer, $K_1$, by using the deep sensors to estimate the interim parameters. It must, however, be noted, that the capillary head, hydraulic conductivity, and diffusivity are all non-linear functions of the volumetric water content of the soil. Given that the magnitude of the signal generated by the sensors is correlated to this parameter, adjustments to the measuring procedure using different volumes of water could be used to generate the appropriate functions.

%\todo{Doug could you add a figure of the final wire holders from your cad files :) -tb632}
\begin{figure}[h]
    \centering
    \includegraphics[width=0.55\textwidth]{Pictures/moisture_retention/Moisture_probe_Coil_mount.png}
    \caption{CAD image of the 3D-printed coil mounts used in the final moisture probe design.}
    \label{fig:moisture_probe_coil_mounts}
\end{figure}

\begin{figure}[H]
	\centerline{\includegraphics[width=.9\linewidth]{Pictures/moisture_retention/design_evolution.jpg}}
	\captionsetup{justification = centering}
	\captionof{figure}{A compilation of all the moisture retention probe designs, from earliest proof-of-concepts (left) to final working prototype (right). See \Cref{smrdesignevolution} for a complete explanation.}
	\label{fdesignevolution}
\end{figure}
%\todo{jh2109 show the different stages the  went through until its final iteration, explaining why different directions of development were taken}

\begin{figure}[H]
\centering
\begin{subfigure}[t]{0.6\textwidth}
	\includegraphics[height = 5.5cm]{Pictures/moisture_retention/ProbeCoils.jpeg}
	\captionsetup{justification = centering}
	\caption{Moisture retention probe connected to base unit: here the sensors at three different depths were made using different spacings on the 3D-printed sleeves for testing purposes. Each pair of wires soldered onto the readout part of a capacitive moisture probe also can be seen; this would normally sit within the peripheral sensor box.}
\end{subfigure}
\begin{subfigure}[t]{0.35\textwidth}
		\includegraphics[height = 5.5cm]{Pictures/moisture_retention/moisture_complete.jpg}
	\captionsetup{justification = centering}
	\caption{Moisture retention probe packaged inside an enclosure, with \gls{USB} to connect to the base unit. In the final version this \gls{USB} socket is also enclosed.}
\end{subfigure}
\caption{Pictures of the moisture retention peripheral.}
	\label{ffieldtest2b}
\end{figure}


\subsection{Testing and Validation}
Testing and validation of the sensor system was performed in multiple ways. In the early stages of the development, the response was measured within the lab by lowering the probe into a container of water, or by applying a damp sponge to the probe. Once a consistent response was observed, testing of soil analogues and local fields commenced.

During the development of the moisture sensor it was important to view the data output from each probe \enquote{live} during field testing. For this reason, a PC-based application was written. The PC interface allowed the user to monitor the voltage readout of each individual probe while in use (\cref{fmrpcscreenshot}). This allowed to gain an understanding of how \enquote{good} or \enquote{bad} soil responds to irrigation, how iterative design changes affected the data readout, and the degree to which voltage changes could be induced by the introduction of water via irrigation. A test algorithm was written within the PC app to provide the user with a time variable, measured as the time taken for irrigation to reach the various probe depths. An individual sensor registered the presence of irrigation by a defined drop in voltage. This voltage drop was predefined within the source code based on numerous field studies in different soil conditions. Two measurements are yielded by the current algorithm: the top soil result, given as $t_\text{mid sensor} - t_\text{top sensor}$, and the bottom soil result, given as $t_\text{bot sensor} - t_\text{mid sensor}$. This algorithm was later adapted for the base unit.

Unlike other peripheral sensors, benchmarking the moisture probe against a known measurement is difficult. The designed moisture probe is a novel sensor and by extension produces an essential, yet new parameter measurement which cannot be directly compared to anything.

There were several attempts to validate the \textcolor{red}{moisture retention probe}. Firstly, fractionated sand with a known granule size was used. This type of sand is commonly used within earth sciences as a research tool. There were practical limitations in using it as a benchmark within the lab. The main problem was the fact that any sufficiently small glass or plastic container caused a path of low resistance down the sides of the container. This resulted in water flowing preferentially down the sides of the container until reaching the bottom of the vessel and filling from the bottom, triggering a response at the bottom of the probe first. This is of course non-representative of a field test where the surface of the material being tested effectively extends to infinity.

It was decided to move to a less standard validation procedure, where fields with known quality of soil and various levels of management were tested. These sites were farms, National Trust grounds, and from the land surrounding the West Cambridge Campus. In speaking to land managers it was possible to assess to what degree the land had been managed, what had been grown there, how often the ground is watered, and even the last time the area experienced rainfall. Ensuring experimental control, such as the amount of rainfall a site had seen, was impossible, which added another level of difficulty to the validation experiments.

\begin{figure}
	\centerline{\includegraphics[width=.9\linewidth]{Pictures/moisture_retention/pc_app_screenshot.png}}
	\captionsetup{justification = centering}
	\captionof{figure}{A screenshot of the PC interface used during developmental field testing of the moisture probe. Readings from A0 (top sensor), A1 (mid-sensor) and A2 (bottom sensor) probes displayed graphically as three distinct colours with time shown on the X-axis and raw sensor reading (voltage) shown on the y-axis. }
	\label{fmrpcscreenshot}
\end{figure}

Once the probe was mechanically robust enough and provided repeatable data, the above functionality was integrated into the slave microcontroller within the moisture retention sensor system. Without the use of the PC interface, the system provides the user with a moisture retention \enquote{score} as described in \cref{smeasurementprotocol}.
This score is saved to the SD card in the base unit, alongside time and geographical coordinates.


\subsubsection{Field Testing} \label{sec:moistureprobe_field}
%\todo{bw422 - we need to add a bunch of images and plots here}
%\todo{jh2109 - happy to do the plotting with the python+seaborn I set up for the presentation. Just let me know which csvs to do. I have added the Anglesey Abbey data for now} 

\paragraph{West Cambridge Campus}
The initial field tests were performed in an area of very low quality soil, which is not professionally managed and not currently used for farming. This acted as a suitable test bed for early versions of the probe. Though poor quality, the grounds became useful for mechanical robustness testing of the probes. The ground was generally very compacted and clay based resulting in difficult test conditions. Only the top and mid-sensors responded due to two factors: poor quality compacted ground resulted in low infiltration rates, and in the soil holding a high volume of water nearly saturating the bottom sensor.


\paragraph{Anglesey Abbey}
Following interest from the National Trust, SoliCamb was invited to test sensor systems in the nursery gardens of Anglesey Abbey. This provided access to extensively managed soil. Two primary sites were tested: a site which had been recently ploughed and ready for planting, and a site that had been left bare for several months. It had rained recently which led to a near-saturated readings as soon as the probe was inserted into the ground. Despite this, it was possible to generate clear readings from both sites and to distinguishing between the fields, generating a filtration rate between all three sensor depths.

\begin{figure}
	\centerline{\includegraphics[width=.8\linewidth]{Pictures/moisture_retention/anglesey_abbey_recently_ploughed.pdf}}
	\captionsetup{justification = centering}
	\captionof{figure}{Anglesey Abbey: good quality soil. A clear, sharp response can be seen at the top sensor (\SIrange{0}{5}{cm}) due to fast infiltration. The mid-sensor (\SIrange{10}{15}{cm}) responds next, providing a time difference for the moisture retention score algorithm. The bottom sensor (\SIrange{20}{25}{cm}) responds last, with a greater time difference, indicating the poorer soil quality and increased moisture saturation at the \SIrange{20}{25}{cm} depth.}
	\label{fangleseyabbeyrecentlyploughed}
\end{figure}

\begin{figure}
	\centerline{\includegraphics[width=.8\linewidth]{Pictures/moisture_retention/anglesey_abbey_field_left_bare.pdf}}
	\captionsetup{justification = centering}
	\captionof{figure}{Anglesey Abbey: low quality soil. Signal artefacts at the beginning stem from the user walking across the sampled soil before the actual measurement was begun. The top sensor (\SIrange{0}{5}{cm}) begins to respond following irrigation, middle and bottom sensors (\SIrange{10}{15}{cm}, \SIrange{20}{25}{cm}, respectively) begin to slowly respond after 5-10 minutes. The base unit would report this with the worst possible moisture retention score of 1 to avoid the user being held up for too long.}
	\label{fangleseyabbeyleftbare}
\end{figure}


\paragraph{Radwell Grange Farm}
A collaboration that emerged through Cambridge 105 Radio feature led to further field testing of the sensor system on an crop farm which had recently been harvested. The probe was tested in two fields: a recently harvested rapeseed crop, and a barren field of the same soil type, left unmanaged for at least one season. Unfortunately, there had been extensive rainfall two days prior to the test and so the ground was saturated with moisture. All three sensors classified as ``wet'' when inserted into the ground. The top sensor responded reliably in both sites, while the mid-sensor provided a measurable, albeit weak response. The testing highlighted the inherent problems with using a point moisture sensor system in a country which experiences frequent periods of rainfall. However, these field tests did lead to the careful consideration of the application of a remote, constant monitoring probe within agriculture, which is discussed in more detail in \cref{smrpcontinuousmonitoring}.\\


\begin{figure}[H]
\centering
\begin{subfigure}[t]{0.35\textwidth}
\centering
	\includegraphics[height = 6cm]{Pictures/moisture_retention/fieldtest2a.jpg}
	\captionsetup{justification = centering}
	\caption{Early field testing showing the probe being implanted into the ground by a user.}
	\label{ffieldtest2a}
\end{subfigure}
\begin{subfigure}[t]{0.5\textwidth}
\centering
	\includegraphics[height = 6cm]{Pictures/moisture_retention/MoistureRetWell.jpeg}
	\captionsetup{justification = centering}
	\caption{Picture of field testing at Baldock Farm showing secondary well being topped up as the water diffuses through the soil in all directions.}
	\label{ffieldtest2b}
\end{subfigure}
\caption{Pictures from in-field testing of the moisture retention system.}
\end{figure}




\clearpage
\section{C-3NpH}
\subsection{Background and idea}
\subsubsection{Importance of soil nitrates and pH}
%%SURAJ%%
%Nitrogen is a vital nutrient for plant growth: it is a major part of chlorophyll (crucial for photosynthesis) and a key building block of proteins. It is also often the main limiting nutrient in soil, hence is supplemented by applying fertiliser.
Nitrates, a naturally occurring form of nitrogen in soil, are created from nitrification, i.e. conversion from ammonium. They are used as food by plants for growth and production. Soil nitrate levels vary widely, depending on soil type, climate conditions, rainfall, and fertilising practices (\cref{fig:soilnitrates}). In unfertilised or commercial crop soil, background nitrate levels range from \SI{5}-\SI{50}{ppm}; nitrates at \SI{20}-\SI{50}{ppm} are considered sufficient for growing plants in vegetable gardens. However, variation with moisture means that nitrate fluctuations exist depending on soil osmosis. Once nitrates levels are measured, the end user can decide on the amount of fertiliser to apply to the soil, based on recommended guidelines \cite{laqua}.


%begin{wrapfigure}{r}{7cm}
%	\centering
%	\includegraphics[width=\linewidth]{Pictures/C-3NpH/nitrates_1.png}
%	\caption{Nitrates chart \cite{Pattison2010}}
%	\label{fig:nitratechart}
%\end{wrapfigure}

\begin{figure}[h!]
	\centering
	\includegraphics[width=\linewidth]{Pictures/C-3NpH/nitratechart.png}
	\captionsetup{justification = centering}
	\caption{Understanding the impact of soil nitrates.}
	\label{fig:soilnitrates}
\end{figure}

Soil pH is another common measurement to evaluate soil chemical properties. It determines the availability of essential nutrients to the soil, and helps estimate the toxicity of other elements based on their known relationship to pH. Soil pH is affected by many factors, e.g. nature and type of inorganic and organic matter, amount and type of exchangeable ions, soil-solution ratio, salt or electrolyte content, and CO$_2$ content. Moreover, it influences the solubility of various compounds, the relative ionic bonding to exchange sites, and microbial activities. Soil pH may be used as a relative indicator of base saturation, depending on the predominant clay type. Optimal pH ranges differ between crop types; therefore, its routine monitoring is important for crop productivity and yield\cite{soilph}.

\subsubsection{Current detection approaches and C-3NpH} \label{sec:current_C3nph}

In a laboratory, nitrate testing is conducted by performing a colorimetric nitrate-reduction reaction, for spectrophotometer analysis (\cref{subfig:HACH}). Although the most accurate, its associated equipment is non-portable and expensive (\textsterling67/test \cite{hach}), and spent cuvettes require careful disposal (due to 2,6-dimethylphenol used). Hence, simpler test strips or \gls{ISE}-based probes are used instead for field testing. Nitrate test strips (\cref{subfig:DCSstrips}) are cheap (\textsterling0.16/test \cite{DCSnit}) and widely available. Typically quantified by eye against a colour chart, there is inherent subjectivity, especially when deciding between close shades. \gls{ISE}-based probes range widely in price and quality. They work by measuring the potential difference between an electrode specific to the desired detection species (i.e. nitrates), and a reference electrode. At the lower end of ISE probes, electrode corrosion is the prevailing issue. %, with a longevity of two weeks in the soil for \textit{in situ} monitoring. 
Those at the higher end (e.g. LAQUATwin, Horiba as shown in \cref{subfig:laqua}) cost hundreds of pounds, and typically require calibration against a known solution (e.g. distilled water) before each measurement.\cite{laqua} 

\begin{figure}[h!]
	\centering
	\begin{subfigure}[b]{0.4\linewidth}
		\centering
		\includegraphics[height=4cm]{Pictures/C-3NpH/Intro_HACH.png}
		\caption{HACH spectrophotometer}
		\label{subfig:HACH}
	\end{subfigure}
	\begin{subfigure}[b]{0.27\linewidth}
		\centering
		\includegraphics[height=4cm]{Pictures/C-3NpH/Intro_DCSnitratestrips.png}
		\caption{Nitrate test strips}
		\label{subfig:DCSstrips}
	\end{subfigure}
	\begin{subfigure}[b]{0.27\linewidth}
	\centering
		\includegraphics[height=4cm]{Pictures/C-3NpH/Intro_laquatwin.png}
		\caption{ISE probes}
		\label{subfig:laqua}
	\end{subfigure}
	\captionsetup{justification = centering}
	\caption{Conventional methods of nitrate sensing including (a) laboratory-based method of spectrophotometer (HACH DR3900) \cite{hach}, (b) simple test strips \cite{DCSnit}, and (c) \gls{ISE} probes \cite{laqua}}.
	\label{fig:nitmethods}
\end{figure}   	


Based on the principle of ISEs, pH meters are the most commonly used equipment for pH measurements. They require calibration in buffer solutions before each use, careful handling, and a reasonable amount of sample liquid ($\sim$\SI{10}{\ml}). Hence, test strips are a good alternative, due to their simpler protocol and low cost. 

C-3NpH provides quantification of colour strips, which allows for measurement of soil pH, nitrate concentration (and other parameters limited only by strip availability). %When connected to the base unit, it provides geo- and time-stamping, highly valuable for field testing. 
The simple dipping procedure only requires a small volume of liquid sample ($\sim$\SI{2}{\ml}), and gives rapid results. Taking into consideration advice from the \gls{UCPP} for affordable and single-point measurement sensors, the design of C-3NpH couples the use of cheap and commercially-available test strips, and a colour sensor for their quantification. This is achieved by creating mathematical models that map out the colour gradients of the strips. The final reading is delivered to the user via the base unit as a single digital reading. 

%------------------------------------------------

\subsubsection{Aims \& objectives} \label{sec:aimobj_c3nph}
%Chyi
C-3NpH aims to reduce the subjectivity found in conventional visual inspection of test strips. The device was developed as a low-cost, simple, and portable field sensor for nitrate detection. pH detection with C-3NpH was presented as an educational tool for aspiring engineers. Moreover, as soil sampling is required in the usage of C-3NpH, developing a set of protocols is essential for ensuring successful translation from laboratory to field. A hand-held centrifuge, \gls{WG}, was additionally investigated to eliminate settling variability across soils, in addition to allowing the user to expedite soil-supernatant separation process. \\

\textbf{Objectives for C-3NpH}
\begin{itemize}
    \item Construct and validate mathematical models based on standard solutions.
    \item Assess the precision of C-3NpH under the models.
\end{itemize}

\begin{figure}[H]
	\centering
	\begin{minipage}[c]{0.45\textwidth}
	\begin{tcolorbox}[width=\textwidth]
 	\noindent \textbf{Nitrates model}
        \begin{itemize}
            \item Develop field testing protocols.
            \item Perform field testing.
        \end{itemize}
        \end{tcolorbox}
	\end{minipage}
		\begin{minipage}[c]{0.45\textwidth}
		\begin{tcolorbox}[width=\textwidth]
	\noindent \textbf{pH model}
        \begin{itemize}
            \item Develop an educational tool.
            \item Assess sensor variability.
        \end{itemize}
        \end{tcolorbox}
	\end{minipage}
	\end{figure}



%------------------------------------------------
\clearpage
\subsection{Design decisions}

A detailed methodology for experiments can be found in \cref{C3nph_meth}. This section highlights the key decisions made when developing C-3NpH.

\subsubsection{Colour sensor (TCS34725 by Adafruit)} \label{sec:cs_c3nph}
%%CHYI%%

The colour sensing element consists of a photodiode array, with either red, green, blue (RGB), or clear filters, and a white LED. Raw frequency outputs from the RGB and clear channels are recorded using an external microprocessor. Initially, the TCS3200 model (manufactured by DFRobot) was tested; even though it excelled in distinguishing between primary colours, its sensitivity was too low for gradients of the same colour. The TCS34725 (Adafruit, \cref{fig:tcsada}), equipped with an infrared filter provided more meaningful and reproducible results; hence, it was adopted for the final design.\cite{ada}

\begin{figure}[h!]
	% \begin{figure}[h!]
	\centering
	\includegraphics[width=0.45\linewidth]{Pictures/C-3NpH/tcsada.png}
	\captionsetup{justification = centering}
	\caption{Colour Sensor (TCS34725) by Adafruit. \cite{ada}}
	\label{fig:tcsada}
	%\end{figure}
\end{figure}

\subsubsection{Test strips} \label{sec:ts_c3nph}

%%CHYI%%
Theoretically, C-3NpH is capable of quantifying any colour strip. For the initial proof of concept, nitrate (\cref{subfig:no3strips}) and pH strips (\cref{subfig:pHstrips}) were chosen based on strip dimensions, significance for soil health, and availability. Different strips were investigated, but excluded from further analysis (\cref{tab:strips}). 

\begin{figure}[h!]
	\centering
	\begin{subfigure}[b]{0.6\linewidth} 
		\centering
		\includegraphics[height=3cm]{Pictures/C-3NpH/no3strips_strips.png}
		\caption{Nitrates (DCS Products, UK)} 
		\label{subfig:no3strips}
	\end{subfigure}
	\begin{subfigure}[b]{0.35\linewidth}
	\centering
		\includegraphics[height=3cm]{Pictures/C-3NpH/pH_strips.png}
		\caption{pH (Cole-Parmer, USA)} 
		\label{subfig:pHstrips}
	\end{subfigure}
	\captionsetup{justification = centering}
	\caption{Nitrate test strips were dipped in increasing concentration (ppm) of nitrate standard solutions, while pH test strips were dipped in different pH standard solutions}
	\label{fig:teststrips}
\end{figure}  

	\begin{table}[H]
		\centering
		\begin{tabular}{ l  c  c  c  c } 
			\hline
			%\multicolumn{4}{c}{pH} \\
			Test strips & High cost & Colour saturation* & Thick colour pad** & Plastic sheeting*** \\ 
			\hline
			Ammonium & & $\diamond$ & & \\ 
			
			Phosphates & & $\diamond$ & $\diamond$ & $\diamond$\\
			
			pH & & $\diamond$ & $\diamond$ &\\
			
			Potassium & $\diamond$ & & &\\
			\hline
		\end{tabular}
		\captionsetup{justification = centering}
		\caption{Evaluation of test strips investigated (all were purchased from DCS Products, UK). \small [N.B. *Equilibriates to the same colour, no longer giving a true representation; **Too thick for the slot geometry in the current box design; ***Causes reflectance and uneven dye distribution was observed]}
		\label{tab:strips}
	\end{table}


\subsubsection{3D-printed enclosure}  \label{sec:enclosure_c3nph}
The RGB colour sensor is extremely sensitive to environmental variations (e.g. lighting conditions). Initial experiments, conducted with the sensor under ambient conditions, failed to achieve reproducible results. After a series of makeshift enclosures (\cref{fig:boxevo}), a 3D-printed black enclosure was designed (\cref{fig:3Dbox}). This facilitates consistent sensor-test strip interaction within a constant lighting environment; a constrained pathway additionally ensured the strip is held at a fixed height above the sensor.\\

\begin{figure}[H]
	\centering
	\begin{subfigure}[b]{0.55\linewidth} 
		\centering
		\includegraphics[height=3.5cm]{Pictures/C-3NpH/cup.png}
		\caption{First prototype -- paper cup model.}
		\label{subfig:proto1_cup}
	\end{subfigure}
	\begin{subfigure}[b]{0.4\linewidth}
	\centering
		\includegraphics[height=3.5cm]{Pictures/C-3NpH/tunnel.jpeg}
		\caption{Second prototype -- black tunnel model.}
		\label{subfig:proto2_tunnel}
	\end{subfigure}
	\captionsetup{justification = centering}
	\caption{Initial prototypes of enclosure for C-3NpH.}
	\label{fig:boxevo}
\end{figure}   	

\begin{figure}[H]
	\centering
	\includegraphics[width=\linewidth]{Pictures/C-3NpH/DesignDecisions_3DBox.png}
	\captionsetup{justification = centering}
	\caption{Final 3D-printed enclosure and its CAD drawing.}
	\label{fig:3Dbox}
\end{figure}
%move to discussion
%This is compatible with the ubiquitous Fused Filament Fabrication (\gls{FFF}) method (based on the extrusion of molten material), the primary limitation of which is the inability to construct parts which are comprised of unsupported segments.
 %this required in conjunction with the FFF method limitations motivated the use of the multi-part design shown in \cref{fig:3Dbox}.
%The fixed distance between sensor and test strip was substantiated for by the fact that the distance influences only signal intensity and not the information gained therein. 
%The essentially arbitrary distance to which the distance is set will therefore be compensated within the coefficients of any regression model trained with data collected at that distance.

\clearpage
\subsubsection{Prediction models}
For the nitrate test strips, a \gls{PR} model was applied. This describes the non-linear relationship between the independent and dependent variable, as an n$^{th}$ degree polynomial. In this work, a third-order polynomial was employed. 

\begin{figure}[h!]
	\centering 
	\includegraphics[width=0.55\linewidth]{Pictures/C-3NpH/pH_rgb.png}
	\captionsetup{justification = centering}
	\caption{Non-linear distribution of RGB of pH strips.\cite{Shen2012}}
	\label{fig:pHvar}
\end{figure}

For pH strips, polynomial regression was unsuitable due to complexity in establishing a clear trend between RGB and pH. This finding aligns with previous work by Shen \textit{et al.} \cite{Shen2012} (\cref{fig:pHvar}). Therefore, the \gls{KNN} model was used, where the average of \gls{KNN} to the input value is used to define the output, by matching input-output pairs from a stored database. %This is a common statistical model for classification and regression tasks. The 

\subsubsection{Soil sampling and separation}
A main challenge in developing in-field sampling protocol was analyte extraction from soil. A plethora of sampling protocols exist; moreover, sampling depth is commonly tailored to the crop of interest. Typical protocols involve digging to a depth of \SI{20}{cm} (\SI{8}{inches}) \cite{Mitchell2013}, and performing recommended analysis within \SI{24}{hours} of collection \cite{NaturalResourcesConservationService}. An important part of the project was to bridge highly precise laboratory results with fit-for-purpose field testing. Most laboratory procedures utilise air-dried soil and simply measure the moisture content prior to, and post-drying, to report a moisture corrected analyte concentration.\cite{ASTM1993}  However, previous studies have shown that air-dried and field-moist samples produce comparable results.\cite{Schmidhalter2005}  Therefore, field-moist samples were used to simplify the protocol in this work. When dipped in soil supernatant, staining of the test strip may occur. This could skew readings, which can be minimised by an optimised sampling protocol. % is required to minimise the effect. 

\begin{figure}[h!]
	\centering
	\begin{minipage}[t]{0.4\textwidth}
	\centering
	\includegraphics[height=4.7cm]{Pictures/C-3NpH/Whirligig.jpg}
	\captionsetup{justification = centering}
	\caption{3D-printed Whirligig.}
	\label{fig:whirli}
 	\end{minipage}
  	\begin{minipage}[t]{0.55\textwidth}
  	\centering
\includegraphics[height=4.7cm]{Pictures/C-3NpH/eppen_ratio_1.png}
\captionsetup{justification = centering}
	\caption{Soil-solvent ratio (i.e. 1:1-1:9) affects separation.}
	\label{fig:eppen_ratio1}
	\end{minipage}
	\end{figure}

 Byagathavalli \textit{et al.} \cite{Byagathvalli2019} detailed a 3D-fuge as a low-cost and field-appropriate method of centrifugation for DNA extraction and purification, reporting a maximum rotational frequency of \SI{6000}{rpm}. This concept was extrapolated for quick soil separation. The WG  (\cref{fig:whirli}) potentially provides a method to eliminate variability in, and hasten, soil sedimentation. In order to translate from laboratory to field, a large number of parameters for soil separation required investigation; primarily the effect of the WG against the laboratory standard of a \gls{BC} or settling. %The preferred solvent for extracting nitrate or pH vary considerably, however this work aimed to determine a common solvent. 

The low extraction efficiency of soluble salt can be overcome by using dilute salt solutions during extraction, e.g. CaCl$_2$ or KCl (instead of distilled water), which is a popular method for masking seasonal variation in soil pH. Conventional laboratory pH measurements are conducted with either \gls{UHP} or CaCl$_2$ (\SI{0.01}{M})\cite{VanLierop1981}. Conversely, the solvent of choice for nitrates is KCl (\SI{2}{M}).\cite{Pare1995} Given the chemical similarity between NaCl and KCl, in addition to the availability and ease of disposal of NaCl, this was selected as the common solvent. 

For optimal soil sampling, mixing time, and the impact of solvent choice and concentration was investigated. For nitrates, a 1:1 ratio had to be implemented so readings obtained fell within the detectable range of the strips (i.e. \SI{10}-\SI{250}{ppm}), despite difficulty in achieving a clear supernatant.

%For soil pH measurements, literature cited multiple options with varied factors of dilutions ranging from \SI{1:1} to \SI{1:9} (\cref{fig:eppen_ratio1}). \todo{(REFERENCE KG/Suraj)} An increase in soil-water ratio (or the presence of salts) generally decreases soil pH. 

%soil sedimentation variability thereby facilitating a common protocol regardless of location or soil type; in addition to allowing the user expedite soil:supernatant separation. 


%Initial research revealed a vast number of variables with a propensity to influence the accuracy of results. 
%Crucially, the sampling protocol should extract the desired analyte in a representative manner; in addition to this, the observed colour change must only depend on the analyte (and not interfering factors).
%Either; the sampling protocol may not correctly extract the analyte of interest in a representative manner (e.g. dry vs moist soil) or inefficient separation may give a coloured supernatant therefore obviously influencing the ability of C-3NpH to detect the 'correct' colour change. 
%Thus, there were queries surrounding the best method for the extraction of analyte from soil, whether nitrate or exchangeable protons (pH). 


 %It was supposed that disposal of NaCl or water would be simpler, and thus preferred. %Additionally, initial experiments indicated the faster and more efficient separation of soil supernatant with NaCl in comparison to water, and for the combination of these reasons NaCl was the solvent of choice. 

%The ratio of soil to solvent was also considered. 

% detection limits of colour strips dictated that a 1:1 ratio was employed despite this being the most difficult to separate into a clear supernatant and pellet. Experiments were therefore carried by settling, which prolonged the separation time but resulted in a clear supernatant suitable for dipping strips.


%------------------------------------------------



%------------------------------------------------

\subsection{Testing and validation -- Nitrates}

\subsubsection{Validating the accuracy of the PR model}
\noindent A colour map of the RGB values corresponding to the nitrate strips was produced by dipping test strips in increasing concentrations of standard solutions (\cref{fig:RGB_nitrates}). The values from triplicate strips indicated high variability in the raw output; this is believed to arise from the sensitivity of the sensor to strip placement. Despite C-3NpH’s bespoke 3D-printed enclosure, minor alteration in the placement of strips, and the angle at which they are placed (i.e. slightly bent strips), were potential sources of noise observed (shift in raw RGB measurements). As stated in (\cref{ratio}), to correct for this fluctuation, a more stable ratio of $R/(R+G)$ was introduced.
 
\begin{figure}[H]
	\centering
	\includegraphics[width=\linewidth, trim = {0 0 0 1cm}, clip]{Pictures/C-3NpH/Results_RGB_nitrates.png}
	\captionsetup{justification = centering}
	\caption{Raw RGB output for triplicate nitrate strips tested on standard solutions of 10-250 ppm.}
	\label{fig:RGB_nitrates}
\end{figure}	

\begin{figure}[H]
	\centering
	\begin{subfigure}[b]{0.48\linewidth} 
		\centering
		\includegraphics[width=\linewidth]{Pictures/C-3NpH/Results_ratio_nitrates.png}
		\caption{Effectiveness of defined ratio.}
		\label{subfig:RGB_nitrates}
	\end{subfigure}
	%\vfill
	\begin{subfigure}[b]{0.48\linewidth}
		\includegraphics[width=\linewidth]{Pictures/C-3NpH/Results_PRmodel_nitrates.png}
		\caption{Validation of PR model}
		\label{subfig:pr_nitrates}
	\end{subfigure}
	\captionsetup{justification = centering}
	\caption{Construction and validation of the mathematical model for nitrate strips.}
	\label{fig:nitrates_valid}
\end{figure}  

A plot of the ratio with nitrate concentrations (\cref{subfig:RGB_nitrates}) depicted a good polynomial curve with minimal variability. Hence, this was used to build a model based on a third-order polynomial regression, to predict nitrate concentrations. \Cref{subfig:pr_nitrates} demonstrates the accuracy of the developed model to predict nitrate concentration based on strip colour.

\subsubsection{In laboratory testing of soil}
\paragraph{Precision testing for nitrates} 
To demonstrate %C-3NpH 
precision, 3 soil samples were prepared, %for nitrate testing, 
10 nitrate strips dipped in each, and nitrate concentration measured by C-3NpH as per the protocol outlined in \cref{Nitrateproto}. % Strips were left for \SI{3}{min} to develop before insertion into C-3NpH.
%\Cref{fig:precision_nitrates} illustrates the 10 strips dipped in Sample 1, after they have been left to develop. 
The similarity in colour between all strips from sample 1 was confirmed by the quantitative measurements of C-3NpH (\cref{fig:precision_nitrates}). These results show an average variation of \SI{0.2}{ppm} between the 3 samples, confirming the high precision of C-3NpH in quantifying nitrate strip colour. 

\begin{figure}[h!]
	\centering
	\begin{minipage}[c]{0.5\textwidth}
		\centering
		%Table
		%\begin{table}
		\begin{tabular} {l c c}
			\toprule
			\textbf{Sample} & \textbf{C-3NpH (ppm)} & \textbf{\% error} \\
			\midrule
			1 & $13.3$ & 1.5 \\ 
			2 & $12.7$ & 1.6 \\ 
			3 & $13.3$ & 1.5 \\ 
			\midrule
			Average & $13.1$ & 1.5\\ 
			\bottomrule
		\end{tabular} 
		\label{table:nitrates_precision} 
	\end{minipage} 
	\begin{minipage}[c]{0.40\textwidth}
		\centering
		\includegraphics[width=\textwidth]{Pictures/C-3NpH/Results_precision_nitrates.png} \end{minipage}
			\captionsetup{justification = centering}
			\caption{Precision test with nitrate strips \small [N.B. \% error was calculated based on a minimum of 10 replicates.]} \label{fig:precision_nitrates}
\end{figure}

\paragraph{Testing different solvents}
Different solvents were tested for nitrate extraction from soil. Whilst industry extraction uses \SI{2}{M} KCl; here, NaCl was investigated in conjunction. Different solvents resulted in only a minor difference of \SI{2.2}{ppm} in the measured concentration (\cref{fig:diffsolv_nitrate}). The HACH spectrophotometer was used to further confirm NaCl suitability for extracting soil nitrates. The results showed a very high correlation between the values with only a \SI{0.2}{ppm} difference in measurements. The coloured rows in \cref{fig:diffsolv_nitrate} depict different nitrate bands used to choose remediation methods (e.g. quantity of fertiliser to apply). As shown, the values of nitrate obtained for both solvents, measured by C-3NpH and the HACH spectrophotometer, fell within the same nitrate band. Based on these results, \SI{2}{M} NaCl was chosen as the extractant to measure soil nitrate in field testing.


\begin{figure}[h!]
	\centering
	\includegraphics[width=0.8\linewidth]{Pictures/C-3NpH/Results_differentsolvent_nitrates.png}
		\captionsetup{justification = centering}
	\caption{C-3NpH readings for nitrates from Wolfson College soil extracted with different solvents (i.e. \SI{2}{M} NaCl and KCl). Nitrate level bandings taken from \cite{Pattison2010}.}
	\label{fig:diffsolv_nitrate}
\end{figure}

\paragraph{Testing different soils and composts} 
The next step in the validation of C-3NpH involved testing nitrates in soil from a variety of sources. The goal of the experiment was to test the ability of C-3NpH to produce an accurate representation of nitrate levels in different crop soils.
%, based on its comparison with the industrial standard spectrophotometer (HACH kit). 
Soil samples were obtained from sites in Cherry Hinton, Wolfson College, and Churchill College; they were tested for nitrate using C-3NpH and the HACH spectrophotometer. Results indicated only a minor difference between C-3NpH and the industrial standard, as follows: \SI{2.2}{ppm} (Cherry Hinton), \SI{5.3}{ppm} (Wolfson College), \SI{4.5}{ppm} (Churchill College). \Cref{fig:soilcomp} shows that the preliminary trend obtained by C-3NpH measurement was consistent to that from the HACH spectrophotometer. 
Another goal of this experiment was to validate the measurement range of C-3NpH, using different soil types (i.e. \SI{<25}{ppm}) and nitrate-rich compost; \cref{fig:soilcomp} demonstrated that compost was distinguished from crop soils as expected.

\begin{figure}
	\centering
	\includegraphics[width=\linewidth]{Pictures/C-3NpH/Results_differentsoilscompost_nitrates.png}
	\captionsetup{justification = centering}
	\caption{C-3NpH readings for nitrates in different soils and compost. The soils tested (highlighted in blue dotted box) were additionally compared to readings from the HACH spectrophotometer.}
	\label{fig:soilcomp}
\end{figure}

It was observed that C-3NpH values were consistent to those from the HACH spectrophotometer; where Cherry Hinton, Wolfson and Churchill College soils fell within the expected nitrate range (\cref{fig:soilcomp}). However, for nitrate-rich compost (\SI{>250}{ppm}), measurements from C-3NpH differed from the HACH spectrophotometer. This could be attributed to the fact that the regression model for C-3NpH was optimised using typical soil sample concentrations (\SI{<250}{ppm}). Another potential reason for the lack of correlation was heterogeneity in compost samples, as illustrated by the wide error bars. Samples are usually sieved prior to analysis so as to avoid this issue \cite{ISO2006}, and should be implemented in future work.  



\newpage
\subsubsection{Field testing}

%\begin{wrapfigure}{r}{10cm}
%	\centering
%	\includegraphics[width=\linewidth]{Pictures/C-3NpH/fieldtesting_nitr.pdf}
%	\caption{Field testing data (triplicate measurement for all of three sites were taken at each field) }
%	\label{fig:field_nit}
%\end{wrapfigure} 

Field testing for nitrates was conducted in Anglesey Abbey and at Radwell Grange Farm with two main objectives: C-3NpH required validation in a real-life setting (encompassing sample collection, preparation and measurement); and to assess end-user interactions for informing future designs
(further discussion is available in \cref{Sample Sites}).

C-3NpH was able to detect the significantly higher nitrate level in Anglesey Abbey nursery soil (\cref{table:nitratetesting}) than in the Dahlia garden. Also at Radwell Grange Farm, \cref{table:nitratetesting} considerably higher nitrate values were measured at the allotment (\SI{79.1}{ppm}), compared to the untended site (\SI{7.6}{ppm}). This was expected as the allotment is used to grow leguminous plants, associated with nitrogen-fixing bacteria. It was also noted that higher nitrate readings incurred larger variability. This could be a consequence of heterogeneous pad colour, arising from the sampling procedure; and requires further optimisation. 

\begin{figure}[h!]
	\centering
	\begin{minipage}[c]{0.45\textwidth}
	\centering
        \begin{tabular}{l c}
            \toprule
            \textbf{Anglesey Abbey} & Nitrates (in ppm) \\
            \midrule
            Dahlia garden &  $12.0 \pm 2.0$ \\
            Carbon-rich soil &  $18.0 \pm 0.8$ \\
            Nursery &  $38.0 \pm 4.0$ \\
            \bottomrule
        \end{tabular}
      %  \subcaption{}
 	\end{minipage}
 		\begin{minipage}[c]{0.45\textwidth}
	%\textbf{Radwell Grange Farm}
     \resizebox{\textwidth}{!}{
     \begin{tabular}{l c}
     \toprule
          \textbf{Radwell Grange Farm} & Nitrates (in ppm) \\
         \midrule
            Rapeseed &  $9.6 \pm 0.4$ \\
            Untended &  $7.6 \pm 0.8$ \\
            Wheat &  $12.1 \pm 1.2$ \\
            Allotment &  $85.6 \pm 12.6$ \\
            \bottomrule
        \end{tabular}
      %  \subcaption{}
        }
        \end{minipage}
        \captionsetup{justification = centering}
        \captionof{table}{C-3NpH readings from field testing in Anglesey Abbey and Radwell Grange Farm. \small [N.B. Standard error was calculated based on a minimum of 6 replicates.]}
        \label{table:nitratetesting}
	\end{figure}
	
%moved the following to outreach : At Anglesey Abbey, the citizen scientist had previous experience of horticultural training, and was especially interested in pH testing and being able to conduct the tests themselves on site. This contrasted to previous conversations with an arable farmer, who showed interest in obtaining the results, but not in performing the tests. Current field testing protocols were easy-to-follow, however, settling time (observed as \SI{20-30}{min} on the day) was cumbersome. Therefore, developments with the WG are proposed to be beneficial to the end user by reducing sampling time. Concerns raised by the citizen scientist at Anglesey Abbey pertained to cross-contamination between inserted strips inserted, and the expected long settling time incurred with water extraction.

\clearpage
\subsection{Testing and validation -- pH} \label{phresults_c3nph}
 
\subsubsection{Constructing and validating the k-NN model}
RGB values for pH strips dipped in standard solutions (\cref{subfig:RGB_pH}) did not follow as clear a trend as observed previously with the nitrate strips (\cref{subfig:RGB_nitrates}). This warranted fitting to a k-NN model, which gave a good correspondence when the model was re-validated with tested pH solutions (\cref{subfig:kNN_pH}). Initially, a range of pH {3}-\SI{9} was tested. However, the model failed to distinguish below pH {4} due to the similarity in hue between those strips. C-3NpH has a resolution of pH {0.5} (i.e. corresponding to interval of pH solutions tested); this is primarily limited by the test strip colour gradient, which has a resolution of pH 1. To improve this, narrow range strips can be used. The possibility of using these was investigated, but not pursued further as their pH ranges (i.e. pH {4.8}-\SI{6.2}) did not cover the spectrum of soil pH (i.e. pH {3}-\SI{9}).


\begin{figure}[h!]
	\centering
	\begin{subfigure}[b]{\linewidth} 
		\centering
		\includegraphics[height=7cm]{Pictures/C-3NpH/Results_RGB_pH.png}
		\caption{Raw RGB output from C-3NpH.}
		\label{subfig:RGB_pH}
	\end{subfigure}
	%\vfill
	\begin{subfigure}[b]{\linewidth}
	\centering
		\includegraphics[height=7cm]{Pictures/C-3NpH/Results_kNNmodel_pH.png}
		\caption{Validation of k-NN model.}
		\label{subfig:kNN_pH}
	\end{subfigure}
	\captionsetup{justification = centering}
	\caption{Construction and validation of the mathematical model for pH measurements.}
	\label{fig:knn}
\end{figure}   	

\newpage
\subsubsection{In laboratory testing of soil}

\paragraph{Precision testing for pH}
A similar precision test (to that carried out for nitrate strips) was repeated. The consistent pH readings by C-3NpH across three samples of 10
strips, and from by-eye comparison of the strips (\cref{fig:precision_pH}) highlighted the precision of C-3NpH.

\begin{figure}[h!]
	\centering
	\begin{minipage}[c]{0.5\textwidth}
		\centering
		%Table
		%\begin{table}
		\begin{tabular} {l c c}
			\toprule
			\textbf{Sample} & \textbf{pH from C-3NpH} & \textbf{\% error} \\
			\midrule
			1 & $5.2$ & 0 \\
			2 & $5.2$ & 0 \\
			3 & $5.6$ & 3.6 \\
			\midrule
			Average & $5.4$ & 1.9\\
			\bottomrule
		\end{tabular}
		\label{table:pH_precision}
		%\end{table}
	\end{minipage}
	\begin{minipage}[c]{0.4\textwidth}
		\centering
		\includegraphics[width=0.8\textwidth]{Pictures/C-3NpH/Results_precision_pH.png}
	\end{minipage}
	\captionsetup{justification = centering}
	\caption{Precision test with pH. \small [N.B. \% error was calculated based on a minimum of 6 replicates.]}
	
	\label{fig:precision_pH}
\end{figure}

\paragraph{Spiking experiment}

To investigate the sensitivity of C-3NpH, soil supernatant was spiked with solutions of different pH to induce an artificial pH range. This was successfully detected by C-3NpH, and further validated by the matching colour pad gradient observed (\cref{fig:pHspike}). It was observed that the resulting supernatants were at a higher pH than spiking solutions; this was to be expected as spiking solutions were inherently diluted. The experiment could be repeated with spiking solutions of lower pH to determine the sensitivity of C-3NpH over a wider range. 
Moreover, this simple experiment was still inadequate to determine soil staining effects (as the dipping supernatant was never clear, but a yellowish liquid), on the pH reading. A method of mitigating the effect of brownness is by dipping two strips in each sample: one with, and the other without pad chemistry. Hence, true reading of a stained colour pad can be calculated by subtracting the purely-stained RGB values (i.e. no pad chemistry) from the total (i.e. normal test strip). 

	\begin{figure}[h!]
		\centering
	\includegraphics[height=8.5cm]{Pictures/C-3NpH/Results_pHspiking.png}
	\captionsetup{justification = centering}
	\caption{C-3NpH is capable of detecting the tested pH range in soil.}
	\label{fig:pHspike}
	\end{figure}

\subsubsection{Applicability of pH model across different C-3NpH devices} \label{para:immerse}
C-3NpH was brought to the Immerse Cambridge Summer School. All students were able to assemble C-3NpH with ease, which highlighted the simplicity of the enclosure design. Results from pH measurements of household solutions are given in \cref{fig:summerdata}. The greater variability in by-eye measurements compared to those by C-3NpH further affirms its purpose in facilitating objective readings. This was especially evident for close shades of colour. Main errors from C-3NpH readings were due to not following protocol (e.g. inserting the strip the wrong side in, waiting too long before insertion).

\begin{figure}[h!]
	\centering
	\includegraphics[height=7.5cm, trim={0 0 0 1cm}, clip]{Pictures/C-3NpH/Discussions_data.png}
	\captionsetup{justification = centering}
	\caption{C-3NpH pH readings are more consistent than by-eye when tested on household solutions by summer school students.}  	
	\label{fig:summerdata}
\end{figure}

\clearpage
\subsection{Testing and validation -- Soil separation}
%The primary objective for the pH model was to use C-3NpH as an educational tool. However, due to the versatility of pH strips, they were used to investigate the benefits of the WG against settling as a separation method.

%\begin{figure}[h!]
%	\centering
%	\includegraphics[width=0.5\linewidth]{Pictures/C-3NpH/Results_whirlivssettling.pdf}
%	\caption{RGB values are similar for Lolworth Soil 2 when Whirligig or settled}
%	\label{fig:whirlivssettling}
%\end{figure}


\subsubsection{Solvent choice}
Four key solvents were used, i.e. NaCl, \gls{UHP}, CaCl$_2$ and KCl. Comparison of the WG to settling using \SI{2}{M} concentrations of each solvent was investigated, and compared by visual inspection of the colour strips in triplicates and the degree of sample separation. Inspection of the WG separated samples (\cref{subfig:solvent_settling}) showed acceptable separation for all solvents except \gls{UHP}. Comparison of the WG to settling illustrated the superiority of the latter in producing a colourless supernatant (\cref{subfig:solvent_WG}). Interestingly, whilst the RGB data collected by C-3NpH showed variability, smoothing from the k-NN model consistently predicted pH {5.2}. This value corresponds to readings from the settling counterparts. When \gls{UHP} was used for settling, the measurement was marginally higher at %pH 5.22 (WG) 
pH {5.3}.

%Figure  illustrates the variation in strip colour between each solvent with water for both methods separating the mixture the most poorly. 
%Comparison of the WG to settling reveals that settling was superior at producing a colourless supernatant. 
%Interestingly once read with C-3NpH, the value returned for all WG samples in each solvent was consistently 5.2, as was the measurement for all solvents when measured after settling for 20-30 minutes. 
%Except for \gls{UHP} which was marginally higher on average across three strips dipped in one sample, where again the comparision between WG and settling was minimal given the resolution of C-3NpH and the predictive model used.(pH 5.22 vs 5.33) . It is therefore proposed that the 'brownness' of the WG samples did not affect the final pH value of the supernatant and could be considered negligible (as raw values did change but once processed with the pH model, returned a value of 5.2).


\begin{figure}[h!]
	\centering
	\begin{subfigure}[b]{\linewidth} 
		\centering
		\includegraphics[height=4cm]{Pictures/C-3NpH/solventchoice_WG.png}
		\caption{Separation achieved by using WG.}
		\label{subfig:solvent_WG}
	\end{subfigure}
	%\vfill
	\begin{subfigure}[b]{\linewidth}
	\centering
		\includegraphics[height=4cm]{Pictures/C-3NpH/solventchoice_settling.png}
		\caption{Separation achieved by settling.}
		\label{subfig:solvent_settling}
	\end{subfigure}
	\captionsetup{justification = centering}
	\caption{Comparison of \SI{2}{M} solvents separated following different approaches.}
	\label{fig:solventchoice}
\end{figure}   	

\subsubsection{NaCl concentration}

It was of interest to determine the effect of salt concentration on the strip colour post-extraction. Hence, three concentrations of NaCl (i.e. \SI{0.5}, \SI{1} and \SI{2}{M}) were investigated; and compared to \gls{UHP}-extracted soil as seen in \cref{fig:solventmolarity}. The samples were bench-top centrifuged to ensure optimal separation; this should distinguish between differences in strip colour from incomplete separation to those that arose from changes intrinsic to NaCl concentration. A negligible difference in the separation achieved at each concentration was observed, all of which C-3NpH reported as pH {5.2}. The industrial standard solvent for soil pH measurements is CaCl$_2$ (\SI{0.01}{M}). Comparative experiments were conducted between NaCl and CaCl$_2$ to validate the choice of using NaCl (see \cref{c3nph:nacl}). 

\begin{figure}[h!]
	\centering
	\includegraphics[height=3.5cm]{Pictures/C-3NpH/solventmolarity.png}
	\captionsetup{justification = centering}
    	\caption{Comparison of the separation ability using different concentrations of NaCl ranging from \SI{0.5}-\SI{2}{M} by a \gls{BC}}
	\label{fig:solventmolarity}
	%\vspace{1em}
\end{figure}

As the WG was investigated as an alternative to settling, experiments were performed to compare the range of centrifuge speed required to achieve efficient separation, using a \gls{BC}. At \SI{6000}{rpm}, NaCl was shown to form a pellet at \SI{15}{s}, but the equivalent with \gls{UHP} required \SI{1}{min} (\cref{subfig:WGBC}). This provided impetus to choose NaCl as the extractant.



\begin{figure}[h!]
	\centering
	\begin{subfigure}[b]{0.45\linewidth} 
		\centering
		\includegraphics[height=3.5cm]{Pictures/C-3NpH/speed.png}
		\captionsetup{justification = centering}
		\caption{For the \gls{BC} at \SI{6000}{rpm}, NaCl solvent forms separates at \SI{15}{s}.}
		\label{subfig:speed}
	\end{subfigure}
	%\vfill
	\begin{subfigure}[b]{0.50\linewidth}
	\centering
		\includegraphics[height=3.5cm]{Pictures/C-3NpH/WGBC.png}
		\captionsetup{justification = centering}
		\caption{For the \gls{BC} at \SI{6000}{rpm}, \gls{UHP} solvent forms separates at \SI{1}{min}.}
		\label{subfig:WGBC}
	\end{subfigure}
	\captionsetup{justification = centering}
	\caption{Separation efficiency of the \gls{BC} at a range of rotational frequency between \SI{2000}-\SI{6000}{rpm} using \SI{0.5}{M} NaCl solvent gave supernatants of the same clarity.}
	\label{fig:WGBC}
\end{figure}   	
 
\subsubsection{Mixing time}

Preliminary investigations of soil-solvent mixing times were performed by manually shaking the mixture for 2 minutes, which was followed by the separation method of choice. Initial results showed minimal change in colour of the pH strips over \SI{24}{hours} of mixing when judged by eye. However, the exact effects of mixing time on pH were inconclusive. This is potentially due to factors involving the quantity of soil sampled, the effect of dilution factor (1:9), and the heterogeneity of soil samples (see \cref{c3nph:mixing}). Literature is varied: from mixing times for pH measurements of \SI{2}{hours} \cite{Mitchell2013}, to early studies reporting minimal mixing time (\SI{5}{min}) followed by overnight standing \cite{Farr1972}. Yet Schmidhalter \cite{Schmidhalter2005} highlighted a number of sampling protocols that could be used for nitrate extraction, where a minimum of \SI{3}{min} manual mixing time is required for moist samples across a range of soil types.

Alternative approaches for soil sampling that are to be explored in the future are listed in \cref{appen:alter}.


%The mixing time of the reported protocol (\SI{2}{min}) is considerably shorter than industrial standards, which state times ranging from \SI{15}{min}-\SI{24}{hours}. Literature is varied, with mixing times for pH measurements of \SI{2}{hours} \cite{Mitchell2013}, with early studies reporting minimal mixing time (\SI{5}{min}) followed by overnight standing \cite{Farr1972}. For this reason, preliminary investigations were performed based on \SI{2}{min} of manual shaking, followed by the separation method of choice. 
%Mixing time experiments were conducted, and initial experiments showed no visual change of pH strip colour over \SI{24}{hours}. However, the exact effects of mixing time on pH were inconclusive for various reasons. Further investigations using a larger mass of soil, a more appropriate dilution factor and homogeneous mixing will be required. It was found that extended mixing times made WG separation more difficult, due to improved homogeneity. Schmidhalter \cite{Schmidhalter2005} highlighted a number of sampling protocols that could be used for nitrate extraction. Namely for the \gls{QNT} soil sampling: a minimum of \SI{3}{min} manual mixing time is required for moist samples across a range of soil types, which is contrary to previous literature reports. \todo{KATIE- reference} 





%------------------------------------------------


%\subsubsection{General Discussions}

%\paragraph{Soil sampling protocol}
%\begin{itemize}[leftmargin=*,label={}]
%\item\textit{\textbf{{Whirligig}}}
%%KATIE%%
%Eppendorf holders
%\item\textit{\textbf{{Optimisation of pH extraction needed}}}
%\end{itemize}

% General Discussion 1: Answer Oliver's Questions - 

% General Discussion 2: Add the potential to correlate nutrients with moisture content to provide a 'holistic' picture of soil health. 
%%Chyi to add to conclusion. 

% General Discussion 3: Rehabilitation of soil based on Nutrient measurements



%protocol was easy-to-follow
%%%keen to understand pH 
	%%% enthusiasm from Head Gardener: target market
%staining of the strip
	% contamination
%water as solvent; aware of long settling time
%%%20-30 min settling time: a bit long
	%%% justification for WG




\chapterimage{Pictures/software/software_chapter.jpg} % chapter heading image for software
\chapter{Science and Technology -- Software}\index{Software}
 \setcounter{page}{44}
\section{General requirements}\index{General requirements}

%Importance of requirements
%How were the requirements gathered

The requirements for the final product had to be gathered to make several important decisions in regards to the software \gls{mvp}.\\

%Talk about user requirements:
%Functional
    %All the data stored together, no individual user accounts
    %Ability to upload the collected data
    %Visualisation of the data on a map
    %Downloading the CSV file with all the measurements
    
\textbf{Functional} requirements\footnote{Functional requirements define what the system should be able to do.}:
\begin{itemize}
    \item ability to upload the field data collected in a \gls{csv} file format;
    \item ability to enter certain parameters later (e.g. organic carbon content or CO$_{2}$ content -- the parameters that were obtained in the laboratory);
    \item visualisation of collected data on a map;
    \item data storage on a remote database;
    \item data download from the remote database as a \gls{csv} file;
    \item password protected website;
    \item no individual user accounts -- all the data stored together, and all the data visualised together.
\end{itemize} \text{} \\

%Non-functional
    %Free for the end-user 
    %User-friendly
    %Non-GIS map solution: the user might or might not be able to use GIS


\textbf{Non-functional} requirements\footnote{Non-functional requirements are the quality attributes.}:
\begin{itemize}
    \item free for the end user;
    \item user-friendly, given a potential user-base from a wide variety of backgrounds;
    \item the map solution has to include an option for users unable to use \gls{gis}.
\end{itemize}

%------------------------------------------------
\section{Application type selection}\index{Application type selection}

Various application types were considered for this project, and the web application was considered to be the optimal solution given the requirements of this project.

%Web application (CHOSEN)​
%Pros: multi-platform, ease of updates (always up to date)​
%Cons: internet connection required, can be slower than native apps​

\paragraph{Web application}\index{Web application}

The main advantage of a web application is it being a multi-platform solution, always up to date. The main disadvantage of a web app is the required reliable internet connection. A web application might be slower than a native app, however, given the specifications of the application to be developed, speed differences would be negligible. Hence, it was decided to develop a web application.

%Program on user's computer​
%Pros: free, does not require internet connection​
%Cons: depends on computer specifications, user required to have Geographic Information System (GIS)​

\paragraph{Local application on user's computer}

The decision was made to avoid writing software which would run locally on the user's computer. Although this solution would be free for the end user and the development team, and would not require internet connection, the disadvantages outweighed the positives. Firstly, data visualisation would be problematic, unless the user had installed a \gls{gis} application. Even assuming the user could install an open-source \gls{gis} application, such as QGIS \cite{qgis}, it is still a complicated application which requires specialist training. Moreover, internet connection would still be required for data to be uploaded to the database. The performance of the program would also rely on user's computer specifications.

%Native phone application​
%Pros: easy front end with available SDKs, compatibility with phone's hardware​
%Cons: triple work – development for Android + iOS + Windows Mobile, for iOS X-Code is required, pass App Store regulations, developer costs, data transfer from the main device to the phone app?​

\paragraph{Native phone application}

An estimated 2.5 billion people in the world own smartphones, with 76\% of the population in developed countries owning at least one.\cite{taylor_silver_2019} However, that percentage drops to only 45\% in emerging economies.

A native phone application would be compatible with the phone's hardware, eliminating the need for a \gls{gps} receiver. The phone camera could be used for scanning a \gls{qr} sticker on the bag with the soil sample collected from the field (instead of the \gls{rfid} tag and reader combination). Another obvious advantage would be the ease of developing the front-end of the application, making it user-friendly and intuitive, achievable with the available \glspl{sdk}. 

Nonetheless, for this project, the disadvantages of this solution outweighed the benefits. Firstly, nowadays all mobile platforms use different native programming language. Due to this, the application would require development for both Android and iOS, to reach the majority of smartphone users. iOS development comes with added complexities: an integrated development environment Xcode is required, which runs exclusively on macOS. Furthermore, it is necessary to join the Apple Developer Program \cite{apple} to distribute the application. Most importantly, data transfer would remain an issue, if the base unit records data on an SD card. Majority of modern smartphones do not support SD cards, which would mean there is no way of easily uploading the data from the base unit to the database. However, it could be a promising avenue to explore in the future if the decision was made to transition from the SD card to wireless data transmission, e.g. Bluetooth.



%------------------------------------------------
\section{Full stack web development}\index{Full stack web development}
%Explain concepts of back end and front end
Web development consists of front-end development (client-side), and back-end development (server-side). 

\textbf{Front-end} is sometimes referred to as ``web design'', and is the part of the website that the user directly interacts with, and the client usually refers to the web browser that the user is using to view the website.

\textbf{Back-end} is what happens ``behind the scenes'', and is responsible for the functionality of the website. It could be thought of as an enabler of the front-end web experience. The back-end of a website is a combination of server, app, and a database.

In this section, various approaches for back-end and front-end development are discussed and the decisions made justified. A high-level details of the code are also provided.

%------------------------------------------------
\subsection{Back-end development}\index{Back-end development}

Available options for back-end development were carefully considered. The back-end was developed using Node.js given the relatively small number of disadvantages it presents and its increasing popularity over PHP.

%Talk about available options:
    %PHP
    %Node.js
    %Other options
%PHP​
%Pros: mature and well-established, designed specifically for web, powerful codebase​
%Cons: only works in the backend (still need JS for client-side), slower than Node.js, becoming less popular than Node.js​

%\textbf{PHP} is an old and well-established programming language, created specifically for web development. It is mature and has a powerful codebase, and is the language of choice for the back-end for such well-known companies as Facebook, Wikipedia, and Slack. However, it comes with some disadvantages -- it works only in the back-end, and JavaScript is still needed for the client-side; it is slower compared to other alternatives, such as Node.js, and in general, its popularity has gone downhill since the emergence of other back-end solutions.

%Node.js​
%Pros: JSON works better than with PHP, better for dynamic applications, better for real-time data, JS used for both client-side and server-side​
%Cons: dependencies, not ideal for heavy algorithms (but we are not using any)​
\textbf{Node.js} is a cross-platform JavaScript run-time environment, which executes JavaScript on the server-side. First of all, it allows the programmer to use JavaScript for both server-side and client-side. Node.js is more efficient in working with \gls{json} format files, compared to PHP. It is well-suited for dynamic applications, as well as for real-time data. The only caveat is that Node.js comes with an issue of dependencies. While a large number of libraries can seem at first as an advantage, potentially simplifying the development, it also comes with the downside of having to ensure that all the libraries are installed and licensed correctly.

%Others? (Python/Django, Microsoft's .NET stack, Google's Go Lang, etc.)​
%Cons: lack of experience on the team​

Other back-end development languages and frameworks have been considered, such as PHP, Python/ Django, Microsoft's .NET stack, Google's Go Lang, etc. However, it was decided against these options due to their disadvantages and/or the lack of experience on the team. 

A flowchart summarising the operation of back-end can be seen in \cref{fig:backend}.

\begin{figure}
    \centering
    \includegraphics[width=13cm,trim={0 1cm 0 1cm},clip]{Pictures/software/backend.pdf}
    \captionsetup{justification = centering}
    \caption{Flowchart explaining how back-end works and enables the front-end experience.}
    \label{fig:backend}
\end{figure}

%------------------------------------------------
\paragraph{Host server}
%Heroku for pricing and ease

For the host server, a \gls{paas} was used. \Gls{paas} is a public cloud service, which has a simple interface allowing developers to easily deploy their applications.

\textbf{Heroku} is a container-based cloud \gls{paas}. It was chosen due to its user-friendliness and attractive pricing -- the development, as well as user testing, were carried out on free tier\footnote{The only noticeable downside of the free tier is the web app going to ``sleep'', which means that after 30 minutes of inactivity it takes slightly longer to load.}. This is sufficient for the first few users. With a growing user-base, Heroku offers a ``hobby'' plan for only \$7 per dyno\footnote{Dyno in Heroku is an instance of a server containing the source code of the app, the dependencies, as well as the pre-loaded environment variables.} per month. The current estimate is that the ``hobby'' plan should be sufficient for the first year.

%------------------------------------------------
\paragraph{Database}

%Talk about available options:
    %NoSQL (MongoDB)
    %SQL (mySQL)
    
Databases come in two varieties: relational, also known as SQL databases (e.g. MySQL), and distributed (non-relational), also known as NoSQL (e.g. MondoDB).

%MySQL (SQL database – relational)​
%Pros: mature and wide user community, uses well-known structured query language (SQL) for defining and manipulating the data, more powerful queries, better for heavy-duty transactional applications, low-maintenance, low-cost, data security​
%Cons: fixed schema, schema changes cause migration procedure – slows down​

Relational databases store the data in tables (called schemas), that consist of rows and columns. MySQL is a mature database management system, which uses a well-known \gls{sql} for defining and manipulating the data. It has a wide user community, which makes it an attractive solution due to the support available. It also has more powerful queries than non-relational databases which makes it better for heavy-duty transactional applications (which would be useful in the future, if the switch is made to real-time data upload). Finally, it is low maintenance, low-cost, and ensures data security. The disadvantages of this solution include migration procedures induced by changes to the schema.
    
%MongoDB (NoSQL database – distributed)​
%Pros: well suited for large dataset, easily composable queries (JSON), good for unstructured data, unstable schema​
%Cons: lack of experience on the team​

MongoDB is a NoSQL database with easily composable queries in JSON (which is superior to SQL for dynamic queries). It is well suited for large datasets and is great for unstructured data. A big advantage of a NoSQL database is an unstable schema, meaning that the user or hardware engineer can add any number of new parameters, and the database would not require any additional maintenance from the web developer. However, it is less straightforward to use than the MySQL database management system, and due to the lack of experience on the team it was decided to choose MySQL. 

%database design tool - MySQL Workbench
MySQL Workbench was chosen as the database design tool, for local database management. JawsDB was chosen as the MySQL database server host -- a Heroku add-on, with sufficient free tier for development and testing, and acceptable prices for the scale-up (\$5 per month for \SI{25}{MB} storage, and \$10 per month for \SI{1}{GB}).

One row of data using the specific schema (defined in \cref{schema_struct}) is approximately \SI{5}{KB}, which means on JawsDB free tier, which offers \SI{5}{MB} storage space, around 1,000 measurements can be stored. Upon scale up to the cheapest paid tier, this number would be increased to 5,000 measurements.

%------------------------------------------------
\paragraph{App functionality}
%Talk through code struct

The server-side code provides app functionality and handles the following functions:
\begin{itemize}
    \item connection to the remote database;
    \item basic authentication, requiring the user to type in a username and a password to access the website;
    \item passing queries to the database, both static and dynamic (for dynamic queries an \texttt{if} loop was implemented to avoid \gls{sql} injections);
    \item using Bootstrap (covered in more detail in the \textit{Front-end development} section);
    \item creation of a JSON object with all the data from the database for data visualisation;
    \item creation of a CSV object with all the data from the database for data download;
    \item upload of the CSV file by the user, parsing it and passing the data to the database.
\end{itemize}

%------------------------------------------------
\subsection{Front-end development}\index{Front-end development}
% Introduce Bootstrap and jQuery
%Bespoke solution (write everything ourselves using HTML and CSS)​
%A lot of work, reinventing the wheel​
%Steep learning curve for producing a production-grade solution​

%Use Bootstrap (CHOSEN)​
%Free and open source​
%Consistent across all the browsers and different devices​
%Easy to integrate and produce beautiful and attractive websites​

Instead of going for a bespoke solution (developing front-end manually using HTML and CSS), Bootstrap was used, which is an open-source CSS framework for front-end web development. In combination with the Themestr.App for customising Bootstrap, this provided an easy to use framework for the development of a user-friendly, attractive website. It also ensured the consistency of the application across all browsers and devices, which is an inarguable advantage given the time constraint associated with this project.

%Use jQuery (CHOSEN)​
%Free and open source​
%Large community and support available​

Furthermore, jQuery was used as a free and open-source JavaScript library, offering a wide variety of functions. This simplified the client-side development significantly thanks to the large community and support available.

A flowchart explaining front-end operation can be seen in \cref{fig:frontend}.

\begin{figure}
    \centering
    \includegraphics[width=13cm,trim={0 4cm 0 4.5cm},clip]{Pictures/software/frontend.pdf}
    \captionsetup{justification = centering}
    \caption{Flowchart explaining front-end operation and its connection with the back-end.}
    \label{fig:frontend}
\end{figure}

%------------------------------------------------
\paragraph{Web design}

The website was designed as a number of HTML pages, with the user being able to choose the page from a menu bar, and the selected page lighting up in a different colour. Every page uses the same template: the SoliCamb logo on the upper left corner, clickable social media icons on the upper right, followed by a menu bar. Under the menu bar, the individual page contents are displayed with explanatory text, and the appropriate functionality (\cref{fig:website_software}).\\

\begin{figure} [H]
    \begin{subfigure}{.48\textwidth}
        \centering
        \includegraphics[width=0.98\linewidth]{Pictures/software/add_data_manually.png}
        \captionsetup{justification = centering}
        \caption{Screenshot of the \latinword{\footnotesize{Add data manually}} page.}
        \label{fig:add_data}
    \end{subfigure}
    \begin{subfigure}{.48\textwidth}
        \centering
        \includegraphics[width=0.98\linewidth]{Pictures/software/download_csv.png}
        \captionsetup{justification = centering}
        \caption{Screenshot of the \latinword{\footnotesize{Download CSV}} page.}
        \label{fig:dl_csv}
    \end{subfigure}
    \captionsetup{justification=centering}
    \caption{Examples of website's graphical user interface (for larger screenshots please refer to the \textit{Appendix X}).}
    \label{fig:website_software}
\end{figure}
\clearpage
%------------------------------------------------
\subsection{Data visualisation}\index{Data visualisation}
%Different options
    %Google maps
    %openMapStreet
For data visualisation, either an \gls{api} or a locally installed GIS software could be used.
    
%​Google Maps API​
%Pros: familiar and intuitive, supports data layers and overlays​
%Cons: must pay after 2500th map load per day, difficult to implement elaborate customizations​

\textbf{Google Maps \gls{api}} has a familiar and intuitive interface, which can undoubtedly contribute to the user-friendliness of the solution. It supports data layers and overlays, which are potentially desirable features in the future for drawing more complex maps. Most importantly, it has a large support community available, including numerous tutorials and clear documentation. While Google Maps API is ``paid'', in practice, the first 2,500 map loads per day are free, after which the developer is charged 0.50 USD for each 1,000$^\text{th}$ load. Given the expected user-base for the initial stages of the project, it is not expected to reach the paid tier. The only disadvantage of Google Maps API is the difficulty of implementation for elaborate customisations. While this might be a problem in the future, for the MVP no advanced customisations were foreseen, which led to the choice of Google Maps API for data visualisation on the web app.

% OpenStreetMap API​
% Pros: completely free, community-driven​
% Cons: less support available, much more time consuming due to the higher complexity of implementation​
\textbf{OpenStreetMap \gls{api}} was also considered due to its benefits of being completely free and community-driven. However, due to the significantly higher complexity of implementation as well as a smaller support community, it was decided to use Google Maps API for the \gls{mvp}, with the possibility of switching in the future.

%GIS​
%Widely used around the world by research institutions, environmental scientists, land-use planners, businesses, etc.​
%Can easily upload a CSV file to create a layer over an existing project​

\textbf{\gls{gis}} is widely used around the world by research institutions, environmental scientists, businesses, etc. Hence, there was an interest in allowing the user to download the CSV file with all the readings from the database to be able to create a layer in GIS with the collected data over an existing project. This process only involves a few steps, which would take no longer than 2 minutes, and would allow professional users to perform customised calculations (such as area average for various parameters) in GIS.


%------------------------------------------------
\paragraph{Google Maps API}\index{Google Maps API}
    %What exactly has been done code wise

The data visualisation HTML file defines the map window size and calls the \texttt{map.js} file, wherein JavaScript the client-side code customises the map in the following way:
\begin{itemize}
    \item the coordinates of Cambridge are defined, and the map is automatically centred at Cambridge with a predefined zoom level;
    \item If the user allows location sharing within the browser, the map is re-centred at the user's location;
    \item marker instances are created with the coordinates obtained from the JSON object from back-end;
    \item a \texttt{for} loop using jQuery is run through all of the data to create content strings for the infowindow for each of the markers, containing information about the rest of the parameters;
    \item the parameter values that are \texttt{null} (have not been collected) are replaced with \texttt{unavailable};
    \item the \texttt{datetime} is reformatted into a user-friendly format;
    \item a \texttt{for} loop is run through all of the markers to add the associated infowindow with the parameter values collected in that location;
    \item marker clusters are formed with the marker instances already defined.
\end{itemize}\text{}\\

    \begin{figure}[H]
    \begin{subfigure}[t]{.48\textwidth}
        \centering
        \includegraphics[width=0.98\linewidth]{Pictures/software/marker_cluster.png}
        \captionsetup{justification=centering}
        \caption{The map centered at Cambridge and a cluster with 3 markers.}
        \label{fig:marker_cluster}
    \end{subfigure}
    \begin{subfigure}[t]{.48\textwidth}
        \centering
        \includegraphics[width=0.98\linewidth]{Pictures/software/marker_infobox.png}
        \captionsetup{justification=centering}
        \caption{Zoomed in map with the individual marker visible, and the infowindow stating the parameters collected in this location.}
        \label{fig:infowindow}
    \end{subfigure}
    \captionsetup{justification=centering}
    \caption{Examples of the \latinword{\footnotesize{Visualise data}} page (for larger screenshots please refer to the \textit{Appendix X}).}
\end{figure}

Upon loading the page, the user sees the map in a smaller ``box'' within the rest of the website, but also has an option of opening it full-screen. Marker clusters are displayed, stating the number of markers within the cluster (\cref{fig:marker_cluster}). Upon zooming in, the clusters break up into individual markers. Upon clicking on a cluster, it zooms in automatically to fit the furthest away markers at the edges of the displayed map. Upon clicking on a marker, an infowindow opens, listing the parameters and the values collected in that location (\cref{fig:infowindow}). Multiple infowindows can be opened at the same time.

This map allows the users without any previous training in mapping software to easily visualise collected data and to observe the data collected by other people.

%------------------------------------------------
\section{User experience test}\index{User experience test}
%Quotes from people - consider splitting into positive and negative feedback
A user experience test was performed, with volunteers recruited from outside the SoliCamb team, from a wide variety of backgrounds. While the general comment was that the website is user-friendly and intuitive, there were a few comments for improvements. These are further discussed in \cref{soft_future}. For full quotes from the interview, please refer to \cref{quotes}.

%----------------------------------------------------------------------------------------
%	CHAPTER 3 - Citizen engagement and outreach
%----------------------------------------------------------------------------------------

\chapterimage{Pictures/chapter_heads/solicamb_outreach.jpg} % Chapter heading image

\chapter{Citizen Engagement and Outreach}

\section{Citizen engagement}\index{Citizen engagement}

    \subsection{Importance of citizen engagement} \label{Citizen engagement}
    %Author:  - Sarah
    Citizen engagement is an essential component of any research project. It promotes awareness, education and interaction with members of the public, formation professional collaborations, and provides a means to gain funding. Public engagement is, by definition, a two-way process \cite{Public_engagement_definition}. Firstly, it serves to benefit the public by involving and educating them in a particular area of scientific research. Secondly, it serves to benefit the researcher by gaining insight into a specific user need, or as a means to collect larger, more diverse and representative data sets. In accordance with this definition, the team outlined key objectives prior to conducting any outreach events. \Cref{fig:vendiagram} was used to ensure that our project's overall public engagement would be mutually beneficial. There were three main events planned in order to fulfil the above objectives: Immerse Cambridge Summer School, MakeSpace Make-a-thon, and Agri-Tech East Hack-a-thon. These are discussed further in the following sections. 
    
    \begin{figure}
        \centering
        \includegraphics[width=0.8\linewidth]{Pictures/Outreach/vendiagram.png}
        \caption{Key objectives for public engagement events. These were proposed in order to obtain a balanced approach to public engagement and ensure mutual benefit for SoliCamb and the end user.}
        \label{fig:vendiagram}
    \end{figure}
    
       % \begin{table}[h!]
        %\resizebox{\textwidth}{!}{%
        %\begin{tabular}{@{}lllll@{}}
        %\toprule
        %Objective & SoliCamb’s gain & Audience gain & Service &  \\ \midrule
        %Promote STEM in teenage populations &  & X & X &  \\
        %Educate the public &  & X & X &  \\
        %\begin{tabular}[c]{@{}l@{}}Raise awareness of\\ the global soil health problem\end{tabular} & X & X & X &  \\
        %Provide an open-source platform & X & X &  &  \\
        %Understand user need (farmers, gardeners, academics) & X & X &  &  \\
        %Generate data for device and model validity & X &  &  &  \\
        %Generate user feedback on device design and usability & X & X &  &  \\ \bottomrule
        %\end{tabular}%
        %}
        %\caption{\small Key objectives to be met with public engagement events. Meeting these targets %will ensure SoliCamb's project's engagement is mutually beneficial for parties involved. }
        %\label{tab:public_engagement}
        %\end{table} 

 %\begin{table}[h!]
        %\resizebox{\textwidth}{!}{%
  %      \centering
   %     \begin{tabular}{@{}lcccc@{}}
    %    \toprule
     %   Objective & SoliCamb & Audience & Service &  \\ \midrule
     %   Promote STEM  &  & $\surd$ & $\surd$ &  \\
      %  Education &  & $\surd$ & $\surd$ &  \\
      %  \begin{tabular}[c]{@{}l@{}}Raise awareness \end{tabular} & $\surd$ & $\surd$ & $\surd$ &  \\
      %  Provide open-source platform & $\surd$ & $\surd$ &  &  \\
       % Understand users' need & $\surd$ & $\surd$ &  &  \\
    %    Generate data and model validity & $\surd$ &  &  &  \\
     %   Obtain feedback & $\surd$ & $\surd$ &  &  \\ \bottomrule
      %  \end{tabular}%
        
       % \caption{Key objectives for public engagement events. These were designed to provide a balanced approach to public engagement and ensure mutual benefit for SoliCamb and the end user.}
        %\label{tab:public_engagement}
        %\end{table} 
    
   
    
   %citizen science - 
   In contrast to public engagement, which has a symbiotic element, citizen science can be carried out without any obvious participant gain, other than sheer interest and enjoyment. Although there is yet to be an international consensus of what encompasses \enquote{citizen science} \cite{heigl2019opinion}, the Oxford English dictionary definition is \enquote{scientific work undertaken by members of the general public, often in collaboration with or under the direction of professional scientists and scientific institutions}. Participation from citizen scientists is hugely beneficial for the purpose of developing the team's project. Firstly, for understanding if the device is user-friendly and fit for in-field operation by a non-expert. Secondly, as the technology is open-source, the team hoped to benefit from interactions and collaborations with a network of global citizen scientists, who are interested in this field of work. For the purpose of this project, SoliCamb defined target citizen scientists as non-experts, but someone who is interested in either the design and functionality, or using and testing the device. Whilst some of SoliCamb's public engagement events did have elements of citizen science, separate events solely focused on citizen science were conducted at Anglesey Abbey and Radwell Grange Farm (see \cref{Sample Sites}).
   
 
    \subsection{Professional collaborators} \label{Collaborators}
    % Elena: List the main sets/categories of groups/people we reach out to and those that replied and we successfully collaborated with (with contact list in appendix) 
    %Elena can I write this with you if you wanted to take the lead and let me know what to fillin :)  (Katie) ,
    Throughout the duration of the Team Challenge, SoliCamb has contacted multiple academics and other experts in the field of soil science, farming, geology, sensor technologies, as well as data management and visualisation. The main goal of this was to understand the field of soil health and soil health monitoring better, and to estimate what is feasible to achieve within the scope of the project, in regards to budget and time. The main questions were targeted at the end user and focused on technology currently on the market. What is being used at the moment? Where are drawbacks with these methods? What does not exist yet but would be of use for the end user? Which technologies or tests are too expensive or too complicated to use? Which field tests are not quite accurate enough as of yet? 
    Both input and feedback from our contacts have affected the decision making process, and have influenced the direction of our work. %A full list of our contacts can be found at \todo{insert Appendix link}
    
    The \textbf{\gls{UCPP}} were crucial in setting the problem, and gave us first insights into what is needed of a sensor technology. They required a device, which would improve efficiency of field work and allow quantification of the impact and success of efforts to restore soil health, in developing countries. 
    
   \textbf{Miguel Hernandez (GS-fresh)} has provided knowledge about state-of-the-art of sustainable farming methods, soil treatment, soil sampling, and soil analysis within the UK and Europe.
    
    The choice of sensible measurement parameters, which would result in a holistic overview of soil health and seemed feasible to develop in the scope of the project, was supported by \textbf{Philippa Arnold (National Farmers Union)} and \textbf{Paul Flynn (Soil Association)}. Beyond this, Paul Flynn has validated the design and setup of the sensor platform when milestones of the project were reached and expressed interested in further collaborations. 
    
    \textbf{Zimmer and Peacock} have played a major role in our decision to use the colour sensor not only as an educational tool, but as an actual peripheral sensor of our sensor platform to monitor nutrients and pH. Martin Peacock has assured us that electro-chemical methods (industry standard) are not yet robust enough for our purposes, and that indicator paper strips are an appropriate solution to the problem given our time and budget.
    
    \textbf{Dr. Sam Stanier} from the Department of Engineering at the University of Cambridge has been our external expert, who was consulted for questions about the feasibility of the moisture retention probe and its use.
    
    With \gls{SOM} being a primary soil health parameter, but feasibility studies showing no promise in creating a working \gls{SOC} device within the scope of the project, a collaboration with \textbf{PhotosynQ} was initiated. PhotosynQ is an open-source project focusing on plant health originating in the USA. They have developed a tool to monitor the active pool (5-20\%) of \gls{TOC}, which is a good indicator of biological activity, nutrient availability and can be used to predict crop yield. 
    PhotosynQ shared their protocols and \gls{BOM} with us. Sean Reed and Dr. Dan TerAvest were very supportive and willing to share information about the sensor setup along the way. SoliCamb conducted preliminary experiments to assess the feasibility of combining this sensor within our toolkit, the details of which are discussed in \cref{CO2sensor}.
    
    Through a program run at Earthwatch and the University of Oxford, which focuses on fighting soil erosion, we have started talking to \textbf{Dr. Martha Crockatt (Earthwatch)} and \textbf{Dr. Anna Krzywoszynska (University of Sheffield)}. Beyond the soil erosion project, both are currently working on a new project commencing October 2019. The aim is to create a citizen engagement/outreach program to raise awareness for soil health, and getting citizens and technology more involved in the subject, as well as bringing them closer together. Both Dr. Crockatt and Dr. Krzywoszynska focus on citizen science in agriculture and offered SoliCamb the opportunity to take on the technology aspect of the program by introducing people to our sensing platform. At the moment, we are in the process of putting together a program, as well as a detailed prescription on how to use our sensors for the event.
    
    Dr. Krzywoszynska has brought to our attention \textbf{SectorMentor} as a potential collaborator. SectorMentor works on soil health monitoring, and has created an app to easily store and manage data of several soil health parameters. They seem interested in a collaboration with SoliCamb, especially for their work in vineyards. As of now, this is in very early stages.
    
    We had the pleasure to test soil at \textbf{Anglesey Abbey}, property of the \textbf{National Trust}. Head gardener, David Jordan offered SoliCamb the opportunity for future collaboration, and access to the gardens for soil testing,  volunteering to test our future prototypes with colleagues. SoliCamb was also offered the chance to extend this collaboration to other NAtional Trust properties, based on the results of this initial collaboration.
    
    \textbf{Agri-Tech East} are actively supporting us with finding future collaborators and funding. They have invited us to several conferences and events that revolve around sensor technologies, soil health monitoring, and farming, as well as entrepreneurship.
    
    During our participation at the \textbf{Agri-Tech East Hack-a-thon}, we modified our idea to UK-based needs, and thus to an autonomous sensing platform that would be deployed in the field (see \cref{smrpcontinuousmonitoring}). The \textbf{BASF} representative approached SoliCamb and suggested that we get in contact once the project has developed further, to pitch the idea to BASF in the UK or BASF in Germany. 
    
    Depending on future directions, \textbf{Let's Grow}, a Dutch company focused on data visualisation in the agriculture sector, has expressed interest in working with us to improve our online application and user interface.

    
    \subsection{Soil sample sites}\label{Sample Sites} 
    %Author:  - Katie
    As mentioned above, field testing was conducted in Anglesey Abbey and Radwell Grange Farm. These experiments were used to evaluate the user-friendliness of our device by non-experts and to generate in-field data.
    
    
   At Radwell Grange Farm, valuable information on robustness of C-3NpH was obtained by SoliCamb members. In this instance, field-testing was conducted by members of the SoliCamb team only. It was noted that waiting up to 30 minutes for soil to settle was cumbersome, and there was high variability in separation. For some samples, siphoning off the supernatant was challenging. This is something that requires further work to optimise the user experience, especially if multiple sites are to be sampled to map nitrate concentration across a field. This would not be such a hindrance if all soil samples across the field were collected, mixed, homogenised and then measured, where one 30 minute interval for soil to settling would be acceptable.
   
  At Anglesey Abbey, we had the pleasure of working with the head gardener,  as the citizen scientist who had previous experience of horticultural training. They were  especially interested in pH testing and being able to conduct the tests themselves on site. This contrasted to previous conversations with an arable farmer, who showed interest in obtaining the results, but not in performing the tests. Current field testing protocols were easy-to-follow, however, again settling time (observed as \SI{20}-\SI{30}{\minute} on the day) was cumbersome. Therefore, developments with the WG are proposed to be beneficial to the end user by reducing sampling time. Concerns raised by the citizen scientist at Anglesey Abbey pertained to cross-contamination between strips inserted, which requires further experimental investigation (alongside protocol optimisation) despite not being an initial concern during laboratory experiments. On a number of occasions, the colour pad of strips fell off during measurement. This needs to be considered for future design iterations. 
  
   The moisture retention probe was also taken to the above sites (as detailed in \cref{sec:moistureprobe_field}). 
   
   In addition to in-field testing, laboratory validation was conducted on soil from the Cambridgeshire area, including Grange Farm (Lolworth), Madingley Mulch Outdoor Supplies, King's College Allotments, Wolfson College Grounds,  a home garden in Cherry Hinton, and a public walkway in West Cambridge. 
   
   \section{Traditional methods for advertisement}
    Traditional 'print' methods are also a valuable tool for advertisement and generating interest at local events. For departmental and university-wide engagement, SoliCamb posters were made and hung up in the department of \gls{CEB} as well as several other departments. Moreover, during the latter half of the project, an introductory advertisement video for SoliCamb was broadcast on departmental televisions. Team T-shirts, business cards, stickers and professional conference banners were also designed to promote the brand at professional events (\cref{fig:traditional_advertisement}).  Designing several products, targeted at different academic and demographic populations, aimed to increase SoliCamb's marketing reach.
    
    \begin{figure}[ht]
    	\centering
    	\includegraphics[width=0.9\linewidth]{Pictures/Outreach/traditional_advertisement.png}
    	\captionsetup{justification = centering}
    	\caption{ Traditional methods used for advertising SoliCamb. These included posters, business cards, T-shirts, stickers and conference merchandise }
    	\label{fig:traditional_advertisement}
     \end{figure}

  \section{Outreach}
    Online platforms play a crucial role in shaping how information and influence spread among citizens. Social media tools are deemed as very important triggers of change for teaching and learning practices \cite{manca2016facebook}. Such platforms aim to radically transform the academic environment to be more social, open and collaboration-oriented, enabling scientists to communicate their research quickly and efficiently to professionals, as well as to non-expert public members. Although the exact definition of social media is constantly in a state of change, social network sites, blogs and multimedia platforms are typically part of the today's social media landscape \cite{tess2013role}. Currently, Facebook, Twitter and Instagram are the most popular and widely-used social media platforms with 1.56 billion users around the world active on Facebook every day as of March 2019 \cite{investopedia}. Therefore, for both advertising and promoting SoliCamb and affiliated events, the team made use of these three platforms targeting different demographics. In particular, Facebook is more popular amongst middle-aged adults living in rural, suburban, and urban areas, at every income level and educational background \cite{investopedia}. By contrast, internet users living in urban areas, within the academia environment, are more likely to use Twitter as their primary source of news. Younger internet users (18-29 years old), however, make high use of Instagram as a platform for photo/video sharing purposes.
     
        
        \subsection{Social media platforms}\label{Social media}
        % Chiara: importance of this platform for advertising/engaging.  Give stats on this over time course of project, what posts received most likes, how many signed up to events from these platforms etc. 
       A trio of Facebook, Instagram and Twitter accounts have been simultaneously set up on behalf of SoliCamb, in order to engage with the public. As aforementioned, most of the posts published on Facebook and Instagram are aimed to reach the teenage public, keeping them abreast with ongoing work and progress within SoliCamb. To this end, snapshots of the team working on the development of the sensor platform, doing in-field or laboratory testing and attending  popular outreach events (e.g. 105 Radio Cambridge, BBC Radio and That's TV Cambridge), have been shared throughout the 10-week project timeline (\cref {fig:OutreachRadio}). 
       
       
          \begin{figure}[ht]
    	\centering
    	\includegraphics[width=0.55\linewidth]{Pictures/Outreach/OutreachRadio.jpg}
    	\captionsetup{justification = centering}
    	\caption{ Some of SoliCamb's outreach events held between July and August 2019. Upper left shows the team's feature in the Cambridge Independent. Bottom left and upper right show the feature with Cambridge 105 Radio and BBC Radio, respectively. Middle right shows a workshop with Cambridge TV and bottom right shows the team advertising with SoliCamb branded T-shirts}
    	\label{fig:OutreachRadio}
     \end{figure}
       
       
       By contrast, the Twitter page was very useful in reaching a wider and more mature audience.  SoliCamb's initiatives and events were shared within the academic network, gathering interest from soil scientists, engineering enthusiasts and environmental experts. Finally, social media was employed as a tool to share several articles/issues about soil science (e.g. \enquote{\textit{Not another climate horror story}}, The Guardian (2019) \cite{guardian}) 
       
SoliCamb's Facebook, Instagram and Twitter accounts have 185, 75 and 65 followers, respectively, to date.  With regards to the outcome of the social media activities, some insightful data has been collated about the most liked and popular posts as shown, for example, in \cref{fig:FbFollowers}, which has allowed us to better understand the impact and reach of both our online presence as well as our other outreach activities. These data and other insights are discussed further below. 
      
          \begin{figure}[ht]
    	\centering
    	\includegraphics[width=0.8\linewidth]{Pictures/Outreach/Picture3.jpg}
    	\captionsetup{justification = centering}
    	\caption{An overview of SoliCamb's Facebook page followers as of today. The page was originally set up on the 23 June 2019 and ever since, a steadily increasing trend can be seen.}
    	\label{fig:FbFollowers}
     \end{figure}
     
\paragraph{Facebook}

Based on the data from 15 July-15 August, the number of people reached (i.e. the number of people who were registered as having seen any posts from SoliCamb's page on their screen) was 5490 and the number of posts engaged with (i.e. the number of times that people have engaged with our posts through likes or comments) was 699. With respect to the 17 posts published over the weeks, 10 were photos, 6 links and 1 video. The post that raised the most interest among the followers was published on the 6 August, showing footage of our interview feature with That's TV Cambridge, which reached over two and a half thousand people. The second most successful post was related to the Cambridge Independent article release, which reached 1.7 thousand people. In regards to the demographics of those who followed the SoliCamb's Facebook page, the data was grouped by age and gender. The information reported in \cref{fig:FbAge} shows that 46\% and 54\% were women and men, respectively, who were in the 25-34 age range. This demonstrated that the SoliCamb's advertising approach was capable of targeting both genders equally, along with expected demographic statistics, as reported by Facebook \cite{investopedia}. As discussed previously, younger age-groups were targeted through different social media platforms and older age groups through radio and newspaper platforms.

 \begin{figure}[h!]
    	\centering
    	\includegraphics[width=0.95\linewidth]{Pictures/Outreach/FbAgeRange.jpg}
    	\captionsetup{justification = centering}
    	\caption{Facebook SoliCamb followers grouped by age and gender. Overall, both women and men have been reached equally, within the age range 26-34 years old.}
    	\label{fig:FbAge}
     \end{figure}

%\begin{figure}[h!]
    	%\centering
    %	\includegraphics[width=0.7\linewidth]{Pictures/Outreach/MostLikedPict.jpg}
    %	\captionsetup{justification = centering}
    %	\caption{Overview of Facebook posts published with information about the type of post, reach and engagement.}
    %	\label{fig:MostLikedPict}
    % \end{figure}
 
\paragraph{Instagram}
 
The Instagram and Facebook platforms were ran concomitantly by sharing the same pictures and videos in order to reach as many people as possible. To advertise SoliCamb's name on Instagram, SoliCamb  `followed' pages relevant to the project on a daily basis, as well as engaging new `followers' in turn. Post shared on Instagram employed relevant tags (e.g. \@solicamb, \@EPSRC, etc) and hashtags (e.g. \#soilhealth, \#citizenscience, \#southafrica) to extend the target audience to those involved in soil science, the environment and academia.

\begin{figure}[h!]
    	\centering
    	\includegraphics[width=0.55\linewidth]{Pictures/Outreach/Instagram.jpg}
    	\captionsetup{justification = centering}
    	\caption{An Overview of SoliCamb's Instagram frontpage.}
    	  \end{figure}

\paragraph{Twitter} 
The SoliCamb's Twitter account was created as an additional means for broadening the target audience, specifically targeted at experts and enthusiasts from the academia/industry environment. Based on the data from August, the SoliCamb's twitter page raised 17.1 thousand impressions (about 608 impressions per day), \footnote{Impressions: the number of times a tweet is shown on one's timeline} 397 profile visits and 20 `retweets' \footnote{Retweet: followers who advertised SoliCamb by sharing posts from the team page on theirs}. As reported in \cref{fig:Twitter}, the tweet that raised the highest engagement was in regards to SoliCamb's article in the Cambridge Independent, showing again the success of this media release. 

  \begin{figure}[ht]
    	\centering
    	\includegraphics[width=0.9\linewidth]{Pictures/Outreach/Twitter.jpg}
    	\captionsetup{justification = centering}
    	\caption{Twitter statistics showing the main activities over the past 28 days period. The post relating to the release of the Cambridge Independent article raised the most impressions and `retweets'.}
    	\label{fig:Twitter}
     \end{figure}
  
    \subsection{Website and newsletter}
    %Author: Sarah
        The website was created using Wix Website editor, and acted as a platform to document SoliCamb's progress, give interested parties a point of contact, and to raise awareness of our challenge and the functionality of our device. The website has 5 pages which encompass SoliCamb's mission statement, team information, device design, a blog, and a contacts page (\cref{fig:website}). 
        
        \begin{figure}[h!]
	\centering
	\begin{subfigure}[b]{0.45\linewidth} 
		\centering
		\includegraphics[height=5cm]{Pictures/Outreach/website_a.png}
		\caption{}
		\label{subfig:website_a}
	\end{subfigure}
	%\vfill
	\begin{subfigure}[b]{0.45\linewidth}
	\centering
		\includegraphics[height=5cm]{Pictures/Outreach/website_b.png}
		\caption{}
		\label{subfig:website_b}
	\end{subfigure}
	\captionsetup{justification = centering}
	\caption{SoliCamb's (a) website homepage and (b) blog page, where highlighted in red are the interactive functions for user engagement.}
	\label{fig:website}
\end{figure}   	

        
        %\begin{figure}[ht]
    %\centering
    %	\includegraphics[width=\linewidth]{Pictures/Outreach/website.png}
    %	\captionsetup{justification = centering}
    %	\caption{SoliCamb's website homepage (A) and blog page (B). Highlighted in red are the interactive functions which can be used to engage with the user. }
    %	\label{fig:website}
    %\end{figure}
        
     % Number or users and session time
Starting from the release date of the website, up until the day of SoliCamb's final presentation (24 June - 15 August), the website had over 190 users and over 960 page views. The average `session' or viewing time on the site was defined as \enquote{session starting when a user views a page on the website and ends either when they leave or after 30 minutes of inactivity}. Our average user session time was $2.5$ minutes. This is comparable to the industrial benchmark, which states an average session duration of $2-3$ minutes is `good' \cite{user_session_stats, user_session_stats1} and shows that the  website is successful in engaging the user.
        
    % Bounce rates
    Bounce rate is another parameter often used as a measure of user interest and retention. A bounce is defined as \enquote{a single-page session} and bounce rate as  \enquote{the percentage of all sessions in which users viewed only a single page before exiting the site}. In SoliCamb's case, a low bounce is desirable, as the website contains more than one page. Interestingly, the bounce rate for \texttt{Our Missions} and \texttt{About Us} pages were the highest at 100\% and 75\% respectively, whereas the blog page had the lowest bounce rate of less than 30\%. The blog page contains links and interactive tool-boxes for the user to engage with, including options to `like' and leave comments on posts, as well as sign-ups for the weekly newsletter (\cref{subfig:website_b}). In contrast, the other two pages are largely descriptive, which highlight the importance of page design in retaining user interest. In the future, pages with high bounce rates should be redesigned to include more interactive elements, such as videos and links to our other pages.
    
    % User reach - geographical location
    Obtaining information relating to geographical location of SoliCamb's users can be used to assess the success of the project's reach. Although 76\% of users were located locally in the UK, it was interesting to see that the website had a global reach of 10 countries (\cref{fig:user_location}), including those which we have reached out to, or formed collaborations with. Continuing to gain global recognition will help to further promote and develop SoliCamb's project. 
        
    \begin{figure}[ht]
    	\centering
    	\includegraphics[width=0.8\linewidth]{Pictures/Outreach/user_location.png}
    	\captionsetup{justification = centering}
    	\caption{Geographical location of website users.}
    	\label{fig:user_location}
     \end{figure}
        
    % User site acquisition 
    Analysing user acquisition is a useful way to understand where users originate and how successful advertising/marketing schemes are. From  \cref{fig:user_traffic}, it is seen that the majority of users originate from direct and organic sources, with 75\% typing SoliCamb's web-address directly into the URL search bar or via a search engine respectively. Given that the majority of users are locating the website via a direct web-search, demonstrates that the team is advertising the SoliCamb brand well, and not relying too heavily on one media platform to generate interest. Users which were referred from another site came mainly from the Sensor CDT webpage, and interestingly, BlueBear Systems Research Ltd website. BlueBear is a company based in the field of avionics and data management and they featured SoliCamb in their online newsletter after a team visited. This demonstrates the power of forming relationships with experts,in order to reach larger target audiences. Out of the website's social media referrals, 75\% of users came from the Facebook page, suggesting Facebook may be more successful than other social media platforms in reaching a subset of SoliCamb's audience. 
       
    
        \begin{figure}[ht]
    	\centering
    	\includegraphics[width=0.8\linewidth]{Pictures/Outreach/user_traffic.png}
    	\captionsetup{justification = centering}
    	\caption{Source of website user acquisition.}
    	\label{fig:user_traffic}
     \end{figure}
     
     % Newsletter
    The newsletter was initially set up as a one page PDF document that was sent out via email to interested parties and collaborators. The purpose of the newsletter was to track SoliCamb's progress, to keep the audience up to date and to promote public engagement. Mid-way through the project, a blog page was added to the website where all newsletters could be accessed and updated. The blog page  further promoted user engagement by utilising interactive tools such as `like' and `comment' buttons. After the blog page went live, there was a surge in website views and session time (see \cref{fig:media_peak} in \cref{Social media}). The addition of the blog page also made it easier for new users to sign up to the newsletter emailing list which, to date, has 75 sign-ups. 

    \subsection{Media Appearances}
    
    \subsubsection{Newspaper}
        % Chiara/katie: 
        Designed to coincide with other press releases including radio and TV, an article was featured in The Cambridge Independent Newspaper on 31st July 2019. The article was authored by Paul Brackley, Editorial Director, after an interview with two members of the SoliCamb team. The piece featured photographs taken by Keith Heppell and covered an introduction to the importance of soil health, the motivations behind SoliCamb as well as an overview of the sensor hardware. 
        The two-page article highlighted the multiple components of our device, which at the time were being developed independently.  %In the first instance, would have been used by local farmers in South Africa in order to help them out in the assessment of soil quality according to the UCPP programme. As aforementioned, such national plan aims to address the severe land degradation and invasive alien plant infestation in the Umzimvubu catchment. Furthermore, such device could be used by anyone with an interest in soil health. 
        One of the key motivations for organising this article was to provide and advertise that SoliCamb had created a platform that was accessible and open to citizen scientists.
        
        In order to fully engage with the end user it was necessary to first generate interest through readily available channels, and second it was crucial to emphasise the value of our device to local enthusiasts. The press campaign sought to make it clear that this project, and any sensors made, were not reserved only for experts in the field of soil health but focused on making soil health measurement open to anyone. Awareness of the importance of soil health is growing, especially in relation to meeting the Sustainable Development Goals \cite{Keesstra2016} and one of the desired outcomes of SoliCamb was to emphasise that this is a local issue as well as one found in South Africa. Therefore it was crucial that we engaged with local communities through platforms such as newspapers to not only highlight our sensors but iterate the attitude towards preserving our soil in Cambridge. 
        It was important to organise the release of the article at a time when it would have the highest impact. For this reason it was decided that the article should precede radio or TV coverage as an additional form of advertisement guiding local readers to these other platforms.
        
   
        
 \subsubsection{Radio}
         On 31st July 2019, two members of the team went on air at BBC Radio Cambridgeshire for a live interview, discussing soil health and the motivation behind SoliCamb. The aim of this broadcast was threefold: firstly, to raise awareness on the importance of soil health and the impact of degrading soil on the environment and living ecosystems; secondly, the feature sought to invite farmers, allotment holders etc to test their soil with our sensor allowing us to validate our prototype and generate data for analysis. It was hoped that the broadcast would engage user interest and promote opportunities to attend upcoming SoliCamb outreach events (e.g.the STEM festival in Peterborough), as well as aiming to set up external collaborations scientific partners. On the 1st August 2019 Neil Whiteside hosted a live interview with two members of the SoliCamb team on Cambridge 105 at 10:30 am. Feedback from Mr. Whiteside showed that the number of listeners was reported as incredibly high demonstrating the effectiveness of SoliCamb's outreach. Expert advice from Mr. Whiteside was key to understanding how best to use the power of these public platforms to generate enthusiasm for our project, due to this the team coordinated all TV, radio and newspaper to release within one week of each other as well as coinciding with the upcoming Make-a-thon at Cambridge Makespace. \\
 \subsubsection{Television}
 
         That's TV Cambridge visited the department on 31st July 2019 and recorded three members of the team in both B-Roll footage and interviews. The segment featured on their show that evening and has since been posted across our social media channels. The footage gathered incorporated the modular device as well as SoliCamb's opinion on the importance of soil health and our incentive to get involved at a local level.  Altogether, the TV and radio releases were a good opportunity to discuss the projects main features and future plans.\par 
         
In contrast to pre-scripted articles that could be drafted, radio took the form of live  information sharing. Prior to appearing on radio and TV, the release of information about our project had been relatively self directed. For example, it was the SoliCamb outreach team who drafted pieces detailing information presumed to be relevant for a wider audience but formed entirely from an internal perspective. Only when questioned by external parties (e.g. radio hosts) did it become apparent that those outside the project were perhaps more interested in a different perspective, to what had already been pitched. These insights will be valuable moving forward as the team engage more with local communities and are expected to improve our communication strategy with diverse audiences.
         
        % media peaks 
        Following the aforementioned media releases on 31st July and 1st August, it was promising to see a surge in online engagement on social media pages and website (\cref{fig:media_peak}). Again, following a re-post of said events, spikes in online engagement were also noted. Observing peaks in online engagement, after major outreach events, demonstrates the value and success of advertisement in boosting overall reach. This is an avenue which should be pursued further in the future. 
        
        \begin{figure}[ht]
    	\centering
    	\includegraphics[width=0.9\linewidth]{Pictures/Outreach/media_peak.png}
    	\captionsetup{justification = centering}
    	\caption{ User engagement on website and Facebook platforms over the duration of our project. After major outreach events and media releases, SoliCamb's user engagement show a surge in activity.}
    	\label{fig:media_peak}
     \end{figure}
    
    
    \section{Public Engagement Events}
    % brief overview of the types/categories of events run.- Katie. 
    With a key component of the Team Challenge being citizen science and community engagement, the outreach and public engagement work focused on sustaining a bi-directional relationship.  As each event required detailed plans, considerable effort was focused on understanding the main objectives and desired outcomes, ensuring that the event was mutually beneficial.
    
    Firstly, C-3NpH was intended as an educational tool. This would allow SoliCamb to collaborate with Immerse Cambridge Summer School, working with teenagers (16-18 years old) as citizen scientists who engaged in learning while generating data to validate the pH models between multiple C-3NpH sensors. The participants learnt basic sensor design and electronics, which was a topic not previously studied for most of them. The summer school work was tailored to a specific audience and was pitched to individuals with no experience of engineering or sensor technology. 
    In contrast, the second public engagement piece was the mini Make-a-thon run at Cambridge Makespace. The audience here was more in line with what SoliCamb envisaged as their citizen scientist end user. The third event and again a different category was the Agritech East Hack-a-thon. This was intended not only to showcase SoliCamb's work but also to network and engage with local industry, with a focus on generating interest in setting up potential collaborations. The approach taken by the team highlights the well-rounded and diverse nature of public engagement, the intricate aims, structures and outcomes, of which are detailed in the following sections.
    
\subsection{Cambridge Immerse: Summer school workshop }

        The first opportunity to interact with potential users and carry out usability studies of C-3NpH was via summer schools. We collaborated with Immerse Education,  which runs a series of summer schools including classes and activities related to the field of engineering. The students were aged between 16-18 and were from various backgrounds. Although the students had no formal engineering education, they were generally in the process of applying to \gls{STEM} courses at university.
        The workshop was structured so that the students attended a morning session on the principles of electronics, programming and sensing. The session culminated in a practical session where the students built their own Arduino-based fire alarm, which demonstrated the principles of sensor input, data processing and actuator output.
        The afternoon session introduced the principles underpinning the C-3NpH sensor, and allowed students to build and test their own system using components produced by SoliCamb. The session was mutually beneficial as it allowed us to carry out basic user interaction and ease of use testing for this design and prototype. (For further details, see \cref{para:immerse}.)
        The workshops were very well received by the student and Cambridge Immerse Education, who invited SoliCamb to run a second workshop after the initial agreement, which had been for only one day. %and the SoliCamb team who all benefited from the days that were run.
        
    %%Added by Chyi: Feel free to omit :) %%%
   \begin{figure}[ht]
	\centering
	\begin{subfigure}[b]{0.45\linewidth} 
		\centering
		\includegraphics[width=\linewidth]{Pictures/Outreach/summerschool1.jpg}
		%\captionsetup{justification = centering}
		%\caption{}
%		%\label{subfig:summer_assem}
	\end{subfigure}
	\begin{subfigure}[b]{0.45\linewidth}
		\includegraphics[width=\linewidth]{Pictures/Outreach/summerschool2.jpg}
%		\captionsetup{justification = centering}
%		\caption{}
%		\label{subfig:summer_C3NpH}
	\end{subfigure}
%	\captionsetup{justification = centering}
	\caption{SoliCamb at the Cambridge Immerse Summer School}
	\label{fig:immerse}
\end{figure}  

\subsection{Cambridge Makespace Make-a-thon}

%\todo{jh2109 Explain the main achievements made there, how the attendees responded to our work}
% Chiara?: Explain the purpose of this event and explain the schedule of the day. State how many people attended, what was observed from this event and what the key outcomes were. 
The team held a Make-a-thon on 3 August, from 11~am to 4~pm, at the Cambridge Makespace venue, after arranging this with Ward Hills (Director of Makespace). %jh2109 removed, he doesn't play a role in the rest of the text, does he?
%(see \Cref{MSpace}). 
Overall, 23 people signed up for the one-day event, out of which 5 were members of SoliCamb. The event was advertised via the major outreach channels (radio and newspaper), in parallel to the other social media platforms. Eventbrite and Meetup links were set up for registration. By running this public engagement event, the SoliCamb team engaged %the public (i.e.
with attendees from different backgrounds, such as PhD students, electrical engineers, software experts, and science enthusiasts.   %\todo{KG- chiara where were the experts from you need to specify here what their field was I think} and enthusiasts from the \enquote{maker} movement. %in collaborating with SoliCamb. 
This event was very rewarding and represented a platform for open discussion and exchange of ideas. By joining forces with this community the foundations of a new network and knowledge base have been laid. This means that in the future, should SoliCamb continue with hardware development, there will be a wider and more diverse group of potential collaborators with whom relationships have already been formed. 

%In addition to soliciting general feedback and suggestions throughout the event, 
It was interesting to hear general opinions, and have this user group critique SoliCamb's device. In addition, there were two particular challenges proposed for the participants to solve. This was in order to %so as to
streamline the event, and %give an initial focus to all attendees:
focus discussion around areas that were of particular importance to SoliCamb's hardware optimisation. The questions posed were:

\begin{itemize}
  \item How do we improve our design in order for it to be waterproof, easy to use and cheap?
  \item How do we measure small capacitance changes using a moisture probe with fewer components (accuracy-complexity trade-off)?
\end{itemize}

The attendees were asked to decide which challenge they would prefer to address, and then were split into two subgroups accordingly. One group worked on designs for device enclosures, while the other worked on the capacitance problem for the moisture retention probe. This made best use of the time available.

%For the SoliCamb team, 
%This event %turned out to be 
%was very rewarding as it %offered the chance to talk and discuss with people from different backgrounds.
%provided a platform for open discussion and ideas exchange. %with people from a range of different scientific backgrounds. %Amongst the participants were PhD students in electrical engineering, experts in Arduino and Raspberry PI as well as software engineers.\todo{chaira I'd mention why this was the intended audience} %and others. % .% I feel this sentence repeats previous content -jh2109
%In particular, the following outcomes were recorded by the SoliCamb team: 
There was a wide range of feedback and advice offered throughout the event, but of particular interest were the comments below.

        \begin{figure}[ht!]
    	\centering
    	\includegraphics[width=0.6\linewidth]{Pictures/Outreach/MSpace.jpg}
    	\captionsetup{justification = centering}
    	\caption{Mini Make-a-thon event at Makespace on 3rd August.}
    	\label{fig:MSpace}
     \end{figure}
\paragraph{Approval of current design}   
   \begin{itemize}[label={$\checkmark$}]
\item Design decisions made to date were broadly correct.
   \item JLBPCB was the correct PCB manufacturer to use for high quality and low cost with a 5 day turnaround. 
   %	\item Consider that not everyone in South Africa will have a smart phone.
   	%\item Cheaper smartphones are generally less powerful, which might not suit our needs.
 \item Modular approach is an extremely good idea. 

   \end{itemize}
   
\paragraph{Suggestions for design alterations}   
\begin{arrowlist}
  \item JLBPCB can also assemble boards.
  \item Eurocircuits was the advised PCB manufacturer to use for a faster 3 day turnaround at higher cost. 
\item Capacitive touch buttons are \enquote{almost for free} at high volume - so should be used.
\item Replacing the base unit with a smart phone would allow for significant cost reductions.
\item Consider Bluetooth rather than USB cables as a cheaper more flexible way to communicate with sensors.
\item In order to increase production volume, a change of the design should be anticipated.
\item Single side component load and double sided board will be cheap for mass manufacture.
\item Integrate all breakout boards onto a double sided PCB with single sided component. %load to got to 1000 volume. 
\item Using a reed switch for power will help with waterproofing.
\item A wire and a grommet would be cheaper than adding a USB connector.
\item Microcontroller only should be considered for measuring capacitance.
\item For simplified phone-app development, frameworks exist; although these might become restrictive later on.
\item A micro-USB port could be used for charging,due to its ubiquity in the smartphone space.

\end{arrowlist}
  
For a discussion of how these changes will be integrated into future designs, see Future Plans section 5. 
  
   
%\begin{itemize}
	%\item When optimising cost there is a danger of buying fake parts so always purchase from the manufacturer. 
	%\item The number of vendors selling a component indicates how popular it is. 
	%\item For 1 million volume have a semiconductor company manufacture a custom chip.
	%\item These frameworks offload app signing to an already-verified third party, considerably simplifying. distribution
% as a microUSB is generally associated with charging 
%\end{itemize}
   
%\textbf{Feedbacks: Sagnik, James, Jan}
%{The suggestions on modifying the device design can be summarised as follows:
%\begin{itemize}
    %\item Modification of the user interface to a "plug-and-measure" type experience (discussed briefly in \ref{section:gui}). Lesser user interaction makes the device easier to use.
   % \item Reduction to 2 buttons ("Select" and "Cancel") instead of 3 and use of LED or buzzer to indicate initiation and completion of a measurement.
   % \item Use of single USB port for charging as well as connecting sensors. If multiple ports are necessary, making them different for the user's ease of perception.
  %  \item Incorporation of a waiting time after the device is switched on, to allow the GPS to lock.}
%\end{itemize}

%\todo{KG - Chiara can you add what was implemented\ not implemented and why?}
 
\subsection{Agri-Tech East Hack-a-thon}
Agri-Tech East is an organisation that focuses on improving the productivity and sustainability of agriculture by bringing together scientists and entrepreneurs, with farmers and growers, to create a global innovation hub in Agri-Tech. Agri-Tech East organised a Grow--Hack-a-thon centered on agriculture in the UK, at the Future Business Centre in Cambridge on 5-7th July 2019. 
The event focused on combating different agricultural problems associated with negative environmental impacts, and aimed to develop sustainable solutions that guaranteed the longevity of food production.
Barclays Eagle Labs, WWF and BASF, the main sponsors of the event, set four challenges; soil health improvement, smarter water irrigation systems, managing soil as a main carbon sink and how to enhance biodiversity.

SoliCamb's motivation for attending this event, was the potential for receiving input on how to define and tackle the main challenges that farmers and the agriculture sector face within the UK and Europe. Therefore, SoliCamb sought to improve their understanding of local needs and market gaps, outside of the initial SoliCamb directive that targets developing countries. 

During the two-day competition, the moisture retention probe was re-built with materials available at the Hack-a-thon, demonstrating the adaptability of our design. Guided by feedback from mentors and other attendees, and the event's focus on UK agriculture, the concept of an easily adjustable moisture retention probe was established. This design would allow the measurement of soil water content at certain depths correlating to the root system specific to the crop. The idea was to deploy multiple water retention probes in a field, creating a network of sensors that wirelessly communicated with a base unit, before the collected data can be uploaded onto a cloud. The design concept comprised a smartphone front-end for data access and visualisation to enable on-demand agronomic decision making (see \cref{fhackathonteam,fhackathondemo}). 
Concerning nitrate leaching, future plans are to implement the moisture retention probe with a smart soil sampling mechanism by integrating an inner, removable rod to allow effortless soil sampling to greater depths. The soil samples could potentially be analysed with C-3NpH. 

Different to the rest of SoliCamb's outreach, the event allowed us to interact with farmers, experts in agriculture as well as Agri-Tech mentors in person and for longer than the length of a phone call. This allowed for extensive conversations about agriculture and sensor technologies in the field and led to new ideas. The event created new perspectives for certain aspects of the SoliCamb project.

Main outcomes of the event were the opportunity to interact and build a collaboration with Agri-Tech East (\cref{Collaborators}), and an invitation by the representative of BASF Agricultural Solutions UK to pitch our project idea to BASF (\cref{Collaborators}).


\begin{figure}
	\centerline{\includegraphics[width=.7\linewidth]{Pictures/agritech_hackathon/hackathon_team.jpg}}
	\captionof{figure}{A photograph taken of the SoliCamb Team present at the Hack-a-thon, with the hardware shown during the final presentation and demo.}
	\label{fhackathonteam}
\end{figure}
\begin{figure}
	\centerline{\includegraphics[width=.7\linewidth]{Pictures/agritech_hackathon/hackathon_demo.jpg}}
	\captionof{figure}{The demo shown as part of the presentation during the Hack-a-thon.}
	\label{fhackathondemo}
\end{figure}
        

\section{Overall insights}
 In order to evaluate our project's public engagement, we used a public engagement impact grid, which is commonly used to assess the level of mutual benefit obtained by involved parties (\cref{tab:impact_grid}). It can be seen that after the completion of our team's outreach events the benefits obtained by SoliCamb and target audiences were balanced. This is in accordance with the definition of public engagement and our key objectives, as outlined in \cref{fig:vendiagram}. %\ref{Citizen engagement}
 
\begin{table}[h!]
	\centering
	\begin{tabular}{l c l l l}
			\toprule
			& {Knowledge \& Awareness} 
			& {Attitudes} 
			& {Skills} 
			& {Empowerment} \\
			\midrule
			SoliCamb & $\Diamond$ $\clubsuit$ $\triangle$ 
					& \hspace{1pt} $\Diamond$  
				& $\Diamond$ $\clubsuit$ $\triangle$
				& \hspace{27pt} $\clubsuit$ $\triangle$ \\
			Audience & $\Diamond$ \hspace{7pt}  $\triangle$ 
				& \hspace{1pt} $\Diamond$ $\clubsuit$ $\triangle$
				& $\Diamond$ 
				& \hspace{18pt} $\Diamond$ $\clubsuit$ \hspace{1pt}  \\	
				\bottomrule
			\multicolumn{5}{c}{\small Key: $\Diamond$ Summer School; $\clubsuit$ Make-a-thon; $\triangle$ Hack-a-thon} \\
  		\end{tabular}
  	    	\caption{Public engagement impact grid used to assess the level of mutual benefit obtained by both parties, after the completion of all outreach events. Presence of a symbol indicates that event fulfilled that requirement.}
    	\label{tab:impact_grid}
    \end{table}



   
      %   \begin{table}[h!]
       % \resizebox{\textwidth}{!}{
    %    \begin{tabular}{@{}lcccc@{}}
     %   \toprule
      %   & Knowledge and Awareness & Attitudes & Skills & Empowerment \\ \midrule
      %  SoliCamb & \begin{tabular}[c]{@{}l@{}} Summer School \\ Make-a-thon \\ Hack-a-thon \\ \end{tabular} & Summer school & \begin{tabular}[c]{@{}l@{}}Summer school \\ Make-a-thon \\ Hack-a-thon\end{tabular} & \begin{tabular}[c]{@{}l@{}}Make-a-thon\\ Hack-a-thon\end{tabular} \\
        %Audience & \begin{tabular}[c]{@{}l@{}}Summer school\\ Hack-a-thon\end{tabular} & \begin{tabular}[c]{@{}l@{}}Summer school\\ Hack-a-thon \\ Hack-a-thon\end{tabular} & Summer school & \begin{tabular}[c]{@{}l@{}}Summer school\\ Hack-a-thon\end{tabular} \\ \bottomrule
        %\end{tabular}%
        %}
        %\caption{Public engagement impact grid used to assess the level of mutual benefit obtained by both parties, after the completion of all outreach events. Key: $\faSunO$ summer school; }
        %\label{tab:impact_grid}
        %\end{table}
    
    Outreach %citizen science 
    events successfully obtained feedback on the usability of the device from citizen scientists and also generated in-field data. The main conclusion drawn from %these experiments 
    time spent measuring nitrates and water retention in Anglesey Abbey and Radwell Grange Farm, was the need for optimising the soil sampling and extraction method, considering both the user practicality of the device in addition to ensuring the validity of results. 
    
    Our online platforms were successful in generating a community of interested users. The social media pages were important tools for advertising the team name, spreading information about the motivation and aims of the project and setting up important networks and collaborations for future outreach events. Our website user session times and overall user traffic were comparable to industrial benchmark standards. Furthermore, users from over 10 different countries interacted with the website. In the future, the high bounce rate observed on a specific set of website pages should be improved by re-designing and including interactive elements.   
    
    % chiara-give a summary of social media main conclusions
    

    







%----------------------------------------------------------------------------------------
%	CHAPTER 4 - Future plans
%----------------------------------------------------------------------------------------
\chapterimage{Pictures/chapter_heads/future_plans.jpeg} % Chapter heading image

\chapter{Future Plans}

\section{Hardware}

\subsection{Moisture retention probe: Continuous monitoring solution}%Author: jh2109
\label{smrpcontinuousmonitoring}
%DONE\todo{jh2109 and probably others. Mentioning single-PCB system integration idea, low-cost mass (medium-scale) manufacture); explain how this could fit into what David Jordan (National Trust) and others have asked for}
% @others: Please feel free to add or contribute other aspects of this approach
While a fully portable solution has its benefits (safe from theft, quick to deploy repeatedly at different sampling locations), it also comes with the inherent disadvantage of setup time for every use and very limited, point-wise sampling in time. Many of the experts, especially within the UK, were more interested in a continuous monitoring solution, which had been ruled out early on in the project due to the requirements given by UCPP.

\todo{jh2109 I had to revert most changes made here. The way it had been rewritten was simply untrue, making it sound as though we had done those things already.}
Throughout the project, multiple approaches to design the moisture retention probe have been tested. From this experience gained, one promising future vision consists of a fully integrated, one-board PCB solution where the coils on the rod of the moisture retention probe are replaced by inter-digitated traces on the PCB. One of the reasons such a solution was disfavoured for the portable in field testing probe were wicking issues. Whenever such a board was inserted into the ground, water would follow the cable's path. For a continuous monitoring solution however,this could be circumvented by embeddeding the probes horizontally at different depths. This was trialled for the initial proof of concept (see \cref{smrdesignevolution}), but discontinued due to the excessive amount of work per hole dug, which would be amortised in time for a continuous monitoring solution.

The appeal of leaving the moisture retention probe in the ground to measure continuously is not only in eliminating the setup time required for each measurement, but in entirely automating the system to become autonomous and not require any human intervention. Natural rainfall would trigger the sensor and replace the manual irrigation required for the portable in-field measurement protocol. The inherently large-scale, high-volume irrigation will improve measurement quality and reliability by eliminating the variation of how the system is irrigated. Furthermore, local weather forecasts or local weather station measurements can be correlated into the measurement algorithm.

Furthermore, unlike the present solution, such a system would have significantly reduced difficulties with variation due to soil compaction (a new equilibrium is established over time), and the PCB integration would allow point-of-measurement analog-to-digital conversion, which would free up the road to a no-added-component capacitance measurement that has been shown to work as part of the Cambridge Makespace Make-a-thon. The idea is that the microcontroller begins to charge the moisture measurement capacitor and thereafter repeatedly samples its voltage over time to find the charging timescale and hence capacitance. This measurement was not possible to implement in the present solution due to the high parasitic and variable (wire movement) capacitance of the wires leading up to the sensing element, but would be feasible for the one-board solution.

Integration of this continuous measurement into a soil health monitoring toolkit would likely be via a solar-powered measurement sub-station communicating to a mains-powered base station, which was explored during the Agri-Tech East Hack-a-thon. Such a remote sensing approach is feasible because the communication distance is only to the main-powered base station in an office, at most a few hundred meters away, rather than kilometres as in the UCPP context. In particular, collaborators at the National Trust have raised interest in such a solution, opening up the opportunity to field-test this approach at a nearby site. 

\subsection{C-3NpH} \label{future:c3nph}
\paragraph{Measurements of other soil nutrients} \label{future_nutrients_c3nph}

As aforementioned, C-3NpH has the ability to measure other nutrients (\textit{viz.} nitrites, phosphates, ammonia), provided that suitable test strips (i.e. with a mappable colour gradient) can be found. Apart from narrow range test strips to improve resolution, the colour of the illuminating LED (currently white) may be matched to the tested strip colour, i.e. a reddish LED for the nitrates strips to better distinguish between different shades of purple-pink. 
\paragraph{Comparison of results with commercial laboratory test}

In addition to field testing, soil was collected for lab validation. This was sent to a local industrial soil testing laboratory (ChemTest, see \cref{Chemtest}). The pH results returned were significantly different to those measured by C-3NpH and this was thought to be due to differences in sample preparation. It was concluded that the extraction protocol here was not sufficient for determining a pH measurement. Sample drying, mixing time and sample size are key influencing factors that have differed from protocols conducted in industry. This forms a crucial part of the future plans for developing C-3NpH in order to confirm the validity of results obtained in the field. Elimination of any discrepancies or errors introduced from sampling remain a priority for C-3NpH development.

\paragraph{Moisture-corrected nitrate values}

Soil moisture content, in addition to determining structural properties, inherently affects nutrient availability in the soil. As the test strips are calibrated in ppm of nitrate, it would need to be converted to ppm of nitrate-nitrogen, on a dry soil basis to determine the amount of nitrogen available to the crop \cite{moisturenitrate}.
Therefore, using C-3NpH in conjunction with the moisture retention probe will then provide the capability to report values of nitrates in dry soil, without having to air dry them overnight. 

		\begin{table}[h!]
		\centering
		\begin{tabular} {l c c}
			\toprule
			& \multicolumn{2}{c}{\textit{Correction factor}} \\
			\midrule
			\textbf{Soil texture} & \textbf{Moist soil} & \textbf{Dry soil} \\
			\midrule
			Sand & 2.3 & 2.6 \\
			Loam & 2.0 & 2.4 \\
			Clay & 1.7 & 2.2 \\
			\bottomrule
			\end{tabular}
		\label{table:corrfacnit}
		\caption{Correction factor for soil nitrates \cite{moisturenitrate}}
		\end{table}

 Hartz \cite{Hartz1994} illustrated for a moisture content range of 6-30$\%$, field-moist sample preparation by displacement (i.e. adding soil to a known volume of solvent until the meniscus has been moved by a set amount, gave an acceptable correlation to those values obtained from laboratory-based extraction (using \SI{2}{M} KCl). He further provided evidence that the semi-quantitative \gls{QNT} negated the need for dried soil, provided conversion factors of moisture content is known \cite{Hartz2010}. \Cref{table:corrfacnit} states a reference chart of correction factors to apply, based on soil type and moisture content. However, further work will need to be conducted to obtain an accurate moisture-corrected report of nitrate, based on values obtained from SoliCamb’s moisture-retention probe.
 
\paragraph {Providing mitigating protocols}
Knowledge of soil nitrate guides the user to make management decisions to increase yield and decrease production costs. In the case of residual nitrates in agricultural soil, nutrient management practices should be adjusted to decrease leaching into groundwater or prevent their runoff into surface water. C-3NpH could be used for governing soil conservation efforts, by dictating the required amount of fertiliser to add. The three key times where knowing the nitrate concentration would be useful are:
\begin{itemize}
\item Before the start of a growing season, to provide a baseline reading for the field. 
\item Throughout the growing season, before side dressing crops (fertiliser application) as a part of the nutrient management plan. 
\item Post-harvest, to determine if there is excess nitrate remaining in the soil.
\end{itemize}

Despite the theoretical benefits discussed above, further work is needed to make this device practically suit the needs of the agricultural community. In addition to this, a knowledge base of conservation practices/mitigating protocols for different levels of soil nitrate (obtained using C-3NpH) at different critical time points throughout the year would have to be formulated.

%\subsection{Spectroscopy for soil carbon (Sagnik)} % Consider to move to appendix? - link from some hardware section



\section{Software} \label{soft_future}
%Talk about choosing noSQL database (MongoDB) for unstable schema
Currently, a relational database is used for data storage. However, towards the end of the project, it was realised that if the system is used by a scientist, there is a possibility of them wanting to add extra columns, e.g. the error associated with a measurement. Further discussions with potential users are required to establish whether the ability to add extra columns would be useful. If it was decided to be essential, then it would be necessary to switch from a relational database (MySQL) to NoSQL (MongoDB). Moreover, if it was decided that dynamic queries are required for more descriptive and user-friendly data visualisation, MongoDB would also be a superior choice.

%Design changes based on already done user experience tests
Some minor web design changes were suggested by the user experience test volunteers, such as ``add information about what the data means'' and ``should add pictures to make the website more appealing''. These improvements are easy and quick to implement. Also, it could be beneficial to consider implementing the functionality for people to create individual accounts (ideally linked to the social media of their choice) and having an option of storing their collected data privately or sharing it. 

%Map changes: colour maps by specific parameters, timeline management, GPS averaging
The data visualisation is currently relatively basic, but in the future the next steps would be to:
\begin{itemize}
    \item allow the user to select parameter of interest and colour code the map accordingly;
    \item timeline management requires consideration, whether to have a slider allowing the user to see the history of measurements, or graphs within infowindows showing how each parameter has changed with time;
    \item GPS averaging might be needed in order to allow the user to record values that have been collected in the same location (yet according to GPS have slightly different coordinates).
\end{itemize}

%Keep iterating based on user feedback
Finally, more user experience testing is required and further website iterations based on user feedback would be performed.


\section{Outreach}
To date, SoliCamb's outreach and public engagement has been successful in building a community of professionals and citizen scientists. The team aims to continue running multiple events to take full advantage of the network built over the previous few months. Firstly, SoliCamb has been invited to participate in  Peterborough STEM Festival and at Sensors Day and the Sensors  Showcase in October 2019. The former is a two-day event designed to engage and inspire families and young STEM enthusiasts.  The latter consists of two conferences led by experts from academia and industry highlighting the latest developments in sensor technology.

In order to finance future developments and events, funding applications are currently being considered which include the the Public Engagement Starter Fund and the Cambridge Africa ALBORADA Research Fund, both from the University.\\

Through local outreach events SoliCamb plans to continue engaging with citizen scientists around Cambridge. This has already proven successful for facilitating user testing and field testing, and conversations about extending field testing to multiple National Trust sites have been initiated. If the sensor system sent to Ethiopia generates sufficient interest, an international collaboration could also be considered. Along these lines, a Development i-Teams project will be run in partnership with the \gls{CGE} through Michaelmas Term 2019. The students recruited for the project will be able to help SoliCamb by identifying regions in Africa where this project could be the most valuable and establish useful contacts. The ALBORADA Trust can then further enable Cambridge researchers to engage with African colleagues. This avenue is particularly interesting as the sensors could be highly valuable for conservation efforts. \\
Make-a-thon-type events in countries such as Ethiopia can be envisaged where the team presents the project and walks through the production of multiple sensors with the local public. The citizen scientists can then collect data which will easily be mapped on the web application developed for this purpose. 


%----------------------------------------------------------------------------------------
%	CHAPTER 4 - Conclusions
%----------------------------------------------------------------------------------------

\chapterimage{Pictures/title_page/Conclusion.jpg} % Chapter heading image

\chapter{Conclusions} %500 words MAXXXX

%WHat was the problem

%What have we done

%What have we learnt


The aim of the project and SoliCamb's goal was to develop low cost, open source sensors for monitoring the key indicators of soil health. Great progress has been made in achieving this with a first
prototype designed and built, as well as two sensor peripherals being developed and validated. 
The project evolved considerably during the 10-week period from a device aimed purely at developing countries or resource limited markets, into a device  that has real potential for adding value to how those in the developed world agriculture and farming industries manage their land. 

Ultimately, a base unit, two peripheral sensors and associated data management platform were developed. 
The base unit serves the purpose of offering a user interface as well as storing geographically and time stamped data produced by the peripherals. 
The first peripheral consisted of a probe that can be inserted into the ground and irrigated with water to measure moisture infiltrative properties of the ground in question, based on feedback from our collaborators this peripheral seems the most valuable as a tool within agriculture and land management although there are clearly still some technical challenges and questions in the implementation.
The second peripheral was a nitrate and pH sensor. This was built to fit the nitrate test strips as a cheap and easy alternative to spectrophotometry. The device, C-3NpH, was pivoted at two separate audiences. The ease with which the model and RGB colour sensor could detect changes in colour gradients meant that C-3NpH was successfully deployed in field at two test sites within the Cambridgeshire area. On the other hand, the pH model was limited by the strip resolution in addition to the complexities in distinguishing between colours and not just a single gradient as with nitrate. Therefore, pH sensing was used as an educational tool and demonstrated to local citizen scientists the benefits of quantitative measurement, sensor design and electronics with Arduino. Both facets of C-3NpH demonstrated success and are poised to move forward, using optimised soil sampling procedures. 

A key deliverable was to facilitate data management, initially for those in UCPP, but the added value of combining a web application with our hardware in the UK quickly became apparent. To this end a  web application was constructed that allows the user to store and monitor data recorded across locations and over time. Data can be uploaded manually using simple csv files in a pre-defined format by using the RFID tags that were designed to accompany the modular base unit.

Our outreach channels created a strong network of academic and industrial collaborators, with whom we will continue to work in the future. It was crucially important that SoliCamb expanded it's network beyond that of the University in order to best understand the context of soil health and where the device fit in this arena. 

The exact future of the project remains an open question. Given the momentum and attention we have generated there are multiple directions the project could be taken. Collaborators within charities, universities and industry are actively interested in the topic of open-source sensing, citizen empowerment and smart agriculture; encompassing both developing and developed countries. Several members of the team have already expressed interest in pursuing the project further. The next 6 months will be a pivotal time for SoliCamb as the networks and opportunities created over these 10 week are expected to come to fruition, along with proposals for the funding where SoliCamb stands in good stead to submit thorough and detailed applications. 


\chapterimage{Pictures/title_page/teamphoto.jpg}
\chapter{Report Contributions}

\begin{table}[h]
\begin{tabular}{|l|l|}
\hline
\textbf{Author} & \textbf{Chapters/sections} \\ \hline
T. Baissas &  
\begin{tabular}[c]{@{}l@{}}
        \textbf{Science and Technology -- Hardware}: Moisture Retention\\ 
        \textbf{Future Plans}: Moisture Retention, Outreach
    \end{tabular}
\\ \hline
C. Barberio & 
\begin{tabular}[c]{@{}l@{}}
\textbf{Introduction}: Soil Health\\ 
 \textbf{Citizen Engagement and Outreach}: Online Public Engagement, \\ Social Media Platforms, Radio and TV Feature, Make-a-thon\\
  \textbf{Future Plans}: Outreach
\end{tabular}
\\ \hline
S. L. Barron &  
  \begin{tabular}[c]{@{}l@{}}
        \textbf{Introduction}: User requirements\\ 
        \textbf{Citizen Engagement and Outreach}: Importance of citizen engagement, \\
        Website and Newsletter, Traditional methods for advertisement,\\ Overall insights\\ 
        \textbf{Future Plans}: Outreach
    \end{tabular}
\\ \hline
E. Bondareva &  
    \begin{tabular}[c]{@{}l@{}}
        \textbf{Project Outline}: Project management, Project progression\\ 
        \textbf{Science and Technology -- Software}\\ 
        \textbf{Future Plans}: Software
    \end{tabular}
\\ \hline
C. W. Chung &  
    \begin{tabular}[c]{@{}l@{}}
        Executive Summary, Acknowledgements \\ 
        \textbf{Project Outline}: Allocation of tasks (\cref{sec:Task_allocation})\\ \textbf{Science and Technology -- Hardware -- C-3NpH}: \\
        %Introduction (\cref{sec:current_C3nph}): Current detection approaches and C-3NpH; \\
        %Aims \& Objectives (\cref{sec:aimobj_c3nph}); \\
        %Design Decisions (\cref{sec:cs_c3nph}): Colour sensor, Test strips, 3D printed enclosure; \\
        %Results \& Discussion (\cref{phresults_c3nph}): pH;  \\ %: Introduction, Aims \& Objectives, Design Decisions, Results \& Discussions \\
        \textbf{Future Plans -- C-3NpH} (\cref{future_nutrients_c3nph})
        %: Measurements of other soil nutrients \\
        %\textbf{Appendices}: C-3NpH \\
    \end{tabular}
\\ \hline
K. Gibson &  
\begin{tabular}[c]{@{}l@{}}
        \textbf{Project Outline}: User Requirements, Aims and Objectives \\
        \textbf{Citizen Engagement}: Soil Sampling Sites, TV and Radio Feature, \\
        Outreach and Public engagement \\
        \textbf{Science and Technology - Hardware}: C-3NpH \\
        \textbf{Conclusions} \\
        \textbf{Future Plans}: C-3NpH \\
    \end{tabular}
\\ \hline
J. Heck &  
\begin{tabular}[c]{@{}l@{}}
\textbf{Science and Technology -- Hardware}: Moisture Retention Probe:\\
Background and Idea, Measurement Protocol (Author)\\ 
\textbf{Citizen Engagement and Outreach}: Hackathon (Editing),\\
Make-a-Thon (Editing)\\
\textbf{Future Plans}: Moisture Retention Probe (Author)\\
\textbf{Appendix}: Moisture Retention: Probe Design Evolution (Author)
\end{tabular}
\\ \hline
J. T. Meech &  
\\ \hline
S. Middya &
\begin{tabular}[c]{@{}l@{}}
        \textbf{Introduction}: User requirements\\
        \textbf{Science and Technology -- Hardware}: Main device concept,\\ Device firmware\\ 
        \end{tabular}
\\ \hline
S. Pavagada &  
\begin{tabular}[c]{@{}l@{}}
        \textbf{Science and Technology -- Hardware}: C-3NpH \\
        \textbf{Future Plans: C-3NpH} \\ 
        \end{tabular}
\\ \hline
E. Sch{\"a}fer & 
\begin{tabular}[c]{@{}l@{}}
    \textbf{Science and Technology -- Hardware} : \gls{CO2} sensor\\
    \textbf{Citizen Engagement and Outreach}: Professional collaborators\\
    \textbf{Citizen Engagement and Outreach}: Agri-Tech East Hack-a-thon\\
\end{tabular}
\\ \hline
X. Tan &  
\begin{tabular}[c]{@{}l@{}}
        \textbf{Science and Technology -- Hardware}: C-3NpH (Prediction Models) \\
        \end{tabular} 
\\ \hline
D. C. van Niekerk & \begin{tabular}[c]{@{}l@{}}
        \textbf{Science and Technology -- Hardware}: Design Concept,\\ 
        Moisture Retention - Simulation and Analytical Treatment\\
        \end{tabular} 
\\ \hline
B. Woodington &  
\begin{tabular}[c]{@{}l@{}}
        \textbf{Science and Technology -- Hardware}: Moisture Retention\\ 
        \textbf{Citizen Engagement and Outreach}: Immerse Education Workshop\\ 
        \textbf{Conclusions}\\ 
        \end{tabular}
\\ \hline
Y. Wu &  
\begin{tabular}[c]{@{}l@{}}
        \textbf{Citizen Engagement and Outreach}: Agri-Tech East Hack-a-thon \\
        \end{tabular} 
\\ \hline
\end{tabular}
\end{table}


%----------------------------------------------------------------------------------------
%	BIBLIOGRAPHY
%----------------------------------------------------------------------------------------
\chapterimage{Pictures/chapter_heads/bibliography.jpg}
\chapter*{Bibliography}
\addcontentsline{toc}{chapter}{\textcolor{ocre}{Bibliography}}
\tocless \section*{Books}
%\addcontentsline{toc}{section}{Books}
\printbibliography[heading=bibempty,type=book]
\tocless \section*{Articles}
%\addcontentsline{toc}{section}{Articles}
\printbibliography[heading=bibempty,type=article]

%----------------------------------------------------------------------------------------
%	Appendices
\chapterimage{Pictures/title_page/Appendice.jpg}
\chapter{Appendices}
%We could use the appendices package to typeset the appendices as not another chapter

\section{Hardware}
\tocless \subsection{Feasibility studies for soil organic matter determination in soil}
Soil organic carbon (\gls{SOC}), a product of on-site biological decomposition of plant, bacterial and animal residues, is a highly important indicator of overall soil quality \cite{Gregorich, Haynes, WanderDrinkwater}. Its composition and breakdown rate affect chemical, physical and biological properties of the soil, such as nutrient availability, biodiversity, and biological activity as well as the soil structure and porosity, thus water holding capacities and water infiltration rate. 

The methods of evaluating \gls{SOC} are based on \textit{ex situ} measurements carried out under controlled conditions in a laboratory setting, resulting in a multiple day process from sample collection to data availability. Being able to monitor \gls{SOC} in field would allow an immediate overview over several parameters of the soil while allowing appropriate actions in real time and would decrease the amount of data handling necessary. Thus \gls{SOC} is one of the main parameters that were listed from end users as tool of need for in field soil monitoring. 

\subsubsection{Standard method -- Loss on ignition (LOI)} \label{StandardmethodsLOI}
The standard procedure to determine the amount of \gls{SOC} is the \textbf{\gls{LOI}} test which determines the weight loss of a dried soil sample (here dried at \SI{70}{\celsius} for 17 hours)  after combustion of the organic material by exposing it to \SI{500}{\celsius} for 8 hours. The amount of mass lost after the \gls{LOI} treatment is equal to the \gls{SOC} amount of the sample. \gls{LOI} is an easy, cheap and highly reliable method. Due to its implementation requiring a high precision mass balance, a drying oven, a temperature controlled muffle furnace and preheated crucibles next to the soil sample of interest, this procedure is not modifiable for in field use. 

A more accurate method is the \textbf{dry oxidation} of soil for which an automated carbon analyser is used to measure the \gls{CO2} amount that has evolved. The application of high temperature oxidises the organic matter present in soil and is manifested as a loss in the mass of the soil sample. Estimation of \gls{SOC} by wet oxidation of soil is also accepted as a standard procedure. It involves chemical treatment of soils samples with strong oxidising agents like dichromates in an acidic medium at \SI{175}{\celsius}. Any organic content present in the sample is converted to $CO_{2}$. $Cr^{6+}$ in $Cr_{2}O_{7}$; on the other hand is reduced to $Cr^{3+}$, which is associated with a sharp colour change from orange to green. The amount of carbon can be calculated by quantifying the change in optical absorbance corresponding to the generated $Cr^{3+}$ or by titrating the residual $Cr^{6+}$. It is also assumed that the average valence of the carbon present in soil is zero. The method of wet chemical oxidation has been modified over the years to account for the interference caused by ions, including $Fe^{2+}$, $Cl^{-}$, and $Mn^{3+}$/$Mn^{4+}$.\cite{Chatterjee2009} 

%All the \textit{ex situ} measurement techniques described above are robust and often considered as the standard. However, they require sample preparation, have quite involved measurement protocols or require a long turnaround time (up to 5 days). 
Besides chromate based chemical test kits, which cause a safety hazard to the environment and individuals no \gls{SOC} tests are readily available for in field usage. %Due to \gls{SOC}



\subsubsection{Feasibility study -- Spectroscopy }\label{SOC_feasibility}
Techniques based on spectroscopy and remote sensing often present a low-cost and easy-to-use solution. \gls{IR} spectroscopy has proven to be a powerful candidate for proximal measurement of soil carbon and soil health monitoring in general. The frequency ranges used for this purpose are the \SI{400}{nm} to \SI{2500}{nm} (\gls{NIR}) and the \SI{2500}{nm} to \SI{25000}{nm} (\gls{MIR}).\cite{Chatterjee2009, Bellon-Maurel2011, ViscarraRossel2015} \gls{IR} spectroscopy is usually implemented in a diffused reflectance mode or an \gls{ATR}. In the first, light reflected in all directions by rough surfaces of the sample is analysed; \gls{ATR}, on the other hand, depends on total internal reflection at the interface of a crystal and the sample. This produces an evanescent field into the sample, which bears the useful information about material composition. The absorbance in the \gls{MIR} region is attributed to the characteristic molecular vibrations of the chemical compounds present in the sample; whereas the \gls{NIR} region corresponds to overtones and combinations of the fundamental bands making the absorbance (extinction coefficients) lower than the former. It is also sensitive to physical structure and other external factors.\cite{Bellon-Maurel2011} The \gls{IR} spectra of soil cannot be directly correlated with the \gls{SOC}. This is because the spectra gives information about the nature of chemical bonds at the molecular level and does not quantify \gls{SOC} as a whole. Quantitative \gls{SOC} values are hence obtained by correlating the spectral information with known measurements of reference samples. This is performed by various statistical methods that establish empirical relations between the target attribute, i.e. \gls{SOC} and absorbance values at selected wavelengths. Commercial spectroscopes include elaborate optics for accurate performance, which make them bulky and unsuitable for deploying in the field. Portable \gls{MEMS} technology based ones are available, but they are expensive. With these problems in mind, there have been recent efforts in developing low-cost open-source spectrometers at the expense of a limited spectral range.\cite{oursci_blog} The hand-held device works at few discrete frequencies between \SI{365}{nm} and \SI{1800}{nm} and connects to a smart-phone application for data analysis.

As described earlier, since the spectral response depends on various physical factors, its correlation to meaningful \gls{SOC} values changes with the soil type. For example, excellent agreement ($R^{2}$ = 0.961) is observed if all calibration samples have similar particle size distribution; while it becomes worse for samples with heterogeneous particle sizes.\cite{Russel2003} These suggests that calibration is largely location specific. Apart from that, water, a prime component of soil, absorbs strongly in the \gls{NIR} range. Hence, drying the samples prior to spectroscopic analysis is essential to get rid of the interference from water molecules. Such additional sample processing steps, in many occasions, inhibit the reliable use of \gls{IR} spectroscopy in the field. In view of these practical difficulties, determination of \gls{SOC} through IR spectroscopy was considered unfeasible and abandoned. 

\subsubsection{\gls{CO2} sensor -- PhotosynQ}\label{CO2sensor}
After several feasibility studies (\cref{SOC_feasibility}), SoliCamb decided that developing a reliable sensor for \gls{SOC} would be dependent on a vast amount of research and developing steps which would not be feasible within the time frame or the budget available for the project. Due to multiple collaborators as well as experts and academics in the field expressing high interest and elucidating the demand for a soil organic carbon sensor we have tried to find a suited but existing technology that could be integrated into the SoliCamb sensor platform rather than developing our own sensor. \\
A collaboration with PhotosynQ, an open source project originating in the USA, who are working on providing a platform to create, share and analyse plant health information was initiated. In order to link soil properties to plant health, they have developed the SoilspeQ, which allows the measurement of carbon mineralisation within soil. The measurement provides information on the active pool of the total organic carbon which consists of rapidly cycling organic material (5-20 \% of \gls{SOC}) \cite{Gregorich,Wander04} and is a good indicator for short term nutrient availability \cite{Haynes, Lewis, Wander04}. Beyond this, carbon mineralisation allows the prediction of crop yield and total above ground biomass. PhotosynQ has been developing the device for quite some time and has successfully utilised it in remote areas in Australia and Malawi.\\
PhotosynQ’s SoilspeQ measures carbon mineralisation from a 24 hour burst aerobic incubation using a setup which includes a Teensy microcontroller and a IRGA K30 \gls{CO2} sensor. 
For a valid measurement, soil needs to be air dried for 48 hours before 25 grams are transferred to an air tight container of known volume. The soil is re-wetted uniformly in a 5:1 ratio using water. The lid of the container is to be prepared with an airtight seal which allows the extraction of 30 ml of air from the jar after a 24 hours incubation at room temperature, using a syringe and needle. The air in the syringe is then applied to the \gls{CO2} sensor in a uniform, smooth movement. Subsequently, the \gls{CO2}  measurement is used as a proxy and converted to determine the rate of organic matter mineralisation which is interlinked with short term soil nutrient availability. \cite{HurissoCulman}.\\
To circumvent using the PhotosynQ WebApp and allowing the potential integration of the device in the SoliCamb sensor platform, we have extracted the information of how to convert the \gls{CO2} to a carbon mineralisation from their firmware (\cref{fig:conversion-code}), allowing an instantaneous readout with our device. This will also facilitate the creation of the final \gls{csv} file. Instead of a Teensy board, we are yet using an Arduino, due to compatibility between the IDE of the Arduino and the Blue Pill microcontroller, which is used in our base unit(see \cref{fig:SetupDetailled},\cref{fig:CO2_setup}). 

\begin{figure}[h]
\centering
\includegraphics[width = 6in]{Pictures/CO2/Conversion-code.png} 
\caption{Code describing the conversion step from \gls{CO2} to amount of carbon per kg of soil, extracted from the PhotosynQ firmware and re-purposed for the SoilCamb device.}
\label{fig:conversion-code}
\end{figure}

Preliminary experiments have been performed with soil provided by Madingley Mulch and allotments at King's College. Experiments have been run in triplicates and results have been compared to the result of the \gls{LOI} test of the same batch of soil. The \gls{LOI} test was carried out in the Department of Plant Sciences under the supervision of Dr. Yi Zhang (see \cref{StandardmethodsLOI} for \gls{LOI} protocol).

\begin{figure}[h]
    \centering
    \begin{subfigure}[b]{0.48\linewidth}        %% or \columnwidth
        \centering
        \includegraphics[width=\linewidth]{Pictures/CO2/CO2scale.jpg}
        \caption{The equipment used for soil preparation and the 24 h burst test using air tight containers.}
        \label{fig:SetupA}
    \end{subfigure}
    \begin{subfigure}[b]{0.48\linewidth}        %% or \columnwidth
        \centering
        \includegraphics[width=\linewidth]{Pictures/CO2/CO2sensor.jpg}
        \caption{showing the setup of the \gls{CO2} sensor using an Arduino Uno}
        \label{fig:SetupB}
    \end{subfigure}
    \caption{A detailled overview of the components of the \gls{CO2} sensor setup.}
    \label{fig:SetupDetailled}
\end{figure}


\begin{figure}[h!]
\centering
\includegraphics[width = 6in]{Pictures/CO2/CO2setupf.jpg} 
\caption{Preliminary setup of the rebuilt \gls{CO2} sensor developed by PhotosynQ using a 500 mL Schott Pryex flask, a 60 mL syringe and an Arduino Uno microcontroller.}
\label{fig:CO2_setup}
\end{figure}

The results of both tests  shown in \cref{fig:CO2_LOI} show the same trend for four different soil samples from different soil types. 


\begin{figure}[h!]
    \centering
    \includegraphics{Pictures/CO2/CO2_LOI.png}
    \caption{Comparison of results from the \gls{CO2} sensor to \gls{LOI} for different soils. \small[N.B. Error bars are based on standard deviation.]}
    \label{fig:CO2_LOI}
\end{figure}

%% (\cref{fig:CO2_LOI}) picture reference for you Elena

The \gls{LOI} test result represent the percentage of \gls{SOC} content of the soil samples, whereas the results from the \gls{CO2} sensor, represent the amount of active organic carbon of the soil only. These experiments were to test if the carbon mineralisation data can be used as a proxy for \gls{SOC} or if the results are a measure of biological activity and nutrient availability only. This would affect the role its role, if integrated in the SoliCamb platform. The test has been run using various soil types, and we can conclude that the method, unlike other methods, is independent of the type of soil that is to be analysed. The variability of the results is low, but there are a number of steps where errors could be introduced by the user. 

So far, the test has only been carried out with air dried soil. We are interested to test the influence of temperature and water content on the respiration rate measured. Beyond that, we have received interesting input from collaboration partners at the Oxford Ecosystems Lab on how to potentially speed up the process of carbon mineralisation. This has not yet been tested by PhotosynQ, and we are planning on investigating this in future. Furthermore we want to understand the link of the measurements with nutrient availability, but running more tests focusing on nitrate availability in the soil. We are working on further validating the system, and adjusting the setup and handling of the single components in order to decrease the possibility for user errors. So far, we can conclude that the device shows high potential and is most likely to be integrated into the SoliCamb sensor platform. 



\tocless \subsection{Base unit}
\subsubsection{GPS accuracy testing}\label{section:gps_accuracy}
%Author: Sagnik
 \begin{figure}[ht]
		    \centering
	   \includegraphics[width=\linewidth]{Pictures/Hardware/gps_accuracy.jpg}
			\caption{Visual representation of the different co-ordinates reported by the GPS for the same position (on Google Maps).}
			\label{gps_accuracy}
    	\end{figure}
UBlox Neo 6-M \gls{gps} modules were chosen based on their affordability and ease of availability. Moreover, being very common among hobbyists, plenty of documentation and troubleshooting is available for reference. Geo-tagging is considered an important part of the sensing exercise for tracking the quality of soil at a location over a period of time. Hence, the co-ordinates of the same position reported by the \gls{gps} over one hour was analysed as a test of its accuracy. \Cref{gps_accuracy} shows the spatial spread of 10 representative co-ordinates thus measured. Statistical analysis suggested that the location accuracy is \SI{12.76}{m}. This is acceptable for our purpose as soil quality is expected to change only over larger distances. 

\subsubsection{Process flow of the base unit firmware}\label{appendix:processflow}

The base unit houses different hardware sub-units of different latency. The need to schedule them carefully have given rise to the process flow shown in \cref{firmware_flowchart}. On start-up, the \gls{gps} and SD card are initialised since they are the compulsory requirements. Powering the \gls{gps} from the very beginning has its consequences on the battery life, but is helpful as it allows enough time to lock. After the sensing operation is executed, the date, time, and location data are read from the \gls{gps}. This is performed in a loop until the data is valid or the timer is exceeded, whichever is earlier. If the geo-tagging is successful, sensor data along with the date-time and location are logged into a \gls{csv} file in the SD card. Otherwise, the user is intimated, and the entire sensing process and acquired data are abandoned. Even if multiple sensor measurements are taken at the same location, data from each sensing exercise, are logged as separate rows in the \gls{csv} file. Although non-intuitive, this simplifies the data logging process from the firmware perspective. Thus, measurements from different sensors may bear slightly different co-ordinates depending on the \gls{gps} accuracy. This is taken care of during data post-processing; for example, by limiting the significant digits of the co-ordinates. This is complemented by adding unique device and sensor IDs corresponding to each measurement.

\subsubsection{Battery life} 
%Author: James
The battery life was measured using the state-of-the-art Keithley battery measurement setup~\cite{JM_batTestSetup}. A battery characteristic for the specific 1850 cell used in the device was measured using the experimental setup shown in ~\cref{Figure:JM_smutest}. The battery characteristic was then uploaded to the battery simulator, and the battery life of the device was measured using the experimental setup shown in \cref{Figure:JM_batsim}. 

\Cref{Figure:JM_esrsoc} shows the discharge curve of one of the 18650 cells used to power the device. The discharge curve was obtained using a Keithley 2450~\cite{JM_2450SMU}. The device shuts off at \SI{2.88}{V} leaving 1.72 \% of the battery capacity unused. This could be improved by redesigning the regulator to work down to a lower voltage. It can also be seen that the equivalent series resistance of the cell rises sharply as it approaches the fully discharged state. \Cref{Figure:JM_dist} shows a plot where the battery characteristic obtained using the 2450 was uploaded to the Keithley 2281S battery simulator~\cite{JM_2281} which was used to power the device. The base unit was configured with the LCD back-light on, the most power hungry peripheral sensors connected, and with the GPS and RFID reader active. This caused the device to draw the maximum possible current leading to the minimum possible battery life. The device worked for 16 hours and 12 minutes, so it follows that two cells will power it for 32 hours and 24 minutes. The voltage and current plot highlights that the supply is extremely noisy with nominal \SI{15}{mV} voltage noise and current oscillations of \SI{24}{mA}. The voltage oscillated with a amplitude of \SI{132}{mV} which is worrying. This should be investigated and rectified with filtering in future designs. The voltage across the terminals of the base unit is consistently lower than that across the output of the battery simulator due to the burden voltage of the ammeter~\cite{JM_burden}.


    
    
\begin{figure}
	    \centering
	    \includegraphics[width = 200pt]{Pictures/Hardware/JM_smutest.pdf}
	    \captionsetup{justification = centering}
		\caption{Experimental setup for the battery drain test on the source measurement unit.}
		\label{Figure:JM_smutest}
	\end{figure} 
	
\begin{figure}
	    \centering
	    \includegraphics[width = 200pt]{Pictures/Hardware/JM_batsim.pdf}
	    \captionsetup{justification = centering}
		\caption{Experimental setup for the battery simulator test to determine the minimum battery life of the base unit.}
		\label{Figure:JM_batsim}
	\end{figure} 
	
	\begin{figure}
	    \centering
	    \includegraphics[width = 300pt]{Pictures/Hardware/JM_dist.pdf}
	    \captionsetup{justification = centering}
		\caption{Voltage and current curves obtained from powering the base unit from the battery simulator until the regulator drops out.}
		\label{Figure:JM_dist}
	\end{figure}

   
\begin{figure}
		    \centering
		    \includegraphics[height=400pt]{Pictures/Hardware/esrsoc.eps}
		    \captionsetup{justification = centering}
			\caption{Plot of the battery stage of charge and ESR vs voltage, it can be seen that the ESR increases as the battery voltage drops. The battery is mostly discharge when the voltage reaches x volts and the supply drops out.}
			\label{Figure:JM_esrsoc}
    	\end{figure}
    	
\subsubsection{Troubleshooting} 
%Author:  - James + Sagnik
The probability of receiving a \gls{pcb}, soldering on the components, programming the microcontroller and observing it do exactly what was intended is extremely small. It is significantly more likely that the device will not work the first time around. It is therefore important to first determine whether the issue is with the hardware or the software. If the issue is in software, then print statements can be used to quickly determine where the system fails. If the issue is with hardware, first voltages should be checked to identify any power supply issues, and then connections should be checked using a multi-meter (with the circuit disconnected from the power). Finally, an oscilloscope should be used to asses the signal integrity of communications, and then a logic analyser used as a last resort to interrogate the bits being sent between devices. Extensive troubleshooting was required to get the hardware designed in the project working. Troubleshooting of an intermittent problem with the \gls{gps} resulted in the addition of an external antenna. The close proximity of the antenna to metal objects with the case caused the intermittency as described in~\cite{JM_antenna}. 
    	

\subsubsection{OOPI}
\label{sec:OOPI_Appendix}

The OOPI software boasts three primary features of interest, namely: the robust, atomic communication protocol underlying the OOPI API, the formulation of a general sensor model to facilitate the development of a generic, sensor non-specific base unit and the use of the object oriented coding paradigm to develop an abstracted model of the sensor with which the base unit can interface polymorphically. These are discussed in more detail below.

\paragraph{Transactions}
\label{sec:OOPI_Transactions}
The SPI communications standard is somewhat loosely defined, and allows for synchronous communication between Master and Slave. The Master may, at any time, simultaneously send and receive data provided that the Slave has been given sufficient time to load data into a transmit register and store data from a receive register. The fundamental problem being that, the Slave cannot initiate transactions, making it difficult for Slave modules to dictate the flow of a measurement procedure. In order to facilitate this, OOPI implements an atomic transaction protocol, with a guaranteed series of events which allows for complex message passing between Master and Slave. The flow of control of an OOPI atomic transaction is shown in \cref{fig:OOPI_Transaction}, consisting of: an initial handshaking exercise to verify proper functioning of the hardware; followed by a request made by the Master of the Slave; and concluded by a reply made by the Slave to the Master. The utility of this protocol is that the requests and replies are universally defined structures, namely mCmd and sCmd, which contain instances of an enumerated instruction set with string, integer and float parameters. This allows the Master to issue human-readable commands to the Slave as per usual, as well as for the Master the request the Slave to issue commands to the Master itself, such as display an instruction to the user. The Slave may also reply with an identity structure or compound data structure to convey measurement data of arbitrary form. 

\paragraph{General sensor modelling}
\label{sec:OOPI_General_Sensor}
Sensors of the type which are of interest to this project can generally be modelled as requiring the enactment of a measurement procedure, which may require participation by the user. In particular, measurement procedures are thought of as a set of steps, at each step the sensor is entitled to issue an instruction to the Master, the user, or both; and may require the user to confirm conformance to the instruction via the Master. The control flow of the current implementation is given in \cref{fig:OOPI_Measurement}. When the user requests a measurement to be enacted, the Master will request the total number of steps/instructions in that sensor's measurement cycle. The Master will then loop through that cycle, carrying out the instructions issued by the sensor at each stage, which include display messages to the user, waiting for user input, and pausing the program. The sensor firmware itself requires a slightly greater complexity due to its allowed flexibility; transactions initiated by the Master trigger an interrupt routine in the Slave, which parses the request made by the Master, alters the global state of the Slave and replies appropriately. In the main loop, the sensor will progress though its measurement cycle in response to changes in state, thus allowing the sensor to dictate the measurement procedure from within a hardware architecture whereby only the Master can initiate transactions. Furthermore, this modelling exercise allows the same firmware to be utilised for any sensor connected to the Master module.

    \begin{figure}[h!]
        \centering
        \includegraphics[width=0.85\textwidth]{Pictures/Hardware/OOPI_Measurement_Reduced.png}
        \caption{Activity diagram illustrating the flow of control within the OOPI framework for a measurement instance. Red arrows indicate synchronised communication events, blue arrows indicate the flow of information, black arrows indicate the flow of control and green arrow indicate Actor influence. }
        \label{fig:OOPI_Measurement}
    \end{figure}%
    

\paragraph{Object oriented sensors}

\begin{figure}
    \centering
    \includegraphics[width=0.6\textwidth]{Pictures/Hardware/OOPI_Class_Diagram.png}
    \caption{Class Diagram of the sensor abstraction used by the Master Module}
    \label{fig:OOPI_Class_Diagram}
\end{figure}

In order to enforce non-specificity of the Master code, the \gls{OOP} has been adopted to encapsulate the communications protocol and provide an interface, which enforces the model envisioned in \cref{sec:OOPI_General_Sensor}. The encapsulation provided by the \gls{OOP} coding allows the firmware and the library to become maximally decoupled, which in turn, allows the implementation of OOPI to be maintained and updated without interfering with the function or form of any implemented firmware. An additional feature of OOP is inheritance, which allows for polymorphic code. In this instance, functional main code can pass the sensor object as a parameter to functions accepting instances of one of the base classes; the sensor object will then be trimmed, allowing that function access to only the portion of the interface provided by said base class. This strengthens the modelling paradigm and enforces a separation of concerns as functions cannot stray from their designated responsibility by accessing methods beyond said responsibility. 

In particular, a sensor is modelled as an entity that can provide data, has an identity which can be queried, issue and be issued commands. Each aspects of the complete abstracted sensor model are modelled by each of the base classes. In addition, each base class contains a communicative object, which encapsulated the atomic OOPI transaction protocol described in \cref{sec:OOPI_Transactions}.
%\includepdf[pages=-]{Pictures/Hardware/OOPI_Hyperlinked.pdf} 

\tocless \subsection{C-3NpH} 
\label{C3nph_meth}

%\subsubsection{Evaluation of different test strips for C-3NpH} \label{c3nph:teststrips}


\subsubsection{Note on data analysis}
For all graphs, standard error was calculated by $\sigma/\sqrt{S}$, where $\sigma$ is standard deviation and $S$ is sample number. It was based on a minimum of 3 strips in 3 replicate samples. 

\subsubsection{Nitrates} \label{Nitrateproto}

\paragraph{\gls{PR} model construction based on standard solutions} 
%\begin{itemize}[leftmargin=*,label={}]
\subparagraph{Preparing nitrate standard solutions}
\noindent A 1000 ppm nitrate stock solution was made by dissolving %13.7 mg% 
sodium nitrate (NaNO$_3$, Merck, USA) in %10 \\ml% 
\acrshort{UHP}; this was subsequently diluted into desired concentrations by addition of %more
\acrshort{UHP} to %create 
afford standard solutions. Concentrations used in the final calibration model were: 10, 15, 25, 40, 50, 75, 100, 150, 200 and \SI{250}{ppm}. Triplicate nitrate strips (DCS Products, UK) were dipped in each respective standard solution for \SI{2}{s}, and left to develop for \SI{3}{\\minuteute} before insertion into C-3NpH. %The colours of the strips darken with time; hence drying time was optimised to ensure that accurate all readings collected were reliable. Fig.\ref{subfig:no3_1} depicts the gradient observed in the colour strips, at different nitrate concentrations.  
\subparagraph{PR model} \label{ratio}
\noindent For each nitrates standard solution, triplicate C-3NpH readings were collected. Due the fluctuation in raw RGB output from C-3NpH, a ratio between the red and green channels ($\frac{R}{R+G}$) was adopted for analysis instead of the raw RGB values to give more robust prediction. %Collected
Data was separated into training set (66\%) and validation set (33\%). Using the \textit{Numpy} package in Python, different degrees of the polynomial model was applied on the training set; model performance was then evaluated on the validation set. Results showed that a third degree polynomial model gave the best performance. 
%\end{itemize}

\paragraph{In laboratory testing with soil}
%\begin{itemize}[leftmargin=*,label={}]

\subparagraph{Testing of different solvents on Wolfson College soil} 
%\item\textit{\textbf{{Testing of different solvents on Wolfson College soil}}}
\noindent Field-moist soil from Wolfson College and a given solvent were added to a falcon tube in equivalent weights. Two solvents, \SI{2}{M} KCl (industrial standard) and \SI{2}{M} NaCl were tested. The falcon tube was shaken manually for two minutes, before leaving to settle. Settling to give a clear, %dippable
supernatant occurred after \SI{15}-\SI{45}{\minute}. \SI{0.5}{\ml} of the supernatant was siphoned into an eppendorf,% to allow for strip dipping to be read by
a strip was dipped then inserted into C-3NpH (in accordance to the protocol given in Section \ref{Nitrateproto}). For testing in the HACH spectrophotometer, \SI{2}{\ml} was first extracted into a chloride-elimination syringe (LCW925, HACH, USA), as chloride interferes with the final colorimetric reading. \SI{1}{\ml} of the filtered supernatant was then added to a nitrates test cuvette (LCK339, HACH, USA) along with \SI{0.2}{\ml} of nitrate-reducing solvent, 2,6-dimethylphenol. The cuvette was shaken to ensure equal dispersion of liquid and left to develop for \SI{2}{\minute}, before the cuvette was inserted into the spectrophotometer (DR3900, HACH, USA). 


\subparagraph{C-3NpH precision test}
\noindent This was conducted on Wolfson Soil with 2M KCl by dipping 10 nitrate strips in each of three sample preparations following the aforementioned procedure.

\subparagraph{Testing on different soils and compost}

\noindent Further soil samples were collected from Cherry Hinton and Churchill College; these were tested alongside two compost samples. Protocol follows that given in the above, using only \SI{2}{M} NaCl used.  HACH spectrophotometer measurements were performed for soil samples only, and not compost. 
%\end{itemize}

\paragraph{Field testing}
C-3NpH was used to test for soil nitrates in Radwell Grange Farm and Anglesey Abbey, alongside the Moisture Retention Probe. An instructional video can be found at \href{https://universityofcambridgecloud-my.sharepoint.com/:v:/g/personal/eb729_cam_ac_uk/EUzpSPU2YTRGkFbAFTVqIYkBwqILsmehg1pN4DrM1jGbfg?e=8rYsFH}{online by clicking here}.


\subsubsection{pH Testing}
\paragraph{\textit{k-NN model construction based on pH solutions}} \label{phproto}
%\begin{itemize}[leftmargin=*,label={}]

\subparagraph{Preparing pH solutions} 

%\item\textit{\textbf{{Preparing pH solutions}}}
\noindent Solutions of a range of pH (pH 4.8, 5.4, 5.7, 6.2, 6.4, 7.2, 8.4, 9.1) were created by diluting Tris buffer (0.05 M, pH 9.1, Merck, USA) with %supermarket-bought
distilled vinegar (pH 2.4). A hand-held pH meter (HI 98107, HANNA Instruments) was used to measure solution pH, adding vinegar until the desired pH. pH strips (Hydrion, pH 1-14, Cole-Parmer) were dipped in triplicate in each standard solution for two seconds, and inserted immediately into C-3NpH.

\subparagraph{\gls{KNN} model} 

%\item\textit{\textbf{{\acrshort{KNN} model}}}
\noindent For the training set, RGB values and their corresponding pH values were stored. The distance between the input and training set data was calculated in Euclidean distance, and each value of the k nearest neighbour weighted by its distance from the input. Upon optimisation, $k = 3$ was selected, therefore, the prediction was based on the weighted average value of the three nearest data points in the training set. 


%\end{itemize}

\paragraph{In laboratory testing of soil}
%\begin{itemize}[leftmargin=*,label={}]


\subparagraph{Precision test of C-3NpH}

%\noindent Brown soil from Madingley Mulch and 0.01M CaCl$_2$ solvent were added to a falcon tube in equivalent weights. The falcon tube was hand shaken for 2 \minute, before leaving to settle; the latter occurring at 15-45 \minute. 0.5 \ml of the supernatant was siphoned into an eppendorf, to allow for strip dipping to be read by C-3NpH (in accordance to the protocol given in Section \ref{phproto}). To test the precision of C-3NpH, samples of the same soil was prepared thrice into separate falcon tubes and 10 strips were dipped into each eppendorf. 

%\todo{Chyi - speak with Katie}

% normally in chemistry Id write experimental like this: 
To \SI{1}{g} of Brown Soil (Madingley Mulch), an equivalent volume of CaCl$_2$ (0.01 M, \SI{9}{\ml}). This was manually shaken (\SI{2}{\minute}) before leaving to settle for 15-\SI{45}{\minute}. The resultant supernatant ($\sim$\SI{0.5}{\ml}) was transferred to an eppendorf and the pH strip dipped. Once dipped, the strip was inserted into C-3NpH and the pH measured as previously described (\ref{phproto}). Measurement precision was evaluated using three sample preparations as just described and 10 replicate strip measurements per sample.

%	\item\textit{\textbf{{Soil spiking experiment}}}

\subparagraph{Spiking experiment}
\noindent Spiking solutions were prepared for pH 4-9, intervals of pH 1, by adding \acrshort{UHP} water to Tris buffer (\SI{0.05}{M}, pH 9.1). %The same 
Madingley Mulch brown soil was prepared %in the manner of 
as for the precision test above. However, % again id write the rest as follows: 
to the supernatatnt (\SI{0.5}{\ml}) was added an equivalent volume of spiking solution (\SI{0.5}{\ml}) before the solution was shaken manually to mix and strips dipped in triplicate. 
%0.5 \\ml of spiking solution was added to the same volume of supernatant, separately for each different pH, before strip dipping.  

%\end{itemize}

\paragraph{Cambridge Immerse Summer School} \label{Summer_C-3NpH}
Summer school students were divided into four groups, %each in possession of a
and given a C-3NpH prototype. Users were instructed to assemble C-3NpH from its 3D printed components, and test the pH models on different household solutions. Instructions given to the user are summarised in the worksheet provided \href{https://universityofcambridgecloud-my.sharepoint.com/:f:/g/personal/eb729_cam_ac_uk/EtEn0MBg-2ZIjL1_mABHpa4BmyKOBFYg7WaYXhi8zPZieQ?e=zcijNp}{online by clicking here}.

\subsubsection{Soil separation methods}
\paragraph{Separation methods}
%Initial experiments focused on the separation efficiency of the WG in comparison to either bench-top centrifugation or settling. Additional experiments to decide solvent choice, centrifuge speed, WG time, WG/settling/bench-top centrifuge and mixing time were carried out as described in the following typical protocol.

For each of the qualitative experiments a general protocol was followed and outlined below. Deviations from the general procedure have been detailed in results and discussion, highlighting the variable of interest.

Soil for testing separation methods was obtained from Grange Farm, Lolworth. %, on the 27th June 2019. 
The collected soil was stored at room temperature in plastic sample bags provided on site. Soil as collected was mixed with the desired volume of extraction solvent (0.5 M NaCl, 2 M NaCl, KCl or CaCl$_2$) in either a 1:1, 1:4 or 1:9 soil-solvent ratio, and shaken by hand for a minimum of \SI{2}{\minute}. For WG separation, an aliquot ($\sim$\SI{1}{\ml}) of the mixed soil sample was transferred to an eppendorf (\SI{1.5}{\ml}) and placed in the WG which was then manually spun at maximum speed for \SI{2}{\minute}. A colour strip was then dipped in the resultant supernatant and read by C-3NpH after the allotted development time as indicated in previous sections. Settled samples were mixed as before, except the aliquot was left to stand until an appreciable volume of supernatant separated from sediment such as to allow a strip to be dipped (typically $<$\SI{30}{\minute}). Samples separated on the \acrfull{BC} were, unless otherwise specified, spun at a speed of 2000 rpm for \SI{20}{s} and the supernatant tested as before.

\paragraph{Time-course experiment} \label{c3nph:mixing}

Advice from industrial protocols described a mixing time of between \SI{15}{\minute} and \SI{24}{hours} and therefore time course experiments were conducted, using both the WG and bench-top centrifuge. Sample preparation proceeded as above except that mixing times were extended to \SI{15}{\minute}, \SI{30}{\minute} and \SI{24}{hours}, after manual shaking. At each time point, an aliquot was taken and either WG or centrifuged as before.

\paragraph{Comparison to industrial standards} \label{c3nph:nacl}
\Cref{subfig:solventind_1to9} shows that \SI{0.5}{M} NaCl read by C-3NpH at a dilution of 1:9 soil-solvent ratio best agreed with visual assignment of pH using two alternative brands of narrow range strips (i.e. Cole-Parmer, USA and Johnson, UK). This was surprising given the industrial preference for CaCl$_2$. However, the correlation of NaCl with CaCl$_2$ is expected to differ for different soil types, hence will have to be considered for the future development of this project. However, it needs to be considered that the strips may not be representative of the true pH if their chemistry was designed to read pH in water and not salt solution. Therefore, this is another avenue that requires further exploration. 

\begin{figure}[h!]
	\centering
	\begin{subfigure}[b]{\linewidth} 
		\centering
		\includegraphics[width=12cm]{Pictures/C-3NpH/solventindustrial_1to4.png}
		\caption{}
		\label{subfig:solventind_1to4}
	\end{subfigure}
	%\vfill
	\begin{subfigure}[b]{\linewidth}
	\centering
		\includegraphics[width=12cm]{Pictures/C-3NpH/solventindustrial_1to9.png}
		\caption{}
		\label{subfig:solventind_1to9}
	\end{subfigure}
	\caption{Comparison of the separation ability and pH by eye of three solvents, i.e. \gls{UHP} (green), \SI{0.01}{M} CaCl$_2$ (red) and \SI{0.5}{M} NaCl (yellow). Samples were prepared in either (a) 1:4 or (b) 1:9 soil-solvent ratio.}
\label{fig:solventindustrial}
\end{figure}   	

% To address separation efficiency and the potential 'brownness' effect, centrifuge speed and the subsequent influence on the pH of supernatant was explored as proxy for 'brownness' on the pH reading by the C-3NpH.

\paragraph{Validating soil sampling with the BC}
It was of interest to see if BC speed impacted the value of pH recorded by C-3NpH. \cref{subfig:speed} illustrates four eppendorfs (prepared with \SI{0.5}{M} NaCl, 1:9 ratio) spun by a \gls{BC} at velocities ranging between 2000-6000rpm (i.e. to the maximum reported speed of the 3D-fuge \cite{Byagathvalli2019}). There was no apparent difference visually with the supernatant colour; and a speed of \SI{2000}{rpm} appeared sufficient to achieve the desired colourless supernatant. All speeds gave a consistent pH reading when measured with duplicate strips in C-3NpH. At \SI{6000}{rpm}, NaCl was shown to form a pellet at \SI{15}{s}, but the equivalent with \gls{UHP} required \SI{1}{\\minuteute} (\cref{subfig:WGBC}). This provided impetus to chose NaCl as the extractant.

\paragraph{Proposed alterations} \label{appen:alter}

%Clearly across climates and daily weather conditions, according to Schmidhalter, calibration and correcting factors should be incorporated before reporting final nitrate concentrations. The final point to consider from Schmidhalter's work is the use of a gravimetric measuring method to measure soil water content negating the need for drying should be incorporated into the recommended protocol for future use of C-3NpH.
In this work, qualitative assessment of separation efficiency has been conducted, and provided the foundations for future development. However,it is acknowledged that soil drying and sieving prior to analysis will impact results. These two variables were not tested in this work, leaving scope for future testing. This is in particular to comparing impact of soil preparation on the ability of C-3NpH to reproduce reliable results based on supernatant colour. At this stage, the range of solvents and conditions explored were insufficient to properly extract exchangeable protons from soil samples. Therefore, pH measured each time was a representation of the baseline solvent pH. This was confirmed visually, when strips dipped in pure solvent produced indistinguishable colour changes to that dipped in extracted soil supernatant. 


\newpage
\subsubsection{ChemTest results} \label{Chemtest}
\begin{figure}[h!]
	\centering
	\includegraphics[width=14cm]{Pictures/C-3NpH/chemtest1.png}
	\end{figure}
	\begin{figure}[h!]
	\centering
	\includegraphics[width=14cm]{Pictures/C-3NpH/chemtest2.png}
	\end{figure}
	\begin{figure}[h!]
	\centering
	\includegraphics[width=14cm]{Pictures/C-3NpH/chemtest3.png}
	\end{figure}
	\begin{figure}[h!]
	\centering
	\includegraphics[width=14cm]{Pictures/C-3NpH/chemtest4.png}
	\end{figure}   
	
	
	
	\clearpage
\tocless \subsection{Moisture Retention Probe}
\subsubsection{Design Evolution of the Moisture Retention Probe}
\label{smrdesignevolution}
As seen in \cref{fdesignevolution}, the design process of the moisture retention probe was one of iterative improvements that can be likened to \emph{rapid prototyping}. Each of the probes shown there was field-tested to identify its strengths and weaknesses.

The first test on the left-most side was done with a commercial moisture sensing PCB for the maker community. Its sensing elements are imprinted as traces on the PCB, making the unit self-contained. They are not intended to be embedded into the ground, as the top part containing the readout electronics are fully exposed to the elements (the design is mostly used with potted plants as a drought alarm, and only partially inserted). Hence, to run a full field test, the electronic components were encased in two-component epoxy, taking care to cover the sensing elements with only a thin film, and a long, robust cable was soldered on. This was used to proof-of-concept whether manual irrigation through tens of centimetres of soil would lead to significant signal change, which was confirmed with signal changes above \SI{30}{\percent} with modest (dL) irrigation volumes.

Next, the design moved to a cylindrical design with coiled wires at several different depths, which would remain as the main design form factor. Each pair of wires was soldered onto a commercial moisture sensing PCB which was cut to keep only the part containing the readout electronics. Magnet wire was used initially as it is easy to coil, and therefore can give strong signal with many windings. Downsides of this approach were the low robustness of this small wire, and the difficulty in fixing the wire to the rod. Superglue was found to not adhere strongly enough; two-component epoxy was difficult to apply evenly and thinly enough so as to not shield the sensing elements too much; and gaffa tape or heat shrink was found to wick in moisture.

Note that these earliest designs use serological pipettes merely for their geometry of a hollow rod (for routing the wires above ground) with a conical tip (to drive them into the ground). The reason for this was simply that they were easily available from CEB stores in a variety of diameters. Serological pipettes were found to fail mechanically during field testing. In dry soil especially, the stress applied when removing the probe led to the device breaking apart.  They were thus replaced by commercially available polyethylene barrier pipes. Silicone sealant and epoxy were used to waterproof the final device especially with regards to the holes made to pass the wires through. 

In early designs the coils of paired wires were placed with random spacing and inevitably came in close contact with each other. Cylindrical wire holders with holes both at the top and the bottom were 3D printed to allow for controlled and reproducible placement of the wires on the probe. Several spacing parameters were explored both between the holders themselves and for the coil tracks on the holders. However it was found that constant spacing between the wires provided the most reliable results. The 3D holders can easily be slipped onto the pipe and then fixed at the desired position with two part epoxy. The length of wire placed in the tracks of the 3D holders was stripped of its insulating layer and glued in place. This led to a significant improvement in the sensitivity and signal to noise ratio of the probes. The rest of the wire was kept insulated and placed the connection to the moisture sensor peripheral was protected with heat shrink. 


Apart from the probe inserted into the ground, the readout electronics were evolved from hand-soldering on a strip-board to a custom PCB. The PCB was sized such that it slots into one of the cooling fins of the aluminium enclosure used to provide a sturdy case to interface with the base unit.

\subsubsection{Instruction Protocol}
\label{smrinstructionprotocol}

\textbf{Base unit (main box with touch buttons):  }

\textit{To charge the base unit plug the USB into the top socket. The device should be switched off whilst charging. The other end of the USB can go into any device to charge  (i.e: a PC). Red LEDs indicate charging and blue LEDs indicate that the device is fully charged. Charge after around 2 days of use.} 

\textit{Only use the blue USB cable}  


\textbf{Moisture retention probe (black box with capacitive probe)} 

\todo{jh2109 I suggest to change all messages displayed on the screen to texttt, as I did for one below}
\begin{itemize}
\item Plug the blue USB cable into the black box of the moisture retention probe gently.
\item Switch on the main box (base unit) 
\item Once the main menu is displayed, plug in the other end of the blue USB cable into the lower USB port of the base unit.  
\item Select the “Water Ret.” option. By touching the left most button
\item If it shows “check connection” keep everything plugged in. Switch off and switch on the main box again    
\item Display should show \texttt{Moisture probe found. Serial no. 1}
\item Make a hole the size of the probe in the soil using a metal rod or any suitable instrument 
\item Gently push the probe down the hole until all 3 sensors are covered 
\item Make a secondary hole circa. \SI{5}{cm} deep, circa. \SI{5}{cm} wide and \SI{3}{cm} to \SI{5}{cm} away from the probe 
\item Ensure water cannot wick vertically along the probe 
\item The display shows “Insert probe Press Select”. Once the probe is inserted into the soil, touch SELECT button (left button).  
\item Next, soil wetness values are displayed. This display waits for \SI{5}{seconds}
\item Next, the display shows “Irrigate Probe. Measurement in progress”. Now fill the secondary hole with water keeping it topped up regularly. Keep on doing this until values are displayed 
\item Water retention scores are displayed. The display waits for \SI{5}{seconds} 
\item Data is logged to SD card. 
\end{itemize}

%\subsection{Extracting Soil Parameters from Probe Output} %Author: Doug


\subsubsection{Extracting Soil Parameters from Probe Output}
\label{sec:Extracting_Soil_params}
The measurement enacted by the moisture probe is rich in information; presented here are methods to extract the infiltration rates and the Hydraulic Conductivity of the upper-most layer of soil.
\newline
A simplified model for the infiltration of water into the soil from the water reservoir is that of radial dispersion under the influence of diffusion and gravity. This model is illustrated by \cref{fig:moisture_retention_sketch} and assumes the soil to be homogeneous. By considering the sensors and the reservoir to each be point elements in space, as illustrated in \cref{fig:moisture_retention_schematic}, it becomes clear that the volume of water which causes the primary deflection in each sensor must travel along some radial path with path length $\Delta r_i$ over the time interval $\Delta t_i$. The task is thus to derive the vertical infiltration velocity, $\frac{dy}{dt} \equiv \dot{y}$ as seen by each sensor. The argument follows thusly:

%\begin{gather}
\begin{align}
    y_i &= r_i\cdot sin\theta_i \\
    \implies \frac{dy_i}{dt_i} &= \big(\frac{dr_i}{dt_i}\big)\cdot sin\theta_i \\
    \intertext{But,} 
    \frac{dr_i}{dt_i} &\approx \frac{\Delta r_i}{\Delta t_i} \\
    \implies \dot{y}_i &\equiv \frac{dy_i}{dt_i} \approx \frac{\Delta r_i  \cdot sin\theta_i}{\Delta t_i}\\
    \intertext{Where,}
    \Delta t_i &= x_{sw} \cdot cos\theta_i \\
    \theta_i &= tan^{-1}\big(\frac{y_i}{x_{sw}}\big)\\
    \implies \dot{y}_i &\approx \frac{x_{sw} \cdot sin\theta_i\cdot cos\theta_i}{\Delta t_i}\\
    \intertext{But,}
    sin\theta_i \cdot cos\theta_i
    &= \frac{tan\theta_i}{1+tan^2\theta_i}\\
    &= \frac{y_i/x_{sw}}{1+(y_i/x_{sw})^2}\\
    &= \frac{y_i\cdot x_{sw}}{x^2_{sw}+y^2_i}\\
    \implies \dot{y}_i &\approx \frac{y_i\cdot x^2_{sw}}{\Delta t_i\cdot(x^2_{sw}+y^2_i)} \qed
\end{align}
%\end{gather}

\begin{figure*}[h!]
\begin{subfigure}[b]{.75\textwidth}
  \centering
  \includegraphics[width=\textwidth]{Pictures/moisture_retention/Moisture_probe_sketch.png}
  \caption{Stylised sketch of the moisture retention probe and water reservoir with radial infiltration of water from reservoir into soil.}
  \label{fig:moisture_retention_sketch}
\end{subfigure}%
\vspace{10mm}
\begin{subfigure}[b]{.75\textwidth}
  \centering
  \includegraphics[width=\textwidth]{Pictures/moisture_retention/Moisture_probe_Schematic.png}
  \caption{Schematic illustrating the geometric relationships between the sensors and water reservoir, all modelled as point entities.}
  \label{fig:moisture_retention_schematic}
\end{subfigure}
\caption{Figures illustrating the simplified radial infiltration model.}
\label{fig:Moiture_retention_Radial_infiltration}
\end{figure*}

Now, a comprehensive equation describing the vertical infiltration rate has been proposed by Ogden~\textit{et al.}~\cite{soilmoisturevelocityogdenetal2017} and is given as:
\begin{equation}
    \frac{dy}{dt} = -K\cdot \bigg[\frac{\partial \Psi}{\partial y}-1\bigg] - D\cdot \bigg[\frac{\partial^2\Psi/\partial y^2}{\partial\Psi/\partial y}\bigg]
    \label{eqn:Soil_moisture_velocity_eqn}
\end{equation}
where:
\begin{conditions}
 K     &  Unsaturated Hydraulic Conductivity $[m/s]$ \\
 \Psi  &  Capillary Pressure Head $[m]$\\   
 D     &  Soil Water Diffusivity $[m^2/s]$
\end{conditions}
A parameter of interest is the hydraulic conductivity as it measures the ease with which water percolates through the soil and is therefore a reflection of the soil porosity and interconnectedness. 
\newline
Now, in the context of rough estimates, the second term in \cref{eqn:Soil_moisture_velocity_eqn}, which describes the diffusion like behaviour of the water, can be regarded as negligible relative to the first term, which describes the advection-like behaviour and encapsulates the effect of porosity and gravity~\cite{soilmoisturevelocityogdenetal2017}.Thus,

\begin{align}
    \frac{dy}{dt} &\approx -K\cdot \bigg[\frac{\partial \Psi}{\partial y}-1\bigg]\\
    &= -K\cdot \frac{\partial \Psi}{\partial y} + K
\end{align}

Assume that water infiltrates radially, thus, by the Inverse Square Law:

\begin{gather}
    \Psi = \Psi(y) \propto \frac{1}{y^2}\\
    \implies \frac{\partial \Psi}{\partial y} \propto -\frac{1}{y} \\
    \implies \dot{y} \propto -K \cdot \big(-\frac{1}{y}\big) + K\\
    \implies \dot{y} \approx -C\cdot \frac{K}{y} + K \quad\text{Where C is a constant}\label{eqn:Approx_y_dot_Append}\\
    \intertext{Thus, from \ref{eqn:Approx_y_dot_Append}}
    \dot{y} \to K \quad\text{where}\quad y \gg 1\\
    \implies K_{Average} \approx \dot{y}_3\\
    \intertext{And, from \ref{eqn:Approx_y_dot_Append}}
    C \approx -(\dot{y}-K)\cdot\frac{y}{K}\\
    \implies C_{Approx} \approx -(\dot{y}_2 - K_{Average})\frac{y_2}{K_{Average}}\\
    \intertext{Finally, from \ref{eqn:Approx_y_dot_Append}}
     K \approx \frac{\dot{y}}{\bigg(\frac{C}{y}-1\bigg)}\\
    \implies K_1 \approx \frac{\dot{y}_1}{\bigg(\frac{C_{Approx}}{y_1}-1\bigg)}
\end{gather}

Which gives the Hydraulic Conductivity of the Top Soil layer, $K_1$, by using the deep sensors to estimate the interim parameters. It must, however, be noted, that the capillary head, hydraulic conductivity and diffusivity are all non-linear functions of the volumetric water content of the soil. Given that the magnitude of the signal generated by the sensors is correlated to this parameter, adjustments to the measuring procedure using different volumes of water could be used to generate the appropriate functions.














 

 
\section{Software} 
\tocless \subsection{Schema structure} \label{schema_struct}

Given the choice of a relational database, the schema had to be fixed. For the \gls{mvp}, the schema was designed to contain the following columns: 
\begin{itemize}
    \item \texttt{id} -- an autoincrementing integer, the number of the stored measurement;
    \item \texttt{device\_id} -- an integer signifying the main unit used for the measurement;
    \item \texttt{datetime} -- date and time of the in-field measurement;
    \item \texttt{gps\_lat} -- latitude recorded by the GPS unit;
    \item \texttt{gps\_lng} -- longitude recorded by the GPS unit;
    \item \texttt{nitrate} -- the nitrate concentration recorded in the field;
    \item \texttt{ph} -- pH recorded in the field;
    \item \texttt{moisture\_retention} -- moisture retention single value recorded in the field;
    \item \texttt{co2} -- CO$_2$ content obtained from the lab, added manually by the user later;
    \item \texttt{carbon} -- organic carbon content obtained from the lab, added manually by the user later.
\end{itemize}



%----------------------------------------------------------------------------------------
%	INDEX
%----------------------------------------------------------------------------------------

\cleardoublepage
\phantomsection
\setlength{\columnsep}{0.75cm}
\addcontentsline{toc}{chapter}{\textcolor{ocre}{Index}}
\printindex

%----------------------------------------------------------------------------------------

\end{document}
