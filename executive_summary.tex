In collaboration with the \gls{CGE}, this year’s Sensor CDT Team Challenge was proposed to address soil erosion due to topsoil loss in the Umzimvubu Catchment Area, South Africa. 
%%Stretching 200 km from its source in the Drakensberg Mountains to where it meets the Indian Ocean, the Umzimvubu river flows in parallel to the northern boundary of the Eastern Cape. It has a biodiverse mosaic of ecosystems, which supports more than 70 vulnerable species of flora and fauna.
In recent decades, this former Transkei homeland has seen significant topsoil erosion, primarily due to farming malpractices and the invasion of alien plant species. Soil erosion, however, is not restricted to South Africa: it is reported that the equivalent of 30 football fields of fertile topsoil is lost due to erosion, every day across the world. \cite{saveoursoils} This is primarily driven by the pressures of rapid urbanisation and socio-economic development, for which Umzimvubu forms a classic example. Noting that 99.7$\%$ of our food comes directly from the soil \cite{saveoursoils}, the severe implications of topsoil loss become forefront and highlight the global pertinence of this project. 

SoliCamb has developed a simple, low-cost, modular platform for quantitative measurements indicative of soil quality to be made in-field; this, in turn, enables rapid evaluation of mitigating practices at a local level. The base unit provides the display for the user; it is equipped with data-logging, and time and GPS stamping capabilities. Moreover, it allows the user to attach different peripheral sensors; the plug-and-play nature of the platform leaves scope for further addition of sensors in the future. Currently, two peripheral sensors (i.e. the moisture retention probe and C-3NpH) have been developed. Designed based on the request from the \gls{UCPP} for a measure of soil erosion, the moisture retention probe consists of capacitive sensors positioned at different depths on a rod; it gives the end user a moisture score indicative of irrigation requirement. On the other hand, C-3NpH is a colour sensor that provides qualitative analysis for nitrates and pH test strips; it is aimed towards a UK-based end user, interested in measuring soil health for crop-growing. Design considerations have been made to ensure the affordability (the whole system costs <\textsterling$70$, with the base unit making up 70$\%$) and user-friendliness of the final system. This is in line with the fact that SoliCamb aims to bridge the gap between laboratory and field measurements.  Acknowledging that some measurements (e.g. soil organic carbon) are unfeasible for field measurements, RFID-tagged bags are included to enable the end user to keep track of measurements when sent off for laboratory testing. An online web application with a ubiquitous Google Maps interface has been developed in conjunction; this allows users to input, access, and share data with ease.  Moreover, the open-source setup of the project means that users can access source code to better understand the science behind the system, and potentially build their own sensors and collaborate with SoliCamb. For validation of use, field testing with the complete system was performed in sites around Cambridgeshire, and collected data (including GPS, moisture score, nitrate levels) was successfully uploaded onto the online web application. 

Corroborating with the open-source philosophy, a defining aspect of this project was citizen engagement. SoliCamb participated in and organised events with Agri-Tech East and Cambridge Makespace respectively, where invaluable feedback for the ergonomic design of the base unit and the moisture retention probe was received. Moreover, C-3NpH was tested by students at the Immerse Cambridge Summer School, where it was found that there is minimal variability between different sensors. A host of advertisement strategies were used to promote the project. Within the Department, this consisted of circulating a weekly newsletter, handing out posters and business cards, and broadcasting a television advertisement. To engage the local Cambridgeshire community in soil health, SoliCamb was featured in a medley of media, which ranged from newsprint to radio and television. Moreover, online presence was established through social media and the website; this enabled outreach to a wider audience, with the website impressively receiving views from 10 different countries. 

Over the last 12 weeks, SoliCamb has forged a strong network consisting of academic and industrial collaborators, citizen scientists, and soil health-related charities. This will be useful for the future development of the project and the team; discussions have begun for potential expansion into real-time monitoring, smart agriculture, global development, and many other exciting avenues. \\

\begin{flushright}
\textbf{Team SoliCamb}
\end{flushright}


\begin{figure}[htb]
\centering
\includegraphics[width=0.4\linewidth]{Pictures/title_page/logo_noborder.png}
\end{figure}

