The aim of the project and SoliCamb's goal was to develop low cost, open source sensors for monitoring the key indicators of soil health. Great progress has been made in achieving this with a first
prototype designed and built, as well as two sensor peripherals being developed and validated. 
The project evolved considerably during the 10-week period from a device aimed purely at developing countries or resource limited markets, into a device  that has real potential for adding value to how those in the developed world agriculture and farming industries manage their land. 

Ultimately, a base unit, two peripheral sensors and associated data management platform were developed. 
The base unit serves the purpose of offering a user interface as well as storing geographically and time stamped data produced by the peripherals. 
The first peripheral consisted of a probe that can be inserted into the ground and irrigated with water to measure moisture infiltrative properties of the ground in question, based on feedback from our collaborators this peripheral seems the most valuable as a tool within agriculture and land management although there are clearly still some technical challenges and questions in the implementation.
The second peripheral was a nitrate and pH sensor. This was built to fit the nitrate test strips as a cheap and easy alternative to spectrophotometry. The device, C-3NpH, was pivoted at two separate audiences. The ease with which the model and RGB colour sensor could detect changes in colour gradients meant that C-3NpH was successfully deployed in field at two test sites within the Cambridgeshire area. On the other hand, the pH model was limited by the strip resolution in addition to the complexities in distinguishing between colours and not just a single gradient as with nitrate. Therefore, pH sensing was used as an educational tool and demonstrated to local citizen scientists the benefits of quantitative measurement, sensor design and electronics with Arduino. Both facets of C-3NpH demonstrated success and are poised to move forward, using optimised soil sampling procedures. 

A key deliverable was to facilitate data management, initially for those in UCPP, but the added value of combining a web application with our hardware in the UK quickly became apparent. To this end a  web application was constructed that allows the user to store and monitor data recorded across locations and over time. Data can be uploaded manually using simple csv files in a pre-defined format by using the RFID tags that were designed to accompany the modular base unit.

Our outreach channels created a strong network of academic and industrial collaborators, with whom we will continue to work in the future. It was crucially important that SoliCamb expanded it's network beyond that of the University in order to best understand the context of soil health and where the device fit in this arena. 

The exact future of the project remains an open question. Given the momentum and attention we have generated there are multiple directions the project could be taken. Collaborators within charities, universities and industry are actively interested in the topic of open-source sensing, citizen empowerment and smart agriculture; encompassing both developing and developed countries. Several members of the team have already expressed interest in pursuing the project further. The next 6 months will be a pivotal time for SoliCamb as the networks and opportunities created over these 10 week are expected to come to fruition, along with proposals for the funding where SoliCamb stands in good stead to submit thorough and detailed applications. 
